%% Generated by Sphinx.
\def\sphinxdocclass{report}
\documentclass[letterpaper,10pt,english]{sphinxmanual}
\ifdefined\pdfpxdimen
   \let\sphinxpxdimen\pdfpxdimen\else\newdimen\sphinxpxdimen
\fi \sphinxpxdimen=.75bp\relax
\ifdefined\pdfimageresolution
    \pdfimageresolution= \numexpr \dimexpr1in\relax/\sphinxpxdimen\relax
\fi
%% let collapsible pdf bookmarks panel have high depth per default
\PassOptionsToPackage{bookmarksdepth=5}{hyperref}

\PassOptionsToPackage{booktabs}{sphinx}
\PassOptionsToPackage{colorrows}{sphinx}

\PassOptionsToPackage{warn}{textcomp}
\usepackage[utf8]{inputenc}
\ifdefined\DeclareUnicodeCharacter
% support both utf8 and utf8x syntaxes
  \ifdefined\DeclareUnicodeCharacterAsOptional
    \def\sphinxDUC#1{\DeclareUnicodeCharacter{"#1}}
  \else
    \let\sphinxDUC\DeclareUnicodeCharacter
  \fi
  \sphinxDUC{00A0}{\nobreakspace}
  \sphinxDUC{2500}{\sphinxunichar{2500}}
  \sphinxDUC{2502}{\sphinxunichar{2502}}
  \sphinxDUC{2514}{\sphinxunichar{2514}}
  \sphinxDUC{251C}{\sphinxunichar{251C}}
  \sphinxDUC{2572}{\textbackslash}
\fi
\usepackage{cmap}
\usepackage[T1]{fontenc}
\usepackage{amsmath,amssymb,amstext}
\usepackage{babel}



\usepackage{tgtermes}
\usepackage{tgheros}
\renewcommand{\ttdefault}{txtt}



\usepackage[Bjarne]{fncychap}
\usepackage{sphinx}

\fvset{fontsize=auto}
\usepackage{geometry}


% Include hyperref last.
\usepackage{hyperref}
% Fix anchor placement for figures with captions.
\usepackage{hypcap}% it must be loaded after hyperref.
% Set up styles of URL: it should be placed after hyperref.
\urlstyle{same}


\usepackage{sphinxmessages}
\setcounter{tocdepth}{0}



\title{SkinOptics}
\date{Aug 26, 2024}
\release{0.1.0}
\author{Victor Lima}
\newcommand{\sphinxlogo}{\vbox{}}
\renewcommand{\releasename}{Release}
\makeindex
\begin{document}

\ifdefined\shorthandoff
  \ifnum\catcode`\=\string=\active\shorthandoff{=}\fi
  \ifnum\catcode`\"=\active\shorthandoff{"}\fi
\fi

\pagestyle{empty}
\sphinxmaketitle
\pagestyle{plain}
\sphinxtableofcontents
\pagestyle{normal}
\phantomsection\label{\detokenize{index::doc}}


\sphinxAtStartPar
\sphinxstylestrong{SkinOptics} is a Python package with tools for building human skin computational models for Monte Carlo simulations of light transport, as well as tools for analyzing simulation outputs. It can also be used for teaching and exploring about Optical Properties and Colorimetry.

\sphinxAtStartPar
\sphinxstylestrong{SkinOptics} is under continuos development. New features may be available in the future.

\sphinxAtStartPar
Please remember that \sphinxstylestrong{SkinOptics} is released without any warranty of accuracy.


\chapter{API reference (Toolbox)}
\label{\detokenize{index:api-reference-toolbox}}
\sphinxstepscope


\section{skinoptics}
\label{\detokenize{modules:skinoptics}}\label{\detokenize{modules::doc}}
\sphinxstepscope


\subsection{skinoptics.utils module}
\label{\detokenize{01_utils:module-skinoptics.utils}}\label{\detokenize{01_utils:skinoptics-utils-module}}\label{\detokenize{01_utils::doc}}\index{module@\spxentry{module}!skinoptics.utils@\spxentry{skinoptics.utils}}\index{skinoptics.utils@\spxentry{skinoptics.utils}!module@\spxentry{module}}
\sphinxAtStartPar
Copyright (C) 2024 Victor Lima
\begin{quote}

\begin{DUlineblock}{0em}
\item[] This program is free software: you can redistribute it and/or modify
\item[] it under the terms of the GNU General Public License as published by
\item[] the Free Software Foundation, either version 3 of the License, or
\item[] (at your option) any later version.
\end{DUlineblock}

\begin{DUlineblock}{0em}
\item[] This program is distributed in the hope that it will be useful,
\item[] but WITHOUT ANY WARRANTY; without even the implied warranty of
\item[] MERCHANTABILITY or FITNESS FOR A PARTICULAR PURPOSE.  See the
\item[] GNU General Public License for more details.
\end{DUlineblock}

\begin{DUlineblock}{0em}
\item[] You should have received a copy of the GNU General Public License
\item[] along with this program.  If not, see \textless{}\sphinxurl{https://www.gnu.org/licenses/}\textgreater{}.
\end{DUlineblock}
\end{quote}

\begin{DUlineblock}{0em}
\item[] Victor Lima
\item[] victorporto@ifsc.usp.br
\item[] victor.lima@ufscar.br
\end{DUlineblock}

\begin{DUlineblock}{0em}
\item[] Release Date:
\item[] August 2024
\item[] Last Modification:
\item[] August 2024
\end{DUlineblock}

\begin{DUlineblock}{0em}
\item[] References:
\end{DUlineblock}

\begin{DUlineblock}{0em}
\item[] {[}WSS13{]} Wyman, Sloan \& Shirley 2013.
\item[] Simple Analytic Approximations to the CIE XYZ Color Matching Functions
\item[] \sphinxurl{https://jcgt.org/published/0002/02/01/}
\end{DUlineblock}
\index{biexp\_decay() (in module skinoptics.utils)@\spxentry{biexp\_decay()}\spxextra{in module skinoptics.utils}}

\begin{fulllineitems}
\phantomsection\label{\detokenize{01_utils:skinoptics.utils.biexp_decay}}
\pysigstartsignatures
\pysiglinewithargsret{\sphinxcode{\sphinxupquote{skinoptics.utils.}}\sphinxbfcode{\sphinxupquote{biexp\_decay}}}{\sphinxparam{\DUrole{n}{x}}\sphinxparamcomma \sphinxparam{\DUrole{n}{a}}\sphinxparamcomma \sphinxparam{\DUrole{n}{b}}\sphinxparamcomma \sphinxparam{\DUrole{n}{c}}\sphinxparamcomma \sphinxparam{\DUrole{n}{d}}\sphinxparamcomma \sphinxparam{\DUrole{n}{e}}}{}
\pysigstopsignatures
\sphinxAtStartPar
The biexponential decay function.

\sphinxAtStartPar
\(f(x) = a\mbox{ exp}(-|b|x) + c \mbox{ exp}(-|d|x) + e\)
\begin{quote}\begin{description}
\sphinxlineitem{Parameters}\begin{itemize}
\item {} 
\sphinxAtStartPar
\sphinxstyleliteralstrong{\sphinxupquote{x}} (\sphinxstyleliteralemphasis{\sphinxupquote{float}}\sphinxstyleliteralemphasis{\sphinxupquote{ or }}\sphinxstyleliteralemphasis{\sphinxupquote{np.ndarray}}) \textendash{} function variable

\item {} 
\sphinxAtStartPar
\sphinxstyleliteralstrong{\sphinxupquote{a}} (\sphinxstyleliteralemphasis{\sphinxupquote{float}}) \textendash{} function constant

\item {} 
\sphinxAtStartPar
\sphinxstyleliteralstrong{\sphinxupquote{b}} (\sphinxstyleliteralemphasis{\sphinxupquote{float}}) \textendash{} function constant

\item {} 
\sphinxAtStartPar
\sphinxstyleliteralstrong{\sphinxupquote{c}} (\sphinxstyleliteralemphasis{\sphinxupquote{float}}) \textendash{} function constant

\item {} 
\sphinxAtStartPar
\sphinxstyleliteralstrong{\sphinxupquote{d}} (\sphinxstyleliteralemphasis{\sphinxupquote{float}}) \textendash{} function constant

\item {} 
\sphinxAtStartPar
\sphinxstyleliteralstrong{\sphinxupquote{e}} (\sphinxstyleliteralemphasis{\sphinxupquote{float}}) \textendash{} function constant

\end{itemize}

\sphinxlineitem{Returns}
\sphinxAtStartPar
\begin{itemize}
\item {} 
\sphinxAtStartPar
\sphinxstylestrong{f} (\sphinxstyleemphasis{float or np.ndarray}) \textendash{} evaluated biexponential decay function

\end{itemize}


\end{description}\end{quote}

\end{fulllineitems}

\index{catenary() (in module skinoptics.utils)@\spxentry{catenary()}\spxextra{in module skinoptics.utils}}

\begin{fulllineitems}
\phantomsection\label{\detokenize{01_utils:skinoptics.utils.catenary}}
\pysigstartsignatures
\pysiglinewithargsret{\sphinxcode{\sphinxupquote{skinoptics.utils.}}\sphinxbfcode{\sphinxupquote{catenary}}}{\sphinxparam{\DUrole{n}{x}}\sphinxparamcomma \sphinxparam{\DUrole{n}{a}}\sphinxparamcomma \sphinxparam{\DUrole{n}{b}}\sphinxparamcomma \sphinxparam{\DUrole{n}{c}}\sphinxparamcomma \sphinxparam{\DUrole{n}{d}}}{}
\pysigstopsignatures
\sphinxAtStartPar
The catenary function.

\sphinxAtStartPar
\(f(x) = a \mbox{ cosh}\left[\frac{(x - b)}{c}\right] + d\)
\begin{quote}\begin{description}
\sphinxlineitem{Parameters}\begin{itemize}
\item {} 
\sphinxAtStartPar
\sphinxstyleliteralstrong{\sphinxupquote{x}} (\sphinxstyleliteralemphasis{\sphinxupquote{float}}\sphinxstyleliteralemphasis{\sphinxupquote{ or }}\sphinxstyleliteralemphasis{\sphinxupquote{np.ndarray}}) \textendash{} function variable

\item {} 
\sphinxAtStartPar
\sphinxstyleliteralstrong{\sphinxupquote{a}} (\sphinxstyleliteralemphasis{\sphinxupquote{float}}) \textendash{} function constant

\item {} 
\sphinxAtStartPar
\sphinxstyleliteralstrong{\sphinxupquote{b}} (\sphinxstyleliteralemphasis{\sphinxupquote{float}}) \textendash{} function constant

\item {} 
\sphinxAtStartPar
\sphinxstyleliteralstrong{\sphinxupquote{c}} (\sphinxstyleliteralemphasis{\sphinxupquote{float}}) \textendash{} function constant

\item {} 
\sphinxAtStartPar
\sphinxstyleliteralstrong{\sphinxupquote{d}} (\sphinxstyleliteralemphasis{\sphinxupquote{float}}) \textendash{} function constant

\end{itemize}

\sphinxlineitem{Returns}
\sphinxAtStartPar
\begin{itemize}
\item {} 
\sphinxAtStartPar
\sphinxstylestrong{f} (\sphinxstyleemphasis{float or np.ndarray}) \textendash{} evaluated catenary function

\end{itemize}


\end{description}\end{quote}

\end{fulllineitems}

\index{circle() (in module skinoptics.utils)@\spxentry{circle()}\spxextra{in module skinoptics.utils}}

\begin{fulllineitems}
\phantomsection\label{\detokenize{01_utils:skinoptics.utils.circle}}
\pysigstartsignatures
\pysiglinewithargsret{\sphinxcode{\sphinxupquote{skinoptics.utils.}}\sphinxbfcode{\sphinxupquote{circle}}}{\sphinxparam{\DUrole{n}{r}}\sphinxparamcomma \sphinxparam{\DUrole{n}{xc}}\sphinxparamcomma \sphinxparam{\DUrole{n}{yc}}\sphinxparamcomma \sphinxparam{\DUrole{n}{theta\_i}\DUrole{o}{=}\DUrole{default_value}{0.0}}\sphinxparamcomma \sphinxparam{\DUrole{n}{theta\_f}\DUrole{o}{=}\DUrole{default_value}{360.0}}\sphinxparamcomma \sphinxparam{\DUrole{n}{n\_points}\DUrole{o}{=}\DUrole{default_value}{1000}}}{}
\pysigstopsignatures
\sphinxAtStartPar
The parametric equation of a circle.

\sphinxAtStartPar
\(\left \{ \begin{matrix}
x = r \cos \theta \\
y = r \sin \theta
\end{matrix} \right.\)
\begin{quote}\begin{description}
\sphinxlineitem{Parameters}\begin{itemize}
\item {} 
\sphinxAtStartPar
\sphinxstyleliteralstrong{\sphinxupquote{r}} (\sphinxstyleliteralemphasis{\sphinxupquote{float}}) \textendash{} radius

\item {} 
\sphinxAtStartPar
\sphinxstyleliteralstrong{\sphinxupquote{xc}} (\sphinxstyleliteralemphasis{\sphinxupquote{float}}) \textendash{} center x\sphinxhyphen{}coordinate

\item {} 
\sphinxAtStartPar
\sphinxstyleliteralstrong{\sphinxupquote{yc}} (\sphinxstyleliteralemphasis{\sphinxupquote{float}}) \textendash{} center y\sphinxhyphen{}coordinate

\item {} 
\sphinxAtStartPar
\sphinxstyleliteralstrong{\sphinxupquote{theta\_i}} (\sphinxstyleliteralemphasis{\sphinxupquote{float}}) \textendash{} initial angle {[}degrees{]} (default to 0.)

\item {} 
\sphinxAtStartPar
\sphinxstyleliteralstrong{\sphinxupquote{theta\_f}} (\sphinxstyleliteralemphasis{\sphinxupquote{float}}) \textendash{} final angle {[}degrees{]} (default to 360.)

\item {} 
\sphinxAtStartPar
\sphinxstyleliteralstrong{\sphinxupquote{n\_points}} (\sphinxstyleliteralemphasis{\sphinxupquote{int}}) \textendash{} number of points (default to 1000)

\end{itemize}

\sphinxlineitem{Returns}
\sphinxAtStartPar
\begin{itemize}
\item {} 
\sphinxAtStartPar
\sphinxstylestrong{x} (\sphinxstyleemphasis{np.ndarray}) \textendash{} x\sphinxhyphen{}coordinantes of a circle

\item {} 
\sphinxAtStartPar
\sphinxstylestrong{y} (\sphinxstyleemphasis{np.ndarray}) \textendash{} y\sphinxhyphen{}coordinantes of a circle

\end{itemize}


\end{description}\end{quote}

\end{fulllineitems}

\index{cubic() (in module skinoptics.utils)@\spxentry{cubic()}\spxextra{in module skinoptics.utils}}

\begin{fulllineitems}
\phantomsection\label{\detokenize{01_utils:skinoptics.utils.cubic}}
\pysigstartsignatures
\pysiglinewithargsret{\sphinxcode{\sphinxupquote{skinoptics.utils.}}\sphinxbfcode{\sphinxupquote{cubic}}}{\sphinxparam{\DUrole{n}{x}}\sphinxparamcomma \sphinxparam{\DUrole{n}{a}}\sphinxparamcomma \sphinxparam{\DUrole{n}{b}}\sphinxparamcomma \sphinxparam{\DUrole{n}{c}}\sphinxparamcomma \sphinxparam{\DUrole{n}{d}}}{}
\pysigstopsignatures
\sphinxAtStartPar
The cubic function.

\sphinxAtStartPar
\(f(x) = ax^3 + bx^2 + cx + d\)
\begin{quote}\begin{description}
\sphinxlineitem{Parameters}\begin{itemize}
\item {} 
\sphinxAtStartPar
\sphinxstyleliteralstrong{\sphinxupquote{x}} (\sphinxstyleliteralemphasis{\sphinxupquote{float}}\sphinxstyleliteralemphasis{\sphinxupquote{ or }}\sphinxstyleliteralemphasis{\sphinxupquote{np.ndarray}}) \textendash{} function variable

\item {} 
\sphinxAtStartPar
\sphinxstyleliteralstrong{\sphinxupquote{a}} (\sphinxstyleliteralemphasis{\sphinxupquote{float}}) \textendash{} function constant

\item {} 
\sphinxAtStartPar
\sphinxstyleliteralstrong{\sphinxupquote{b}} (\sphinxstyleliteralemphasis{\sphinxupquote{float}}) \textendash{} function constant

\item {} 
\sphinxAtStartPar
\sphinxstyleliteralstrong{\sphinxupquote{c}} (\sphinxstyleliteralemphasis{\sphinxupquote{float}}) \textendash{} function constant

\item {} 
\sphinxAtStartPar
\sphinxstyleliteralstrong{\sphinxupquote{d}} (\sphinxstyleliteralemphasis{\sphinxupquote{float}}) \textendash{} function constant

\end{itemize}

\sphinxlineitem{Returns}
\sphinxAtStartPar
\begin{itemize}
\item {} 
\sphinxAtStartPar
\sphinxstylestrong{f} (\sphinxstyleemphasis{float or np.ndarray}) \textendash{} evaluated cubic function

\end{itemize}


\end{description}\end{quote}

\end{fulllineitems}

\index{dict\_from\_arrays() (in module skinoptics.utils)@\spxentry{dict\_from\_arrays()}\spxextra{in module skinoptics.utils}}

\begin{fulllineitems}
\phantomsection\label{\detokenize{01_utils:skinoptics.utils.dict_from_arrays}}
\pysigstartsignatures
\pysiglinewithargsret{\sphinxcode{\sphinxupquote{skinoptics.utils.}}\sphinxbfcode{\sphinxupquote{dict\_from\_arrays}}}{\sphinxparam{\DUrole{n}{array\_keys}}\sphinxparamcomma \sphinxparam{\DUrole{n}{array\_values}}}{}
\pysigstopsignatures
\sphinxAtStartPar
Construct a dictionary from two arrays.
\begin{quote}\begin{description}
\sphinxlineitem{Parameters}\begin{itemize}
\item {} 
\sphinxAtStartPar
\sphinxstyleliteralstrong{\sphinxupquote{array\_keys}} (\sphinxstyleliteralemphasis{\sphinxupquote{np.ndarray}}) \textendash{} array with the dictionary keys

\item {} 
\sphinxAtStartPar
\sphinxstyleliteralstrong{\sphinxupquote{array\_values}} (\sphinxstyleliteralemphasis{\sphinxupquote{np.ndarray}}) \textendash{} array with the dictionary values

\end{itemize}

\sphinxlineitem{Returns}
\sphinxAtStartPar
\begin{itemize}
\item {} 
\sphinxAtStartPar
\sphinxstylestrong{dict} (\sphinxstyleemphasis{dictionary}) \textendash{} dictionary constructed with the two arrays

\end{itemize}


\end{description}\end{quote}

\end{fulllineitems}

\index{exp\_decay() (in module skinoptics.utils)@\spxentry{exp\_decay()}\spxextra{in module skinoptics.utils}}

\begin{fulllineitems}
\phantomsection\label{\detokenize{01_utils:skinoptics.utils.exp_decay}}
\pysigstartsignatures
\pysiglinewithargsret{\sphinxcode{\sphinxupquote{skinoptics.utils.}}\sphinxbfcode{\sphinxupquote{exp\_decay}}}{\sphinxparam{\DUrole{n}{x}}\sphinxparamcomma \sphinxparam{\DUrole{n}{a}}\sphinxparamcomma \sphinxparam{\DUrole{n}{b}}\sphinxparamcomma \sphinxparam{\DUrole{n}{c}}}{}
\pysigstopsignatures
\sphinxAtStartPar
The exponential decay function.

\sphinxAtStartPar
\(f(x) = a \mbox{ exp}(-|b|x) + c\)
\begin{quote}\begin{description}
\sphinxlineitem{Parameters}\begin{itemize}
\item {} 
\sphinxAtStartPar
\sphinxstyleliteralstrong{\sphinxupquote{x}} (\sphinxstyleliteralemphasis{\sphinxupquote{float}}\sphinxstyleliteralemphasis{\sphinxupquote{ or }}\sphinxstyleliteralemphasis{\sphinxupquote{np.ndarray}}) \textendash{} function variable

\item {} 
\sphinxAtStartPar
\sphinxstyleliteralstrong{\sphinxupquote{a}} (\sphinxstyleliteralemphasis{\sphinxupquote{float}}) \textendash{} function constant

\item {} 
\sphinxAtStartPar
\sphinxstyleliteralstrong{\sphinxupquote{b}} (\sphinxstyleliteralemphasis{\sphinxupquote{float}}) \textendash{} function constant

\item {} 
\sphinxAtStartPar
\sphinxstyleliteralstrong{\sphinxupquote{c}} (\sphinxstyleliteralemphasis{\sphinxupquote{float}}) \textendash{} function constant

\end{itemize}

\sphinxlineitem{Returns}
\sphinxAtStartPar
\begin{itemize}
\item {} 
\sphinxAtStartPar
\sphinxstylestrong{f} (\sphinxstyleemphasis{float or np.ndarray}) \textendash{} evaluated exponential decay function

\end{itemize}


\end{description}\end{quote}

\end{fulllineitems}

\index{exp\_decay\_inc\_form() (in module skinoptics.utils)@\spxentry{exp\_decay\_inc\_form()}\spxextra{in module skinoptics.utils}}

\begin{fulllineitems}
\phantomsection\label{\detokenize{01_utils:skinoptics.utils.exp_decay_inc_form}}
\pysigstartsignatures
\pysiglinewithargsret{\sphinxcode{\sphinxupquote{skinoptics.utils.}}\sphinxbfcode{\sphinxupquote{exp\_decay\_inc\_form}}}{\sphinxparam{\DUrole{n}{x}}\sphinxparamcomma \sphinxparam{\DUrole{n}{a}}\sphinxparamcomma \sphinxparam{\DUrole{n}{b}}\sphinxparamcomma \sphinxparam{\DUrole{n}{c}}}{}
\pysigstopsignatures
\sphinxAtStartPar
The exponential decay (increasing form) function.

\sphinxAtStartPar
\(f(x) = a(1 - \mbox{exp}(-|b|x) + c\)
\begin{quote}\begin{description}
\sphinxlineitem{Parameters}\begin{itemize}
\item {} 
\sphinxAtStartPar
\sphinxstyleliteralstrong{\sphinxupquote{x}} (\sphinxstyleliteralemphasis{\sphinxupquote{float}}\sphinxstyleliteralemphasis{\sphinxupquote{ or }}\sphinxstyleliteralemphasis{\sphinxupquote{np.ndarray}}) \textendash{} function variable

\item {} 
\sphinxAtStartPar
\sphinxstyleliteralstrong{\sphinxupquote{a}} (\sphinxstyleliteralemphasis{\sphinxupquote{float}}) \textendash{} function constant

\item {} 
\sphinxAtStartPar
\sphinxstyleliteralstrong{\sphinxupquote{b}} (\sphinxstyleliteralemphasis{\sphinxupquote{float}}) \textendash{} function constant

\item {} 
\sphinxAtStartPar
\sphinxstyleliteralstrong{\sphinxupquote{c}} (\sphinxstyleliteralemphasis{\sphinxupquote{float}}) \textendash{} function constant

\end{itemize}

\sphinxlineitem{Returns}
\sphinxAtStartPar
\begin{itemize}
\item {} 
\sphinxAtStartPar
\sphinxstylestrong{f} (\sphinxstyleemphasis{float or np.ndarray}) \textendash{} evaluated exponential decay (increasing form) function

\end{itemize}


\end{description}\end{quote}

\end{fulllineitems}

\index{gaussian() (in module skinoptics.utils)@\spxentry{gaussian()}\spxextra{in module skinoptics.utils}}

\begin{fulllineitems}
\phantomsection\label{\detokenize{01_utils:skinoptics.utils.gaussian}}
\pysigstartsignatures
\pysiglinewithargsret{\sphinxcode{\sphinxupquote{skinoptics.utils.}}\sphinxbfcode{\sphinxupquote{gaussian}}}{\sphinxparam{\DUrole{n}{x}}\sphinxparamcomma \sphinxparam{\DUrole{n}{a}}\sphinxparamcomma \sphinxparam{\DUrole{n}{b}}\sphinxparamcomma \sphinxparam{\DUrole{n}{c}}}{}
\pysigstopsignatures
\sphinxAtStartPar
The Gaussian function.

\sphinxAtStartPar
\(f(x) = a \mbox{ exp}\left[-\frac{1}{2}\frac{(x - b)^2}{c^2}\right]\)
\begin{quote}\begin{description}
\sphinxlineitem{Parameters}\begin{itemize}
\item {} 
\sphinxAtStartPar
\sphinxstyleliteralstrong{\sphinxupquote{x}} (\sphinxstyleliteralemphasis{\sphinxupquote{float}}\sphinxstyleliteralemphasis{\sphinxupquote{ or }}\sphinxstyleliteralemphasis{\sphinxupquote{np.ndarray}}) \textendash{} function variable

\item {} 
\sphinxAtStartPar
\sphinxstyleliteralstrong{\sphinxupquote{a}} (\sphinxstyleliteralemphasis{\sphinxupquote{float}}) \textendash{} function constant

\item {} 
\sphinxAtStartPar
\sphinxstyleliteralstrong{\sphinxupquote{b}} (\sphinxstyleliteralemphasis{\sphinxupquote{float}}) \textendash{} function constant

\item {} 
\sphinxAtStartPar
\sphinxstyleliteralstrong{\sphinxupquote{c}} (\sphinxstyleliteralemphasis{\sphinxupquote{float}}) \textendash{} function constant

\end{itemize}

\sphinxlineitem{Returns}
\sphinxAtStartPar
\begin{itemize}
\item {} 
\sphinxAtStartPar
\sphinxstylestrong{f} (\sphinxstyleemphasis{float or np.ndarray}) \textendash{} evaluated Gaussian function

\end{itemize}


\end{description}\end{quote}

\end{fulllineitems}

\index{heaviside() (in module skinoptics.utils)@\spxentry{heaviside()}\spxextra{in module skinoptics.utils}}

\begin{fulllineitems}
\phantomsection\label{\detokenize{01_utils:skinoptics.utils.heaviside}}
\pysigstartsignatures
\pysiglinewithargsret{\sphinxcode{\sphinxupquote{skinoptics.utils.}}\sphinxbfcode{\sphinxupquote{heaviside}}}{\sphinxparam{\DUrole{n}{x}}}{}
\pysigstopsignatures
\sphinxAtStartPar
The Heaviside step function.

\sphinxAtStartPar
\(H(x) =  
\left \{ \begin{matrix}
0, & \mbox{if } g < 0 \\
\frac{1}{2}, & \mbox{if } g = 0 \\
1, & \mbox{if } g > 0 \\
\end{matrix} \right.\)
\begin{quote}\begin{description}
\sphinxlineitem{Parameters}
\sphinxAtStartPar
\sphinxstyleliteralstrong{\sphinxupquote{x}} (\sphinxstyleliteralemphasis{\sphinxupquote{float}}\sphinxstyleliteralemphasis{\sphinxupquote{ or }}\sphinxstyleliteralemphasis{\sphinxupquote{np.ndarray}}) \textendash{} function variable

\sphinxlineitem{Returns}
\sphinxAtStartPar
\begin{itemize}
\item {} 
\sphinxAtStartPar
\sphinxstylestrong{H} (\sphinxstyleemphasis{float or np.ndarray}) \textendash{} evaluated Heaviside step function

\end{itemize}


\end{description}\end{quote}

\end{fulllineitems}

\index{linear() (in module skinoptics.utils)@\spxentry{linear()}\spxextra{in module skinoptics.utils}}

\begin{fulllineitems}
\phantomsection\label{\detokenize{01_utils:skinoptics.utils.linear}}
\pysigstartsignatures
\pysiglinewithargsret{\sphinxcode{\sphinxupquote{skinoptics.utils.}}\sphinxbfcode{\sphinxupquote{linear}}}{\sphinxparam{\DUrole{n}{x}}\sphinxparamcomma \sphinxparam{\DUrole{n}{a}}\sphinxparamcomma \sphinxparam{\DUrole{n}{b}}}{}
\pysigstopsignatures
\sphinxAtStartPar
The linear function.

\sphinxAtStartPar
\(f(x) = ax + b\)
\begin{quote}\begin{description}
\sphinxlineitem{Parameters}\begin{itemize}
\item {} 
\sphinxAtStartPar
\sphinxstyleliteralstrong{\sphinxupquote{x}} (\sphinxstyleliteralemphasis{\sphinxupquote{float}}\sphinxstyleliteralemphasis{\sphinxupquote{ or }}\sphinxstyleliteralemphasis{\sphinxupquote{np.ndarray}}) \textendash{} function variable

\item {} 
\sphinxAtStartPar
\sphinxstyleliteralstrong{\sphinxupquote{a}} (\sphinxstyleliteralemphasis{\sphinxupquote{float}}) \textendash{} slope

\item {} 
\sphinxAtStartPar
\sphinxstyleliteralstrong{\sphinxupquote{b}} (\sphinxstyleliteralemphasis{\sphinxupquote{float}}) \textendash{} y\sphinxhyphen{}intercept

\end{itemize}

\sphinxlineitem{Returns}
\sphinxAtStartPar
\begin{itemize}
\item {} 
\sphinxAtStartPar
\sphinxstylestrong{f} (\sphinxstyleemphasis{float or np.ndarray}) \textendash{} evaluated linear function

\end{itemize}


\end{description}\end{quote}

\end{fulllineitems}

\index{mod\_gaussian\_Wyman() (in module skinoptics.utils)@\spxentry{mod\_gaussian\_Wyman()}\spxextra{in module skinoptics.utils}}

\begin{fulllineitems}
\phantomsection\label{\detokenize{01_utils:skinoptics.utils.mod_gaussian_Wyman}}
\pysigstartsignatures
\pysiglinewithargsret{\sphinxcode{\sphinxupquote{skinoptics.utils.}}\sphinxbfcode{\sphinxupquote{mod\_gaussian\_Wyman}}}{\sphinxparam{\DUrole{n}{x}}\sphinxparamcomma \sphinxparam{\DUrole{n}{a}}\sphinxparamcomma \sphinxparam{\DUrole{n}{b}}\sphinxparamcomma \sphinxparam{\DUrole{n}{c}}\sphinxparamcomma \sphinxparam{\DUrole{n}{d}}}{}
\pysigstopsignatures
\begin{DUlineblock}{0em}
\item[] The modified Gaussian function needed to calculate some analytical functions 
\item[] from Wyman, Sloan \& Shirley 2013 {[}WSS13{]} (see skinoptics.colors.cmfs).
\end{DUlineblock}

\sphinxAtStartPar
\(f(x) = a \mbox{ exp}\left\{-d \left[\mbox{ ln}\left(\frac{x - b}{c}\right)\right]^2\right\}\)
\begin{quote}\begin{description}
\sphinxlineitem{Parameters}\begin{itemize}
\item {} 
\sphinxAtStartPar
\sphinxstyleliteralstrong{\sphinxupquote{x}} (\sphinxstyleliteralemphasis{\sphinxupquote{float}}\sphinxstyleliteralemphasis{\sphinxupquote{ or }}\sphinxstyleliteralemphasis{\sphinxupquote{np.ndarray}}) \textendash{} function variable

\item {} 
\sphinxAtStartPar
\sphinxstyleliteralstrong{\sphinxupquote{a}} (\sphinxstyleliteralemphasis{\sphinxupquote{float}}) \textendash{} function constant

\item {} 
\sphinxAtStartPar
\sphinxstyleliteralstrong{\sphinxupquote{b}} (\sphinxstyleliteralemphasis{\sphinxupquote{float}}) \textendash{} function constant

\item {} 
\sphinxAtStartPar
\sphinxstyleliteralstrong{\sphinxupquote{c}} (\sphinxstyleliteralemphasis{\sphinxupquote{float}}) \textendash{} function constant

\item {} 
\sphinxAtStartPar
\sphinxstyleliteralstrong{\sphinxupquote{d}} (\sphinxstyleliteralemphasis{\sphinxupquote{float}}) \textendash{} function constant

\end{itemize}

\sphinxlineitem{Returns}
\sphinxAtStartPar
\begin{itemize}
\item {} 
\sphinxAtStartPar
\sphinxstylestrong{f} (\sphinxstyleemphasis{float or np.ndarray}) \textendash{} evaluated function

\end{itemize}


\end{description}\end{quote}

\end{fulllineitems}

\index{natural\_log() (in module skinoptics.utils)@\spxentry{natural\_log()}\spxextra{in module skinoptics.utils}}

\begin{fulllineitems}
\phantomsection\label{\detokenize{01_utils:skinoptics.utils.natural_log}}
\pysigstartsignatures
\pysiglinewithargsret{\sphinxcode{\sphinxupquote{skinoptics.utils.}}\sphinxbfcode{\sphinxupquote{natural\_log}}}{\sphinxparam{\DUrole{n}{x}}\sphinxparamcomma \sphinxparam{\DUrole{n}{a}}\sphinxparamcomma \sphinxparam{\DUrole{n}{b}}\sphinxparamcomma \sphinxparam{\DUrole{n}{c}}}{}
\pysigstopsignatures
\sphinxAtStartPar
The natural logarithm function.

\sphinxAtStartPar
\(f(x) = a \mbox{ ln}(x + b) + c\)
\begin{quote}\begin{description}
\sphinxlineitem{Parameters}\begin{itemize}
\item {} 
\sphinxAtStartPar
\sphinxstyleliteralstrong{\sphinxupquote{x}} (\sphinxstyleliteralemphasis{\sphinxupquote{float}}\sphinxstyleliteralemphasis{\sphinxupquote{ or }}\sphinxstyleliteralemphasis{\sphinxupquote{np.ndarray}}) \textendash{} function variable

\item {} 
\sphinxAtStartPar
\sphinxstyleliteralstrong{\sphinxupquote{a}} (\sphinxstyleliteralemphasis{\sphinxupquote{float}}) \textendash{} function constant

\item {} 
\sphinxAtStartPar
\sphinxstyleliteralstrong{\sphinxupquote{b}} (\sphinxstyleliteralemphasis{\sphinxupquote{float}}) \textendash{} function constant

\item {} 
\sphinxAtStartPar
\sphinxstyleliteralstrong{\sphinxupquote{c}} (\sphinxstyleliteralemphasis{\sphinxupquote{float}}) \textendash{} function constant

\end{itemize}

\sphinxlineitem{Returns}
\sphinxAtStartPar
\begin{itemize}
\item {} 
\sphinxAtStartPar
\sphinxstylestrong{f} (\sphinxstyleemphasis{float or np.ndarray}) \textendash{} evaluated natural logarithm function

\end{itemize}


\end{description}\end{quote}

\end{fulllineitems}

\index{piecewise\_gaussian\_Wyman() (in module skinoptics.utils)@\spxentry{piecewise\_gaussian\_Wyman()}\spxextra{in module skinoptics.utils}}

\begin{fulllineitems}
\phantomsection\label{\detokenize{01_utils:skinoptics.utils.piecewise_gaussian_Wyman}}
\pysigstartsignatures
\pysiglinewithargsret{\sphinxcode{\sphinxupquote{skinoptics.utils.}}\sphinxbfcode{\sphinxupquote{piecewise\_gaussian\_Wyman}}}{\sphinxparam{\DUrole{n}{x}}\sphinxparamcomma \sphinxparam{\DUrole{n}{a}}\sphinxparamcomma \sphinxparam{\DUrole{n}{b}}\sphinxparamcomma \sphinxparam{\DUrole{n}{c}}\sphinxparamcomma \sphinxparam{\DUrole{n}{d}}}{}
\pysigstopsignatures
\begin{DUlineblock}{0em}
\item[] The piecewise Gaussian function needed to calculate some analytical functions 
\item[] from Wyman, Sloan \& Shirley 2013 {[}WSS13{]} (see skinoptics.colors.cmfs).
\end{DUlineblock}

\sphinxAtStartPar
\(f(x) = a \mbox{ exp}\left\{-\frac{1}{2}[(x - b) S(x - b, c, d)]^2\right\}\)
\begin{quote}\begin{description}
\sphinxlineitem{Parameters}\begin{itemize}
\item {} 
\sphinxAtStartPar
\sphinxstyleliteralstrong{\sphinxupquote{x}} (\sphinxstyleliteralemphasis{\sphinxupquote{float}}\sphinxstyleliteralemphasis{\sphinxupquote{ or }}\sphinxstyleliteralemphasis{\sphinxupquote{np.ndarray}}) \textendash{} function variable

\item {} 
\sphinxAtStartPar
\sphinxstyleliteralstrong{\sphinxupquote{a}} (\sphinxstyleliteralemphasis{\sphinxupquote{float}}) \textendash{} function constant

\item {} 
\sphinxAtStartPar
\sphinxstyleliteralstrong{\sphinxupquote{b}} (\sphinxstyleliteralemphasis{\sphinxupquote{float}}) \textendash{} function constant

\item {} 
\sphinxAtStartPar
\sphinxstyleliteralstrong{\sphinxupquote{c}} (\sphinxstyleliteralemphasis{\sphinxupquote{float}}) \textendash{} function constant

\item {} 
\sphinxAtStartPar
\sphinxstyleliteralstrong{\sphinxupquote{d}} (\sphinxstyleliteralemphasis{\sphinxupquote{float}}) \textendash{} function constant

\end{itemize}

\sphinxlineitem{Returns}
\sphinxAtStartPar
\begin{itemize}
\item {} 
\sphinxAtStartPar
\sphinxstylestrong{f} (\sphinxstyleemphasis{float or np.ndarray}) \textendash{} evaluated function

\end{itemize}


\end{description}\end{quote}

\end{fulllineitems}

\index{quadratic() (in module skinoptics.utils)@\spxentry{quadratic()}\spxextra{in module skinoptics.utils}}

\begin{fulllineitems}
\phantomsection\label{\detokenize{01_utils:skinoptics.utils.quadratic}}
\pysigstartsignatures
\pysiglinewithargsret{\sphinxcode{\sphinxupquote{skinoptics.utils.}}\sphinxbfcode{\sphinxupquote{quadratic}}}{\sphinxparam{\DUrole{n}{x}}\sphinxparamcomma \sphinxparam{\DUrole{n}{a}}\sphinxparamcomma \sphinxparam{\DUrole{n}{b}}\sphinxparamcomma \sphinxparam{\DUrole{n}{c}}}{}
\pysigstopsignatures
\sphinxAtStartPar
The quadratic function.

\sphinxAtStartPar
\(f(x) = ax^2 + bx + c\)
\begin{quote}\begin{description}
\sphinxlineitem{Parameters}\begin{itemize}
\item {} 
\sphinxAtStartPar
\sphinxstyleliteralstrong{\sphinxupquote{x}} (\sphinxstyleliteralemphasis{\sphinxupquote{float}}\sphinxstyleliteralemphasis{\sphinxupquote{ or }}\sphinxstyleliteralemphasis{\sphinxupquote{np.ndarray}}) \textendash{} function variable

\item {} 
\sphinxAtStartPar
\sphinxstyleliteralstrong{\sphinxupquote{a}} (\sphinxstyleliteralemphasis{\sphinxupquote{float}}) \textendash{} function constant

\item {} 
\sphinxAtStartPar
\sphinxstyleliteralstrong{\sphinxupquote{b}} (\sphinxstyleliteralemphasis{\sphinxupquote{float}}) \textendash{} function constant

\item {} 
\sphinxAtStartPar
\sphinxstyleliteralstrong{\sphinxupquote{c}} (\sphinxstyleliteralemphasis{\sphinxupquote{float}}) \textendash{} function constant

\end{itemize}

\sphinxlineitem{Returns}
\sphinxAtStartPar
\begin{itemize}
\item {} 
\sphinxAtStartPar
\sphinxstylestrong{f} (\sphinxstyleemphasis{float or np.ndarray}) \textendash{} evaluated quadratic function

\end{itemize}


\end{description}\end{quote}

\end{fulllineitems}

\index{selector\_function\_Wyman() (in module skinoptics.utils)@\spxentry{selector\_function\_Wyman()}\spxextra{in module skinoptics.utils}}

\begin{fulllineitems}
\phantomsection\label{\detokenize{01_utils:skinoptics.utils.selector_function_Wyman}}
\pysigstartsignatures
\pysiglinewithargsret{\sphinxcode{\sphinxupquote{skinoptics.utils.}}\sphinxbfcode{\sphinxupquote{selector\_function\_Wyman}}}{\sphinxparam{\DUrole{n}{x}}\sphinxparamcomma \sphinxparam{\DUrole{n}{y}}\sphinxparamcomma \sphinxparam{\DUrole{n}{z}}}{}
\pysigstopsignatures
\begin{DUlineblock}{0em}
\item[] The selector function needed to calculate some analytical functions 
\item[] from Wyman, Sloan \& Shirley 2013 {[}WSS13{]} (see skinoptics.colors.cmfs).
\end{DUlineblock}

\sphinxAtStartPar
\(S(x,y,z) = y(1 - H(x)) + z H(x)\)

\sphinxAtStartPar
in which \(H(x)\) is the Heaviside step function (see function heaviside).
\begin{quote}\begin{description}
\sphinxlineitem{Parameters}\begin{itemize}
\item {} 
\sphinxAtStartPar
\sphinxstyleliteralstrong{\sphinxupquote{x}} (\sphinxstyleliteralemphasis{\sphinxupquote{float}}\sphinxstyleliteralemphasis{\sphinxupquote{ or }}\sphinxstyleliteralemphasis{\sphinxupquote{np.ndarray}}) \textendash{} function variable

\item {} 
\sphinxAtStartPar
\sphinxstyleliteralstrong{\sphinxupquote{y}} (\sphinxstyleliteralemphasis{\sphinxupquote{float}}\sphinxstyleliteralemphasis{\sphinxupquote{ or }}\sphinxstyleliteralemphasis{\sphinxupquote{np.ndarray}}) \textendash{} function variable

\item {} 
\sphinxAtStartPar
\sphinxstyleliteralstrong{\sphinxupquote{z}} (\sphinxstyleliteralemphasis{\sphinxupquote{float}}\sphinxstyleliteralemphasis{\sphinxupquote{ or }}\sphinxstyleliteralemphasis{\sphinxupquote{np.ndarray}}) \textendash{} function variable

\end{itemize}

\sphinxlineitem{Returns}
\sphinxAtStartPar
\begin{itemize}
\item {} 
\sphinxAtStartPar
\sphinxstylestrong{S} (\sphinxstyleemphasis{float or np.ndarray}) \textendash{} evaluated selector function

\end{itemize}


\end{description}\end{quote}

\end{fulllineitems}


\sphinxstepscope


\subsection{skinoptics.dataframes module}
\label{\detokenize{02_dataframes:module-skinoptics.dataframes}}\label{\detokenize{02_dataframes:skinoptics-dataframes-module}}\label{\detokenize{02_dataframes::doc}}\index{module@\spxentry{module}!skinoptics.dataframes@\spxentry{skinoptics.dataframes}}\index{skinoptics.dataframes@\spxentry{skinoptics.dataframes}!module@\spxentry{module}}
\sphinxAtStartPar
Copyright (C) 2024 Victor Lima
\begin{quote}

\begin{DUlineblock}{0em}
\item[] This program is free software: you can redistribute it and/or modify
\item[] it under the terms of the GNU General Public License as published by
\item[] the Free Software Foundation, either version 3 of the License, or
\item[] (at your option) any later version.
\end{DUlineblock}

\begin{DUlineblock}{0em}
\item[] This program is distributed in the hope that it will be useful,
\item[] but WITHOUT ANY WARRANTY; without even the implied warranty of
\item[] MERCHANTABILITY or FITNESS FOR A PARTICULAR PURPOSE.  See the
\item[] GNU General Public License for more details.
\end{DUlineblock}

\begin{DUlineblock}{0em}
\item[] You should have received a copy of the GNU General Public License
\item[] along with this program.  If not, see \textless{}\sphinxurl{https://www.gnu.org/licenses/}\textgreater{}.
\end{DUlineblock}
\end{quote}

\begin{DUlineblock}{0em}
\item[] Victor Lima
\item[] victorporto@ifsc.usp.br
\item[] victor.lima@ufscar.br
\end{DUlineblock}

\begin{DUlineblock}{0em}
\item[] Release Date:
\item[] August 2024
\item[] Last Modification:
\item[] August 2024
\end{DUlineblock}

\begin{DUlineblock}{0em}
\item[] Example:
\item[] wps\_dataframe (respective to datasetscolorswps.txt)
\end{DUlineblock}


\begin{savenotes}\sphinxattablestart
\sphinxthistablewithglobalstyle
\centering
\begin{tabulary}{\linewidth}[t]{TTTTT}
\sphinxtoprule
\sphinxstyletheadfamily 
\sphinxAtStartPar
illuminant
&\sphinxstyletheadfamily 
\sphinxAtStartPar
observer
&\sphinxstyletheadfamily 
\sphinxAtStartPar
Xn{[}\sphinxhyphen{}{]}
&\sphinxstyletheadfamily 
\sphinxAtStartPar
Yn{[}\sphinxhyphen{}{]}
&\sphinxstyletheadfamily 
\sphinxAtStartPar
Zn{[}\sphinxhyphen{}{]}
\\
\sphinxmidrule
\sphinxtableatstartofbodyhook
\sphinxAtStartPar
A
&
\sphinxAtStartPar
2o
&
\sphinxAtStartPar
1.0985
&
\sphinxAtStartPar
1
&
\sphinxAtStartPar
0.3558
\\
\sphinxhline
\sphinxAtStartPar
D50
&
\sphinxAtStartPar
2o
&
\sphinxAtStartPar
0.9641
&
\sphinxAtStartPar
1
&
\sphinxAtStartPar
0.8250
\\
\sphinxhline
\sphinxAtStartPar
D55
&
\sphinxAtStartPar
2o
&
\sphinxAtStartPar
0.9568
&
\sphinxAtStartPar
1
&
\sphinxAtStartPar
0.9214
\\
\sphinxhline
\sphinxAtStartPar
D65
&
\sphinxAtStartPar
2o
&
\sphinxAtStartPar
0.9504
&
\sphinxAtStartPar
1
&
\sphinxAtStartPar
1.0888
\\
\sphinxhline
\sphinxAtStartPar
D75
&
\sphinxAtStartPar
2o
&
\sphinxAtStartPar
0.9497
&
\sphinxAtStartPar
1
&
\sphinxAtStartPar
1.2257
\\
\sphinxhline
\sphinxAtStartPar
A
&
\sphinxAtStartPar
10o
&
\sphinxAtStartPar
1.1114
&
\sphinxAtStartPar
1
&
\sphinxAtStartPar
0.3520
\\
\sphinxhline
\sphinxAtStartPar
D50
&
\sphinxAtStartPar
10o
&
\sphinxAtStartPar
0.9671
&
\sphinxAtStartPar
1
&
\sphinxAtStartPar
0.8141
\\
\sphinxhline
\sphinxAtStartPar
D55
&
\sphinxAtStartPar
10o
&
\sphinxAtStartPar
0.9580
&
\sphinxAtStartPar
1
&
\sphinxAtStartPar
0.9093
\\
\sphinxhline
\sphinxAtStartPar
D65
&
\sphinxAtStartPar
10o
&
\sphinxAtStartPar
0.9481
&
\sphinxAtStartPar
1
&
\sphinxAtStartPar
1.0733
\\
\sphinxhline
\sphinxAtStartPar
D75
&
\sphinxAtStartPar
10o
&
\sphinxAtStartPar
0.9442
&
\sphinxAtStartPar
1
&
\sphinxAtStartPar
1.2060
\\
\sphinxbottomrule
\end{tabulary}
\sphinxtableafterendhook\par
\sphinxattableend\end{savenotes}

\sphinxstepscope


\subsection{skinoptics.absorption\_coefficient module}
\label{\detokenize{03_absorption_coefficient:module-skinoptics.absorption_coefficient}}\label{\detokenize{03_absorption_coefficient:skinoptics-absorption-coefficient-module}}\label{\detokenize{03_absorption_coefficient::doc}}\index{module@\spxentry{module}!skinoptics.absorption\_coefficient@\spxentry{skinoptics.absorption\_coefficient}}\index{skinoptics.absorption\_coefficient@\spxentry{skinoptics.absorption\_coefficient}!module@\spxentry{module}}
\sphinxAtStartPar
Copyright (C) 2024 Victor Lima
\begin{quote}

\begin{DUlineblock}{0em}
\item[] This program is free software: you can redistribute it and/or modify
\item[] it under the terms of the GNU General Public License as published by
\item[] the Free Software Foundation, either version 3 of the License, or
\item[] (at your option) any later version.
\end{DUlineblock}

\begin{DUlineblock}{0em}
\item[] This program is distributed in the hope that it will be useful,
\item[] but WITHOUT ANY WARRANTY; without even the implied warranty of
\item[] MERCHANTABILITY or FITNESS FOR A PARTICULAR PURPOSE.  See the
\item[] GNU General Public License for more details.
\end{DUlineblock}

\begin{DUlineblock}{0em}
\item[] You should have received a copy of the GNU General Public License
\item[] along with this program.  If not, see \textless{}\sphinxurl{https://www.gnu.org/licenses/}\textgreater{}.
\end{DUlineblock}
\end{quote}

\begin{DUlineblock}{0em}
\item[] Victor Lima
\item[] victorporto@ifsc.usp.br
\item[] victor.lima@ufscar.br
\end{DUlineblock}

\begin{DUlineblock}{0em}
\item[] Release Date:
\item[] August 2024
\item[] Last Modification:
\item[] August 2024
\end{DUlineblock}

\begin{DUlineblock}{0em}
\item[] References:
\end{DUlineblock}

\begin{DUlineblock}{0em}
\item[] {[}HQ73{]} Hale \& Querry 1973.
\item[] Optical Constants of Water in the 200\sphinxhyphen{}nm to 200\sphinxhyphen{}μm Wavelength Region.
\item[] \sphinxurl{https://doi.org/10.1364/AO.12.000555}
\end{DUlineblock}

\begin{DUlineblock}{0em}
\item[] {[}S81{]} Segelstein 1981.
\item[] The complex refractive index of water.
\item[] \sphinxurl{https://mospace.umsystem.edu/xmlui/handle/10355/11599}
\end{DUlineblock}

\begin{DUlineblock}{0em}
\item[] {[}AF90{]} Agati \& Fusi 1990.
\item[] New trends in photobiology recent advances in bilirubin photophysics.
\item[] \sphinxurl{https://doi.org/10.1016/1011-1344(90)85138-M}
\end{DUlineblock}

\begin{DUlineblock}{0em}
\item[] {[}B90{]} Billett 1990.
\item[] Hemoglobin and Hematocrit.
\item[] \sphinxurl{https://www.ncbi.nlm.nih.gov/books/NBK259/}
\end{DUlineblock}

\begin{DUlineblock}{0em}
\item[] {[}H02{]} Hecht 2002.
\item[] Optics. (4th Edition)
\end{DUlineblock}

\begin{DUlineblock}{0em}
\item[] {[}DJ06{]} Donner \& Jensen 2006.
\item[] A Spectral BSSRDF for Shading Human Skin.
\item[] \sphinxurl{http://graphics.ucsd.edu/~henrik/papers/skin\_bssrdf/}
\end{DUlineblock}

\begin{DUlineblock}{0em}
\item[] {[}S*06{]} Salomatina, Jiang, Novak \& Yaroslavsky 2006.
\item[] Optical properties of normal and cancerous human skin in the visible and near\sphinxhyphen{}infrared spectral range.
\item[] \sphinxurl{https://doi.org/10.1117/1.2398928}
\end{DUlineblock}

\begin{DUlineblock}{0em}
\item[] {[}SS06{]} Sarna \& Swartz 2006.
\item[] The Physical Properties of Melanins.
\item[] \sphinxurl{https://doi.org/10.1002/9780470987100.ch16}
\end{DUlineblock}

\begin{DUlineblock}{0em}
\item[] {[}DJV11{]} Delgado Atencio, Jacques \& Vázquez y Montiel 2011.
\item[] Monte Carlo Modeling of Light Propagation in Neonatal Skin.
\item[] \sphinxurl{https://doi.org/10.5772/15853}
\end{DUlineblock}

\begin{DUlineblock}{0em}
\item[] {[}J13{]} Jacques 2013.
\item[] Optical properties of biological tissues: a review.
\item[] \sphinxurl{https://doi.org/10.1088/0031-9155/58/14/5007}
\end{DUlineblock}

\begin{DUlineblock}{0em}
\item[] {[}B*14{]} Bosschaart, Edelman, Aalders, van Leeuwen \& Faber 2014.
\item[] A literature review and novel theoretical approach on the optical properties of whole blood.
\item[] \sphinxurl{https://doi.org/10.1007/s10103-013-1446-7}
\end{DUlineblock}

\begin{DUlineblock}{0em}
\item[] {[}G17{]} Griffiths 2017.
\item[] Introduction to Electrodynamics. (4th Edition)
\item[] \sphinxurl{https://doi.org/10.1017/9781108333511}
\end{DUlineblock}

\begin{DUlineblock}{0em}
\item[] {[}TL23{]} Taniguchi \& Lindsey 2023.
\item[] Absorption and fluorescence spectra of open\sphinxhyphen{}chain tetrapyrrole pigments \textendash{} bilirubins, biliverdins,
\item[] phycobilins, and synthetic analogues.
\item[] \sphinxurl{https://doi.org/10.1016/j.jphotochemrev.2023.100585}
\end{DUlineblock}

\begin{DUlineblock}{0em}
\item[] {[}S*23{]} Sá, Bacal, Gomes, Silva, Gonçalves \& Malta 2023.
\item[] Blood count reference intervals for the Brazilian adult population: National Health Survey.
\item[] \sphinxurl{https://doi.org/10.1590/1980-549720230004.supl.1}
\end{DUlineblock}
\index{Cmass\_from\_Cmolar() (in module skinoptics.absorption\_coefficient)@\spxentry{Cmass\_from\_Cmolar()}\spxextra{in module skinoptics.absorption\_coefficient}}

\begin{fulllineitems}
\phantomsection\label{\detokenize{03_absorption_coefficient:skinoptics.absorption_coefficient.Cmass_from_Cmolar}}
\pysigstartsignatures
\pysiglinewithargsret{\sphinxcode{\sphinxupquote{skinoptics.absorption\_coefficient.}}\sphinxbfcode{\sphinxupquote{Cmass\_from\_Cmolar}}}{\sphinxparam{\DUrole{n}{Cmolar}}\sphinxparamcomma \sphinxparam{\DUrole{n}{molar\_mass}}}{}
\pysigstopsignatures
\sphinxAtStartPar
Calculate the mass concentration from the molar concentration and the molar mass.

\sphinxAtStartPar
\(C_{mass} = M C_{molar}\)
\begin{quote}\begin{description}
\sphinxlineitem{Parameters}\begin{itemize}
\item {} 
\sphinxAtStartPar
\sphinxstyleliteralstrong{\sphinxupquote{Cmolar}} (\sphinxstyleliteralemphasis{\sphinxupquote{float}}\sphinxstyleliteralemphasis{\sphinxupquote{ or }}\sphinxstyleliteralemphasis{\sphinxupquote{np.ndarray}}) \textendash{} molar concentration {[}M{]}

\item {} 
\sphinxAtStartPar
\sphinxstyleliteralstrong{\sphinxupquote{molar\_mass}} (\sphinxstyleliteralemphasis{\sphinxupquote{float}}) \textendash{} molar mass {[}g mol\sphinxhyphen{}1{]}

\end{itemize}

\sphinxlineitem{Returns}
\sphinxAtStartPar
\begin{itemize}
\item {} 
\sphinxAtStartPar
\sphinxstylestrong{Cmass} (\sphinxstyleemphasis{float or np.ndarray}) \textendash{} mass concentration {[}g L\sphinxhyphen{}1{]}

\end{itemize}


\end{description}\end{quote}

\end{fulllineitems}

\index{Cmolar\_from\_Cmass() (in module skinoptics.absorption\_coefficient)@\spxentry{Cmolar\_from\_Cmass()}\spxextra{in module skinoptics.absorption\_coefficient}}

\begin{fulllineitems}
\phantomsection\label{\detokenize{03_absorption_coefficient:skinoptics.absorption_coefficient.Cmolar_from_Cmass}}
\pysigstartsignatures
\pysiglinewithargsret{\sphinxcode{\sphinxupquote{skinoptics.absorption\_coefficient.}}\sphinxbfcode{\sphinxupquote{Cmolar\_from\_Cmass}}}{\sphinxparam{\DUrole{n}{Cmass}}\sphinxparamcomma \sphinxparam{\DUrole{n}{molar\_mass}}}{}
\pysigstopsignatures
\sphinxAtStartPar
Calculate the molar concentration from the mass concentration and the molar mass.

\sphinxAtStartPar
\(C_{molar} = \frac{C_{mass}}{M}\)
\begin{quote}\begin{description}
\sphinxlineitem{Parameters}\begin{itemize}
\item {} 
\sphinxAtStartPar
\sphinxstyleliteralstrong{\sphinxupquote{Cmass}} (\sphinxstyleliteralemphasis{\sphinxupquote{float}}\sphinxstyleliteralemphasis{\sphinxupquote{ or }}\sphinxstyleliteralemphasis{\sphinxupquote{np.ndarray}}) \textendash{} mass concentration {[}g L\sphinxhyphen{}1{]}

\item {} 
\sphinxAtStartPar
\sphinxstyleliteralstrong{\sphinxupquote{molar\_mass}} (\sphinxstyleliteralemphasis{\sphinxupquote{float}}) \textendash{} molar mass {[}g mol\sphinxhyphen{}1{]}

\end{itemize}

\sphinxlineitem{Returns}
\sphinxAtStartPar
\begin{itemize}
\item {} 
\sphinxAtStartPar
\sphinxstylestrong{Cmolar} (\sphinxstyleemphasis{float or np.ndarray}) \textendash{} molar concentration {[}M{]}

\end{itemize}


\end{description}\end{quote}

\end{fulllineitems}

\index{ext\_eum\_Sarna() (in module skinoptics.absorption\_coefficient)@\spxentry{ext\_eum\_Sarna()}\spxextra{in module skinoptics.absorption\_coefficient}}

\begin{fulllineitems}
\phantomsection\label{\detokenize{03_absorption_coefficient:skinoptics.absorption_coefficient.ext_eum_Sarna}}
\pysigstartsignatures
\pysiglinewithargsret{\sphinxcode{\sphinxupquote{skinoptics.absorption\_coefficient.}}\sphinxbfcode{\sphinxupquote{ext\_eum\_Sarna}}}{\sphinxparam{\DUrole{n}{lambda0}}}{}
\pysigstopsignatures
\begin{DUlineblock}{0em}
\item[] The extinction coefficient of EUMELANIN in phosphate buffer as a function of wavelength.
\item[] Linear interpolation of experimental data from Sarna \& Swartz 2006 {[}SS06{]}
\item[] (see their Fig. 16.3\sphinxhyphen{}a) graphically deduced by Jacques and publicly available at
\item[] \textless{}\sphinxurl{https://omlc.org/spectra/melanin/extcoeff.html}\textgreater{}.
\end{DUlineblock}

\begin{DUlineblock}{0em}
\item[] wavelength range: {[}210 nm, 820 nm{]}
\end{DUlineblock}
\begin{quote}\begin{description}
\sphinxlineitem{Parameters}
\sphinxAtStartPar
\sphinxstyleliteralstrong{\sphinxupquote{lambda0}} (\sphinxstyleliteralemphasis{\sphinxupquote{float}}\sphinxstyleliteralemphasis{\sphinxupquote{ or }}\sphinxstyleliteralemphasis{\sphinxupquote{np.ndarray}}) \textendash{} wavelength {[}nm{]}

\sphinxlineitem{Returns}
\sphinxAtStartPar
\begin{itemize}
\item {} 
\sphinxAtStartPar
\sphinxstylestrong{ext} (\sphinxstyleemphasis{float or np.ndarray}) \textendash{} extinction coefficient {[}cm\sphinxhyphen{}1 mL mg\sphinxhyphen{}1{]}

\end{itemize}


\end{description}\end{quote}

\end{fulllineitems}

\index{ext\_from\_Abs\_and\_Cmass() (in module skinoptics.absorption\_coefficient)@\spxentry{ext\_from\_Abs\_and\_Cmass()}\spxextra{in module skinoptics.absorption\_coefficient}}

\begin{fulllineitems}
\phantomsection\label{\detokenize{03_absorption_coefficient:skinoptics.absorption_coefficient.ext_from_Abs_and_Cmass}}
\pysigstartsignatures
\pysiglinewithargsret{\sphinxcode{\sphinxupquote{skinoptics.absorption\_coefficient.}}\sphinxbfcode{\sphinxupquote{ext\_from\_Abs\_and\_Cmass}}}{\sphinxparam{\DUrole{n}{Abs}}\sphinxparamcomma \sphinxparam{\DUrole{n}{Cmass}}\sphinxparamcomma \sphinxparam{\DUrole{n}{pathlength}}}{}
\pysigstopsignatures
\begin{DUlineblock}{0em}
\item[] Calculate the extinction coefficient from the absorbance, the mass concentration
\item[] and the pathlength.
\item[] For details please check Jacques 2013 {[}J13{]}.
\end{DUlineblock}

\sphinxAtStartPar
\(\varepsilon_{mass} = \frac{Abs}{C_{mass} L}\)
\begin{quote}\begin{description}
\sphinxlineitem{Parameters}\begin{itemize}
\item {} 
\sphinxAtStartPar
\sphinxstyleliteralstrong{\sphinxupquote{Abs}} (\sphinxstyleliteralemphasis{\sphinxupquote{float}}\sphinxstyleliteralemphasis{\sphinxupquote{ or }}\sphinxstyleliteralemphasis{\sphinxupquote{np.ndarray}}) \textendash{} absorbance {[}\sphinxhyphen{}{]}

\item {} 
\sphinxAtStartPar
\sphinxstyleliteralstrong{\sphinxupquote{Cmass}} (\sphinxstyleliteralemphasis{\sphinxupquote{float}}) \textendash{} mass concentration {[}g L\sphinxhyphen{}1{]}

\item {} 
\sphinxAtStartPar
\sphinxstyleliteralstrong{\sphinxupquote{pathlength}} (\sphinxstyleliteralemphasis{\sphinxupquote{float}}) \textendash{} pathlength {[}cm{]}

\end{itemize}

\sphinxlineitem{Returns}
\sphinxAtStartPar
\begin{itemize}
\item {} 
\sphinxAtStartPar
\sphinxstylestrong{ext} (\sphinxstyleemphasis{float or np.ndarray}) \textendash{} extinction coefficient {[}cm\sphinxhyphen{}1 mL mg\sphinxhyphen{}1{]}

\end{itemize}


\end{description}\end{quote}

\end{fulllineitems}

\index{ext\_from\_molarext() (in module skinoptics.absorption\_coefficient)@\spxentry{ext\_from\_molarext()}\spxextra{in module skinoptics.absorption\_coefficient}}

\begin{fulllineitems}
\phantomsection\label{\detokenize{03_absorption_coefficient:skinoptics.absorption_coefficient.ext_from_molarext}}
\pysigstartsignatures
\pysiglinewithargsret{\sphinxcode{\sphinxupquote{skinoptics.absorption\_coefficient.}}\sphinxbfcode{\sphinxupquote{ext\_from\_molarext}}}{\sphinxparam{\DUrole{n}{molarext}}\sphinxparamcomma \sphinxparam{\DUrole{n}{molar\_mass}}}{}
\pysigstopsignatures
\begin{DUlineblock}{0em}
\item[] Calculate the extinction coefficient from the molar extinction coefficient
\item[] and the molar mass.
\end{DUlineblock}

\sphinxAtStartPar
\(\varepsilon_{mass} = \frac{\varepsilon_{molar}}{M}\)
\begin{quote}\begin{description}
\sphinxlineitem{Parameters}\begin{itemize}
\item {} 
\sphinxAtStartPar
\sphinxstyleliteralstrong{\sphinxupquote{molarext}} (\sphinxstyleliteralemphasis{\sphinxupquote{float}}\sphinxstyleliteralemphasis{\sphinxupquote{ or }}\sphinxstyleliteralemphasis{\sphinxupquote{np.ndarray}}) \textendash{} molar extinction coefficient {[}cm\sphinxhyphen{}1 M\sphinxhyphen{}1{]}

\item {} 
\sphinxAtStartPar
\sphinxstyleliteralstrong{\sphinxupquote{molar\_mass}} (\sphinxstyleliteralemphasis{\sphinxupquote{float}}) \textendash{} molar mass {[}g mol\sphinxhyphen{}1{]}

\end{itemize}

\sphinxlineitem{Returns}
\sphinxAtStartPar
\begin{itemize}
\item {} 
\sphinxAtStartPar
\sphinxstylestrong{ext} (\sphinxstyleemphasis{float or np.ndarray}) \textendash{} extinction coefficient {[}cm\sphinxhyphen{}1 mL mg\sphinxhyphen{}1{]}

\end{itemize}


\end{description}\end{quote}

\end{fulllineitems}

\index{ext\_from\_mua\_and\_Cmass() (in module skinoptics.absorption\_coefficient)@\spxentry{ext\_from\_mua\_and\_Cmass()}\spxextra{in module skinoptics.absorption\_coefficient}}

\begin{fulllineitems}
\phantomsection\label{\detokenize{03_absorption_coefficient:skinoptics.absorption_coefficient.ext_from_mua_and_Cmass}}
\pysigstartsignatures
\pysiglinewithargsret{\sphinxcode{\sphinxupquote{skinoptics.absorption\_coefficient.}}\sphinxbfcode{\sphinxupquote{ext\_from\_mua\_and\_Cmass}}}{\sphinxparam{\DUrole{n}{mua}}\sphinxparamcomma \sphinxparam{\DUrole{n}{Cmass}}}{}
\pysigstopsignatures
\begin{DUlineblock}{0em}
\item[] Calculate the extinction coefficient from the absorption coefficient
\item[] and the mass concentration.
\item[] For details please check Jacques 2013 {[}J13{]}.
\end{DUlineblock}

\sphinxAtStartPar
\(\varepsilon_{mass} = \frac{1}{\mbox{ln}(10)}\frac{\mu_a}{C_{mass}}\)
\begin{quote}\begin{description}
\sphinxlineitem{Parameters}\begin{itemize}
\item {} 
\sphinxAtStartPar
\sphinxstyleliteralstrong{\sphinxupquote{mua}} (\sphinxstyleliteralemphasis{\sphinxupquote{float}}\sphinxstyleliteralemphasis{\sphinxupquote{ or }}\sphinxstyleliteralemphasis{\sphinxupquote{np.ndarray}}) \textendash{} absorption coefficient {[}mm\sphinxhyphen{}1{]}

\item {} 
\sphinxAtStartPar
\sphinxstyleliteralstrong{\sphinxupquote{Cmass}} (\sphinxstyleliteralemphasis{\sphinxupquote{float}}) \textendash{} mass concentration {[}g L\sphinxhyphen{}1{]}

\end{itemize}

\sphinxlineitem{Returns}
\sphinxAtStartPar
\begin{itemize}
\item {} 
\sphinxAtStartPar
\sphinxstylestrong{ext} (\sphinxstyleemphasis{float or np.ndarray}) \textendash{} extinction coefficient {[}cm\sphinxhyphen{}1 mL mg\sphinxhyphen{}1{]}

\end{itemize}


\end{description}\end{quote}

\end{fulllineitems}

\index{ext\_phe\_Sarna() (in module skinoptics.absorption\_coefficient)@\spxentry{ext\_phe\_Sarna()}\spxextra{in module skinoptics.absorption\_coefficient}}

\begin{fulllineitems}
\phantomsection\label{\detokenize{03_absorption_coefficient:skinoptics.absorption_coefficient.ext_phe_Sarna}}
\pysigstartsignatures
\pysiglinewithargsret{\sphinxcode{\sphinxupquote{skinoptics.absorption\_coefficient.}}\sphinxbfcode{\sphinxupquote{ext\_phe\_Sarna}}}{\sphinxparam{\DUrole{n}{lambda0}}}{}
\pysigstopsignatures
\begin{DUlineblock}{0em}
\item[] The extinction coefficient of PHEOMELANIN in phosphate buffer as a function of wavelength.
\item[] Linear interpolation of experimental data from Sarna \& Swartz 2006 {[}SS06{]}
\item[] (see their Fig. 16.3\sphinxhyphen{}a) graphically deduced by Jacques and publicly available at
\item[] \textless{}\sphinxurl{https://omlc.org/spectra/melanin/extcoeff.html}\textgreater{}.
\end{DUlineblock}

\begin{DUlineblock}{0em}
\item[] wavelength range: {[}210 nm, 820 nm{]}
\end{DUlineblock}
\begin{quote}\begin{description}
\sphinxlineitem{Parameters}
\sphinxAtStartPar
\sphinxstyleliteralstrong{\sphinxupquote{lambda0}} (\sphinxstyleliteralemphasis{\sphinxupquote{float}}\sphinxstyleliteralemphasis{\sphinxupquote{ or }}\sphinxstyleliteralemphasis{\sphinxupquote{np.ndarray}}) \textendash{} wavelength {[}nm{]}

\sphinxlineitem{Returns}
\sphinxAtStartPar
\begin{itemize}
\item {} 
\sphinxAtStartPar
\sphinxstylestrong{ext} (\sphinxstyleemphasis{float or np.ndarray}) \textendash{} extinction coefficient {[}cm\sphinxhyphen{}1 mL mg\sphinxhyphen{}1{]}

\end{itemize}


\end{description}\end{quote}

\end{fulllineitems}

\index{k\_from\_mua() (in module skinoptics.absorption\_coefficient)@\spxentry{k\_from\_mua()}\spxextra{in module skinoptics.absorption\_coefficient}}

\begin{fulllineitems}
\phantomsection\label{\detokenize{03_absorption_coefficient:skinoptics.absorption_coefficient.k_from_mua}}
\pysigstartsignatures
\pysiglinewithargsret{\sphinxcode{\sphinxupquote{skinoptics.absorption\_coefficient.}}\sphinxbfcode{\sphinxupquote{k\_from\_mua}}}{\sphinxparam{\DUrole{n}{mua}}\sphinxparamcomma \sphinxparam{\DUrole{n}{lambda0}}}{}
\pysigstopsignatures
\begin{DUlineblock}{0em}
\item[] Calculate the imaginary part of the complex refractive indext from the absorption coefficient
\item[] and the wavelength.
\item[] For details please check Hecht 2002 {[}H02{]}, Jacques 2013 {[}J13{]} and Griffiths 2017 {[}G17{]}.
\end{DUlineblock}

\sphinxAtStartPar
\(k(\lambda) = \frac{\mu_a(\lambda)\lambda}{4\pi}\)
\begin{quote}\begin{description}
\sphinxlineitem{Parameters}\begin{itemize}
\item {} 
\sphinxAtStartPar
\sphinxstyleliteralstrong{\sphinxupquote{mua}} (\sphinxstyleliteralemphasis{\sphinxupquote{float}}\sphinxstyleliteralemphasis{\sphinxupquote{ or }}\sphinxstyleliteralemphasis{\sphinxupquote{np.ndarray}}) \textendash{} absorption coefficient {[}mm\sphinxhyphen{}1{]}

\item {} 
\sphinxAtStartPar
\sphinxstyleliteralstrong{\sphinxupquote{lambda0}} (\sphinxstyleliteralemphasis{\sphinxupquote{float}}\sphinxstyleliteralemphasis{\sphinxupquote{ or }}\sphinxstyleliteralemphasis{\sphinxupquote{np.ndarray}}) \textendash{} wavelength {[}nm{]}

\end{itemize}

\sphinxlineitem{Returns}
\sphinxAtStartPar
\begin{itemize}
\item {} 
\sphinxAtStartPar
\sphinxstylestrong{k} (\sphinxstyleemphasis{float or np.ndarray}) \textendash{} imaginary part of the complex refractive index {[}\sphinxhyphen{}{]}

\end{itemize}


\end{description}\end{quote}

\end{fulllineitems}

\index{k\_wat\_Hale() (in module skinoptics.absorption\_coefficient)@\spxentry{k\_wat\_Hale()}\spxextra{in module skinoptics.absorption\_coefficient}}

\begin{fulllineitems}
\phantomsection\label{\detokenize{03_absorption_coefficient:skinoptics.absorption_coefficient.k_wat_Hale}}
\pysigstartsignatures
\pysiglinewithargsret{\sphinxcode{\sphinxupquote{skinoptics.absorption\_coefficient.}}\sphinxbfcode{\sphinxupquote{k\_wat\_Hale}}}{\sphinxparam{\DUrole{n}{lambda0}}}{}
\pysigstopsignatures
\begin{DUlineblock}{0em}
\item[] The imaginary part of the complex refractive index of WATER as a function of wavelength.
\item[] Linear interpolation of data from Hale \& Querry 1973 {[}HQ73{]} (see their Table I).
\end{DUlineblock}

\begin{DUlineblock}{0em}
\item[] wavelength range: {[}200 nm, 200 \(\mu\) m{]}
\item[] temperature: 25 ºC
\end{DUlineblock}
\begin{quote}\begin{description}
\sphinxlineitem{Parameters}
\sphinxAtStartPar
\sphinxstyleliteralstrong{\sphinxupquote{lambda0}} (\sphinxstyleliteralemphasis{\sphinxupquote{float}}\sphinxstyleliteralemphasis{\sphinxupquote{ or }}\sphinxstyleliteralemphasis{\sphinxupquote{np.ndarray}}) \textendash{} wavelength {[}nm{]}

\sphinxlineitem{Returns}
\sphinxAtStartPar
\begin{itemize}
\item {} 
\sphinxAtStartPar
\sphinxstylestrong{k} (\sphinxstyleemphasis{float or np.ndarray}) \textendash{} imaginary part of the complex refractive index {[}\sphinxhyphen{}{]}

\end{itemize}


\end{description}\end{quote}

\end{fulllineitems}

\index{k\_wat\_Segelstein() (in module skinoptics.absorption\_coefficient)@\spxentry{k\_wat\_Segelstein()}\spxextra{in module skinoptics.absorption\_coefficient}}

\begin{fulllineitems}
\phantomsection\label{\detokenize{03_absorption_coefficient:skinoptics.absorption_coefficient.k_wat_Segelstein}}
\pysigstartsignatures
\pysiglinewithargsret{\sphinxcode{\sphinxupquote{skinoptics.absorption\_coefficient.}}\sphinxbfcode{\sphinxupquote{k\_wat\_Segelstein}}}{\sphinxparam{\DUrole{n}{lambda0}}}{}
\pysigstopsignatures
\begin{DUlineblock}{0em}
\item[] The imaginary part of the complex refractive index of WATER as a function of wavelength.
\item[] Linear interpolation of data from D. J. Segelstein’s M.S. Thesis 1981 {[}S81{]} collected
\item[] by S. Prahl and publicly available at \textless{}\sphinxurl{https://omlc.org/spectra/water/abs/index.html}\textgreater{}.
\end{DUlineblock}

\begin{DUlineblock}{0em}
\item[] wavelength range: {[}10 nm, 10 m{]}.
\end{DUlineblock}
\begin{quote}\begin{description}
\sphinxlineitem{Parameters}
\sphinxAtStartPar
\sphinxstyleliteralstrong{\sphinxupquote{lambda0}} (\sphinxstyleliteralemphasis{\sphinxupquote{float}}\sphinxstyleliteralemphasis{\sphinxupquote{ or }}\sphinxstyleliteralemphasis{\sphinxupquote{np.ndarray}}) \textendash{} wavelength {[}nm{]}

\sphinxlineitem{Returns}
\sphinxAtStartPar
\begin{itemize}
\item {} 
\sphinxAtStartPar
\sphinxstylestrong{k} (\sphinxstyleemphasis{float or np.ndarray}) \textendash{} imaginary part of the complex refractive index {[}\sphinxhyphen{}{]}

\end{itemize}


\end{description}\end{quote}

\end{fulllineitems}

\index{molarext\_bil\_Li() (in module skinoptics.absorption\_coefficient)@\spxentry{molarext\_bil\_Li()}\spxextra{in module skinoptics.absorption\_coefficient}}

\begin{fulllineitems}
\phantomsection\label{\detokenize{03_absorption_coefficient:skinoptics.absorption_coefficient.molarext_bil_Li}}
\pysigstartsignatures
\pysiglinewithargsret{\sphinxcode{\sphinxupquote{skinoptics.absorption\_coefficient.}}\sphinxbfcode{\sphinxupquote{molarext\_bil\_Li}}}{\sphinxparam{\DUrole{n}{lambda0}}}{}
\pysigstopsignatures
\begin{DUlineblock}{0em}
\item[] The molar extinction coefficient of BILIRUBIN in chloroform as a function of wavelength.
\item[] Linear interpolation of experimental data obtained by J. Li on 1997 with a Cary 3,
\item[] scaled to match 55,000 cm\sphinxhyphen{}1 M\sphinxhyphen{}1 at 450.8 nm {[}AF90{]} and publicly available by S. Jacques and
\item[] S. Prahl at \textless{}\sphinxurl{https://omlc.org/spectra/PhotochemCAD/html/119.html}\textgreater{}.
\item[] The data is also available at PhotochemCAD {[}TL23{]}
\item[] \textless{}\sphinxurl{https://www.photochemcad.com/databases/common-compounds/oligopyrroles/bilirubin}\textgreater{}.
\end{DUlineblock}

\begin{DUlineblock}{0em}
\item[] wavelength range: {[}239.75 nm, 700 nm{]}
\end{DUlineblock}
\begin{quote}\begin{description}
\sphinxlineitem{Parameters}
\sphinxAtStartPar
\sphinxstyleliteralstrong{\sphinxupquote{lambda0}} (\sphinxstyleliteralemphasis{\sphinxupquote{float}}\sphinxstyleliteralemphasis{\sphinxupquote{ or }}\sphinxstyleliteralemphasis{\sphinxupquote{np.ndarray}}) \textendash{} wavelength {[}nm{]}

\sphinxlineitem{Returns}
\sphinxAtStartPar
\begin{itemize}
\item {} 
\sphinxAtStartPar
\sphinxstylestrong{molarext} (\sphinxstyleemphasis{float or np.ndarray}) \textendash{} molar extinction coefficient {[}cm\sphinxhyphen{}1 M\sphinxhyphen{}1{]}

\end{itemize}


\end{description}\end{quote}

\end{fulllineitems}

\index{molarext\_deo\_Prahl() (in module skinoptics.absorption\_coefficient)@\spxentry{molarext\_deo\_Prahl()}\spxextra{in module skinoptics.absorption\_coefficient}}

\begin{fulllineitems}
\phantomsection\label{\detokenize{03_absorption_coefficient:skinoptics.absorption_coefficient.molarext_deo_Prahl}}
\pysigstartsignatures
\pysiglinewithargsret{\sphinxcode{\sphinxupquote{skinoptics.absorption\_coefficient.}}\sphinxbfcode{\sphinxupquote{molarext\_deo\_Prahl}}}{\sphinxparam{\DUrole{n}{lambda0}}}{}
\pysigstopsignatures
\begin{DUlineblock}{0em}
\item[] The molar extinction coefficient for DEOXY\sphinxhyphen{}HEMOGLOBIN in water as a function of wavelength.
\item[] Linear interpolation of data from various sources compiled by S. Prahl and publicly
\item[] available at \textless{}\sphinxurl{https://omlc.org/spectra/hemoglobin/}\textgreater{}.
\end{DUlineblock}

\begin{DUlineblock}{0em}
\item[] wavelength range: {[}250 nm, 1000 nm{]}
\end{DUlineblock}
\begin{quote}\begin{description}
\sphinxlineitem{Parameters}
\sphinxAtStartPar
\sphinxstyleliteralstrong{\sphinxupquote{lambda0}} (\sphinxstyleliteralemphasis{\sphinxupquote{float}}\sphinxstyleliteralemphasis{\sphinxupquote{ or }}\sphinxstyleliteralemphasis{\sphinxupquote{np.ndarray}}) \textendash{} wavelength {[}nm{]}

\sphinxlineitem{Returns}
\sphinxAtStartPar
\begin{itemize}
\item {} 
\sphinxAtStartPar
\sphinxstylestrong{molarext} (\sphinxstyleemphasis{float or np.ndarray}) \textendash{} extinction coefficient {[}cm\sphinxhyphen{}1 M\sphinxhyphen{}1{]}

\end{itemize}


\end{description}\end{quote}

\end{fulllineitems}

\index{molarext\_eum\_Sarna() (in module skinoptics.absorption\_coefficient)@\spxentry{molarext\_eum\_Sarna()}\spxextra{in module skinoptics.absorption\_coefficient}}

\begin{fulllineitems}
\phantomsection\label{\detokenize{03_absorption_coefficient:skinoptics.absorption_coefficient.molarext_eum_Sarna}}
\pysigstartsignatures
\pysiglinewithargsret{\sphinxcode{\sphinxupquote{skinoptics.absorption\_coefficient.}}\sphinxbfcode{\sphinxupquote{molarext\_eum\_Sarna}}}{\sphinxparam{\DUrole{n}{lambda0}}}{}
\pysigstopsignatures
\begin{DUlineblock}{0em}
\item[] The molar extinction coefficient of EUMELANIN in phosphate buffer as a function of wavelength.
\item[] Linear interpolation of experimental data from Sarna \& Swartz 2006 {[}SS06{]}
\item[] (see their Fig. 16.3\sphinxhyphen{}a) graphically deduced by Jacques and publicly available at
\item[] \textless{}\sphinxurl{https://omlc.org/spectra/melanin/extcoeff.html}\textgreater{}.
\end{DUlineblock}

\begin{DUlineblock}{0em}
\item[] wavelength range: {[}210 nm, 820 nm{]}
\end{DUlineblock}
\begin{quote}\begin{description}
\sphinxlineitem{Parameters}
\sphinxAtStartPar
\sphinxstyleliteralstrong{\sphinxupquote{lambda0}} (\sphinxstyleliteralemphasis{\sphinxupquote{float}}\sphinxstyleliteralemphasis{\sphinxupquote{ or }}\sphinxstyleliteralemphasis{\sphinxupquote{np.ndarray}}) \textendash{} wavelength {[}nm{]}

\sphinxlineitem{Returns}
\sphinxAtStartPar
\begin{itemize}
\item {} 
\sphinxAtStartPar
\sphinxstylestrong{molarext} (\sphinxstyleemphasis{float or np.ndarray}) \textendash{} molar extinction coefficient {[}cm\sphinxhyphen{}1 M\sphinxhyphen{}1{]}

\end{itemize}


\end{description}\end{quote}

\end{fulllineitems}

\index{molarext\_from\_Abs\_and\_Cmolar() (in module skinoptics.absorption\_coefficient)@\spxentry{molarext\_from\_Abs\_and\_Cmolar()}\spxextra{in module skinoptics.absorption\_coefficient}}

\begin{fulllineitems}
\phantomsection\label{\detokenize{03_absorption_coefficient:skinoptics.absorption_coefficient.molarext_from_Abs_and_Cmolar}}
\pysigstartsignatures
\pysiglinewithargsret{\sphinxcode{\sphinxupquote{skinoptics.absorption\_coefficient.}}\sphinxbfcode{\sphinxupquote{molarext\_from\_Abs\_and\_Cmolar}}}{\sphinxparam{\DUrole{n}{Abs}}\sphinxparamcomma \sphinxparam{\DUrole{n}{Cmolar}}\sphinxparamcomma \sphinxparam{\DUrole{n}{pathlength}}}{}
\pysigstopsignatures
\begin{DUlineblock}{0em}
\item[] Calculate the molar extinction coefficient from the absorbance, the molar concentration
\item[] and the pathlength.
\item[] For details please check Jacques 2013 {[}J13{]}.
\end{DUlineblock}

\sphinxAtStartPar
\(\varepsilon_{molar} = \frac{Abs}{C_{molar} L}\)
\begin{quote}\begin{description}
\sphinxlineitem{Parameters}\begin{itemize}
\item {} 
\sphinxAtStartPar
\sphinxstyleliteralstrong{\sphinxupquote{Abs}} (\sphinxstyleliteralemphasis{\sphinxupquote{float}}\sphinxstyleliteralemphasis{\sphinxupquote{ or }}\sphinxstyleliteralemphasis{\sphinxupquote{np.ndarray}}) \textendash{} absorbance {[}\sphinxhyphen{}{]}

\item {} 
\sphinxAtStartPar
\sphinxstyleliteralstrong{\sphinxupquote{Cmolar}} (\sphinxstyleliteralemphasis{\sphinxupquote{float}}) \textendash{} molar concentration {[}g L\sphinxhyphen{}1{]}

\item {} 
\sphinxAtStartPar
\sphinxstyleliteralstrong{\sphinxupquote{pathlength}} (\sphinxstyleliteralemphasis{\sphinxupquote{float}}) \textendash{} pathlength {[}cm{]}

\end{itemize}

\sphinxlineitem{Returns}
\sphinxAtStartPar
\begin{itemize}
\item {} 
\sphinxAtStartPar
\sphinxstylestrong{molarext} (\sphinxstyleemphasis{float or np.ndarray}) \textendash{} molar extinction coefficient {[}cm\sphinxhyphen{}1 M\sphinxhyphen{}1{]}

\end{itemize}


\end{description}\end{quote}

\end{fulllineitems}

\index{molarext\_from\_ext() (in module skinoptics.absorption\_coefficient)@\spxentry{molarext\_from\_ext()}\spxextra{in module skinoptics.absorption\_coefficient}}

\begin{fulllineitems}
\phantomsection\label{\detokenize{03_absorption_coefficient:skinoptics.absorption_coefficient.molarext_from_ext}}
\pysigstartsignatures
\pysiglinewithargsret{\sphinxcode{\sphinxupquote{skinoptics.absorption\_coefficient.}}\sphinxbfcode{\sphinxupquote{molarext\_from\_ext}}}{\sphinxparam{\DUrole{n}{ext}}\sphinxparamcomma \sphinxparam{\DUrole{n}{molar\_mass}}}{}
\pysigstopsignatures
\begin{DUlineblock}{0em}
\item[] Calculate the molar extinction coefficient from the extinction coefficient
\item[] and the molar mass.
\end{DUlineblock}

\sphinxAtStartPar
\(\varepsilon_{molar} = M \varepsilon_{mass}\)
\begin{quote}\begin{description}
\sphinxlineitem{Parameters}\begin{itemize}
\item {} 
\sphinxAtStartPar
\sphinxstyleliteralstrong{\sphinxupquote{ext}} (\sphinxstyleliteralemphasis{\sphinxupquote{float}}\sphinxstyleliteralemphasis{\sphinxupquote{ or }}\sphinxstyleliteralemphasis{\sphinxupquote{np.ndarray}}) \textendash{} extinction coefficient {[}cm\sphinxhyphen{}1 mL mg\sphinxhyphen{}1{]}

\item {} 
\sphinxAtStartPar
\sphinxstyleliteralstrong{\sphinxupquote{molar\_mass}} (\sphinxstyleliteralemphasis{\sphinxupquote{float}}) \textendash{} molar mass {[}g mol\sphinxhyphen{}1{]}

\end{itemize}

\sphinxlineitem{Returns}
\sphinxAtStartPar
\begin{itemize}
\item {} 
\sphinxAtStartPar
\sphinxstylestrong{molarext} (\sphinxstyleemphasis{float or np.ndarray}) \textendash{} molar extinction coefficient {[}cm\sphinxhyphen{}1 M\sphinxhyphen{}1{]}

\end{itemize}


\end{description}\end{quote}

\end{fulllineitems}

\index{molarext\_from\_mua\_Cmolar() (in module skinoptics.absorption\_coefficient)@\spxentry{molarext\_from\_mua\_Cmolar()}\spxextra{in module skinoptics.absorption\_coefficient}}

\begin{fulllineitems}
\phantomsection\label{\detokenize{03_absorption_coefficient:skinoptics.absorption_coefficient.molarext_from_mua_Cmolar}}
\pysigstartsignatures
\pysiglinewithargsret{\sphinxcode{\sphinxupquote{skinoptics.absorption\_coefficient.}}\sphinxbfcode{\sphinxupquote{molarext\_from\_mua\_Cmolar}}}{\sphinxparam{\DUrole{n}{mua}}\sphinxparamcomma \sphinxparam{\DUrole{n}{Cmolar}}}{}
\pysigstopsignatures
\begin{DUlineblock}{0em}
\item[] Calculate the molar extinction coefficient from the absorption coefficient
\item[] and the molar concentration.
\item[] For details please check Jacques 2013 {[}J13{]}.
\end{DUlineblock}

\sphinxAtStartPar
\(\varepsilon_{molar} = \frac{1}{\mbox{ln}(10)}\frac{\mu_a}{C_{molar}}\)
\begin{quote}\begin{description}
\sphinxlineitem{Parameters}\begin{itemize}
\item {} 
\sphinxAtStartPar
\sphinxstyleliteralstrong{\sphinxupquote{mua}} (\sphinxstyleliteralemphasis{\sphinxupquote{float}}\sphinxstyleliteralemphasis{\sphinxupquote{ or }}\sphinxstyleliteralemphasis{\sphinxupquote{np.ndarray}}) \textendash{} absorption coefficient {[}mm\sphinxhyphen{}1{]}

\item {} 
\sphinxAtStartPar
\sphinxstyleliteralstrong{\sphinxupquote{Cmolar}} (\sphinxstyleliteralemphasis{\sphinxupquote{float}}) \textendash{} molar concentration {[}M{]}

\end{itemize}

\sphinxlineitem{Returns}
\sphinxAtStartPar
\begin{itemize}
\item {} 
\sphinxAtStartPar
\sphinxstylestrong{molarext} (\sphinxstyleemphasis{float or np.ndarray}) \textendash{} molar extinction coefficient {[}cm\sphinxhyphen{}1 M\sphinxhyphen{}1{]}

\end{itemize}


\end{description}\end{quote}

\end{fulllineitems}

\index{molarext\_oxy\_Prahl() (in module skinoptics.absorption\_coefficient)@\spxentry{molarext\_oxy\_Prahl()}\spxextra{in module skinoptics.absorption\_coefficient}}

\begin{fulllineitems}
\phantomsection\label{\detokenize{03_absorption_coefficient:skinoptics.absorption_coefficient.molarext_oxy_Prahl}}
\pysigstartsignatures
\pysiglinewithargsret{\sphinxcode{\sphinxupquote{skinoptics.absorption\_coefficient.}}\sphinxbfcode{\sphinxupquote{molarext\_oxy\_Prahl}}}{\sphinxparam{\DUrole{n}{lambda0}}}{}
\pysigstopsignatures
\begin{DUlineblock}{0em}
\item[] The molar extinction coefficient of OXY\sphinxhyphen{}HEMOGLOBIN in water as a function of wavelength.
\item[] Linear interpolation of data from various sources compiled by S. Prahl and publicly
\item[] available at \textless{}\sphinxurl{https://omlc.org/spectra/hemoglobin/}\textgreater{}.
\end{DUlineblock}

\begin{DUlineblock}{0em}
\item[] wavelength range: {[}250 nm, 1000 nm{]}
\end{DUlineblock}
\begin{quote}\begin{description}
\sphinxlineitem{Parameters}
\sphinxAtStartPar
\sphinxstyleliteralstrong{\sphinxupquote{lambda0}} (\sphinxstyleliteralemphasis{\sphinxupquote{float}}\sphinxstyleliteralemphasis{\sphinxupquote{ or }}\sphinxstyleliteralemphasis{\sphinxupquote{np.ndarray}}) \textendash{} wavelength {[}nm{]}

\sphinxlineitem{Returns}
\sphinxAtStartPar
\begin{itemize}
\item {} 
\sphinxAtStartPar
\sphinxstylestrong{molarext} (\sphinxstyleemphasis{float or np.ndarray}) \textendash{} molar extinction coefficient {[}cm\sphinxhyphen{}1 M\sphinxhyphen{}1{]}

\end{itemize}


\end{description}\end{quote}

\end{fulllineitems}

\index{molarext\_phe\_Sarna() (in module skinoptics.absorption\_coefficient)@\spxentry{molarext\_phe\_Sarna()}\spxextra{in module skinoptics.absorption\_coefficient}}

\begin{fulllineitems}
\phantomsection\label{\detokenize{03_absorption_coefficient:skinoptics.absorption_coefficient.molarext_phe_Sarna}}
\pysigstartsignatures
\pysiglinewithargsret{\sphinxcode{\sphinxupquote{skinoptics.absorption\_coefficient.}}\sphinxbfcode{\sphinxupquote{molarext\_phe\_Sarna}}}{\sphinxparam{\DUrole{n}{lambda0}}}{}
\pysigstopsignatures
\begin{DUlineblock}{0em}
\item[] The molar extinction coefficient of PHEOMELANIN in phosphate buffer as a function of wavelength.
\item[] Linear interpolation of experimental data from Sarna \& Swartz 2006 {[}SS06{]}
\item[] (see their Fig. 16.3\sphinxhyphen{}a) graphically deduced by Jacques and publicly available at
\item[] \textless{}\sphinxurl{https://omlc.org/spectra/melanin/extcoeff.html}\textgreater{}.
\end{DUlineblock}

\begin{DUlineblock}{0em}
\item[] wavelength range: {[}210 nm, 820 nm{]}
\end{DUlineblock}
\begin{quote}\begin{description}
\sphinxlineitem{Parameters}
\sphinxAtStartPar
\sphinxstyleliteralstrong{\sphinxupquote{lambda0}} (\sphinxstyleliteralemphasis{\sphinxupquote{float}}\sphinxstyleliteralemphasis{\sphinxupquote{ or }}\sphinxstyleliteralemphasis{\sphinxupquote{np.ndarray}}) \textendash{} wavelength {[}nm{]}

\sphinxlineitem{Returns}
\sphinxAtStartPar
\begin{itemize}
\item {} 
\sphinxAtStartPar
\sphinxstylestrong{molarext} (\sphinxstyleemphasis{float or np.ndarray}) \textendash{} molar extinction coefficient {[}cm\sphinxhyphen{}1 M\sphinxhyphen{}1{]}

\end{itemize}


\end{description}\end{quote}

\end{fulllineitems}

\index{mua\_DE\_Salomatina() (in module skinoptics.absorption\_coefficient)@\spxentry{mua\_DE\_Salomatina()}\spxextra{in module skinoptics.absorption\_coefficient}}

\begin{fulllineitems}
\phantomsection\label{\detokenize{03_absorption_coefficient:skinoptics.absorption_coefficient.mua_DE_Salomatina}}
\pysigstartsignatures
\pysiglinewithargsret{\sphinxcode{\sphinxupquote{skinoptics.absorption\_coefficient.}}\sphinxbfcode{\sphinxupquote{mua\_DE\_Salomatina}}}{\sphinxparam{\DUrole{n}{lambda0}}}{}
\pysigstopsignatures
\begin{DUlineblock}{0em}
\item[] The absoption coefficient of DERMIS as a function of wavelength.
\item[] Linear interpolation of experimental data from Salomatina et al. 2006 {[}S*06{]},
\item[] publicly available at \textless{}\sphinxurl{https://sites.uml.edu/abl/optical-properties-2/}\textgreater{}.
\end{DUlineblock}

\begin{DUlineblock}{0em}
\item[] wavelength range: {[}370 nm, 1600 nm{]}
\end{DUlineblock}
\begin{quote}\begin{description}
\sphinxlineitem{Parameters}
\sphinxAtStartPar
\sphinxstyleliteralstrong{\sphinxupquote{lambda0}} (\sphinxstyleliteralemphasis{\sphinxupquote{float}}\sphinxstyleliteralemphasis{\sphinxupquote{ or }}\sphinxstyleliteralemphasis{\sphinxupquote{np.ndarray}}) \textendash{} wavelength {[}nm{]}

\sphinxlineitem{Returns}
\sphinxAtStartPar
\begin{itemize}
\item {} 
\sphinxAtStartPar
\sphinxstylestrong{mua} (\sphinxstyleemphasis{float or np.ndarray}) \textendash{} absorption coefficient {[}mm\sphinxhyphen{}1{]}

\end{itemize}


\end{description}\end{quote}

\end{fulllineitems}

\index{mua\_EP\_Salomatina() (in module skinoptics.absorption\_coefficient)@\spxentry{mua\_EP\_Salomatina()}\spxextra{in module skinoptics.absorption\_coefficient}}

\begin{fulllineitems}
\phantomsection\label{\detokenize{03_absorption_coefficient:skinoptics.absorption_coefficient.mua_EP_Salomatina}}
\pysigstartsignatures
\pysiglinewithargsret{\sphinxcode{\sphinxupquote{skinoptics.absorption\_coefficient.}}\sphinxbfcode{\sphinxupquote{mua\_EP\_Salomatina}}}{\sphinxparam{\DUrole{n}{lambda0}}}{}
\pysigstopsignatures
\begin{DUlineblock}{0em}
\item[] The absoption coefficient of EPIDERMIS as a function of wavelength.
\item[] Linear interpolation of experimental data from Salomatina et al. 2006 {[}S*06{]},
\item[] publicly available at \textless{}\sphinxurl{https://sites.uml.edu/abl/optical-properties-2/}\textgreater{}.
\end{DUlineblock}

\begin{DUlineblock}{0em}
\item[] wavelength range: {[}370 nm, 1600 nm{]}
\end{DUlineblock}
\begin{quote}\begin{description}
\sphinxlineitem{Parameters}
\sphinxAtStartPar
\sphinxstyleliteralstrong{\sphinxupquote{lambda0}} (\sphinxstyleliteralemphasis{\sphinxupquote{float}}\sphinxstyleliteralemphasis{\sphinxupquote{ or }}\sphinxstyleliteralemphasis{\sphinxupquote{np.ndarray}}) \textendash{} wavelength {[}nm{]}

\sphinxlineitem{Returns}
\sphinxAtStartPar
\begin{itemize}
\item {} 
\sphinxAtStartPar
\sphinxstylestrong{mua} (\sphinxstyleemphasis{float or np.ndarray}) \textendash{} absorption coefficient {[}mm\sphinxhyphen{}1{]}

\end{itemize}


\end{description}\end{quote}

\end{fulllineitems}

\index{mua\_HY\_Salomatina() (in module skinoptics.absorption\_coefficient)@\spxentry{mua\_HY\_Salomatina()}\spxextra{in module skinoptics.absorption\_coefficient}}

\begin{fulllineitems}
\phantomsection\label{\detokenize{03_absorption_coefficient:skinoptics.absorption_coefficient.mua_HY_Salomatina}}
\pysigstartsignatures
\pysiglinewithargsret{\sphinxcode{\sphinxupquote{skinoptics.absorption\_coefficient.}}\sphinxbfcode{\sphinxupquote{mua\_HY\_Salomatina}}}{\sphinxparam{\DUrole{n}{lambda0}}}{}
\pysigstopsignatures
\begin{DUlineblock}{0em}
\item[] The absoption coefficient of HYPODERMIS as a function of wavelength.
\item[] Linear interpolation of experimental data from Salomatina et al. 2006 {[}S*06{]},
\item[] publicly available at \textless{}\sphinxurl{https://sites.uml.edu/abl/optical-properties-2/}\textgreater{}.
\end{DUlineblock}

\begin{DUlineblock}{0em}
\item[] wavelength range: {[}374 nm, 1600 nm{]}
\end{DUlineblock}
\begin{quote}\begin{description}
\sphinxlineitem{Parameters}
\sphinxAtStartPar
\sphinxstyleliteralstrong{\sphinxupquote{lambda0}} (\sphinxstyleliteralemphasis{\sphinxupquote{float}}\sphinxstyleliteralemphasis{\sphinxupquote{ or }}\sphinxstyleliteralemphasis{\sphinxupquote{np.ndarray}}) \textendash{} wavelength {[}nm{]}

\sphinxlineitem{Returns}
\sphinxAtStartPar
\begin{itemize}
\item {} 
\sphinxAtStartPar
\sphinxstylestrong{mua} (\sphinxstyleemphasis{float or np.ndarray}) \textendash{} absorption coefficient {[}mm\sphinxhyphen{}1{]}

\end{itemize}


\end{description}\end{quote}

\end{fulllineitems}

\index{mua\_SCC\_Salomatina() (in module skinoptics.absorption\_coefficient)@\spxentry{mua\_SCC\_Salomatina()}\spxextra{in module skinoptics.absorption\_coefficient}}

\begin{fulllineitems}
\phantomsection\label{\detokenize{03_absorption_coefficient:skinoptics.absorption_coefficient.mua_SCC_Salomatina}}
\pysigstartsignatures
\pysiglinewithargsret{\sphinxcode{\sphinxupquote{skinoptics.absorption\_coefficient.}}\sphinxbfcode{\sphinxupquote{mua\_SCC\_Salomatina}}}{\sphinxparam{\DUrole{n}{lambda0}}}{}
\pysigstopsignatures
\begin{DUlineblock}{0em}
\item[] The absoption coefficient of SQUAMOUS CELL CARCINOMA as a function of wavelength.
\item[] Linear interpolation of experimental data from Salomatina et al. 2006 {[}S*06{]},
\item[] publicly available at \textless{}\sphinxurl{https://sites.uml.edu/abl/optical-properties-2/}\textgreater{}.
\end{DUlineblock}

\begin{DUlineblock}{0em}
\item[] wavelength range: {[}370 nm, 1600 nm{]}
\end{DUlineblock}
\begin{quote}\begin{description}
\sphinxlineitem{Parameters}
\sphinxAtStartPar
\sphinxstyleliteralstrong{\sphinxupquote{lambda0}} (\sphinxstyleliteralemphasis{\sphinxupquote{float}}\sphinxstyleliteralemphasis{\sphinxupquote{ or }}\sphinxstyleliteralemphasis{\sphinxupquote{np.ndarray}}) \textendash{} wavelength {[}nm{]}

\sphinxlineitem{Returns}
\sphinxAtStartPar
\begin{itemize}
\item {} 
\sphinxAtStartPar
\sphinxstylestrong{mua} (\sphinxstyleemphasis{float or np.ndarray}) \textendash{} absorption coefficient {[}mm\sphinxhyphen{}1{]}

\end{itemize}


\end{description}\end{quote}

\end{fulllineitems}

\index{mua\_baseline() (in module skinoptics.absorption\_coefficient)@\spxentry{mua\_baseline()}\spxextra{in module skinoptics.absorption\_coefficient}}

\begin{fulllineitems}
\phantomsection\label{\detokenize{03_absorption_coefficient:skinoptics.absorption_coefficient.mua_baseline}}
\pysigstartsignatures
\pysiglinewithargsret{\sphinxcode{\sphinxupquote{skinoptics.absorption\_coefficient.}}\sphinxbfcode{\sphinxupquote{mua\_baseline}}}{\sphinxparam{\DUrole{n}{lambda0}}}{}
\pysigstopsignatures
\begin{DUlineblock}{0em}
\item[] The baseline absorption coefficient as a function of wavelength.
\item[] Equation proposed by S. Jacques based on data for bloodless rat skin.
\item[] For details please check \textless{}\sphinxurl{https://omlc.org/news/jan98/skinoptics.html}\textgreater{}.
\end{DUlineblock}

\sphinxAtStartPar
\(\mu_a(\lambda) = 0.0244 + 8.53\mbox{exp}(-(\lambda-154)/66.2)\)
\begin{quote}\begin{description}
\sphinxlineitem{Parameters}
\sphinxAtStartPar
\sphinxstyleliteralstrong{\sphinxupquote{lambda0}} (\sphinxstyleliteralemphasis{\sphinxupquote{float}}\sphinxstyleliteralemphasis{\sphinxupquote{ or }}\sphinxstyleliteralemphasis{\sphinxupquote{np.ndarray}}) \textendash{} wavelength {[}nm{]}

\sphinxlineitem{Returns}
\sphinxAtStartPar
\begin{itemize}
\item {} 
\sphinxAtStartPar
\sphinxstylestrong{mua} (\sphinxstyleemphasis{float or np.ndarray}) \textendash{} absorption coefficient {[}mm\sphinxhyphen{}1{]}

\end{itemize}


\end{description}\end{quote}

\end{fulllineitems}

\index{mua\_baseline2() (in module skinoptics.absorption\_coefficient)@\spxentry{mua\_baseline2()}\spxextra{in module skinoptics.absorption\_coefficient}}

\begin{fulllineitems}
\phantomsection\label{\detokenize{03_absorption_coefficient:skinoptics.absorption_coefficient.mua_baseline2}}
\pysigstartsignatures
\pysiglinewithargsret{\sphinxcode{\sphinxupquote{skinoptics.absorption\_coefficient.}}\sphinxbfcode{\sphinxupquote{mua\_baseline2}}}{\sphinxparam{\DUrole{n}{lambda0}}}{}
\pysigstopsignatures
\begin{DUlineblock}{0em}
\item[] The baseline absorption coefficient as a function of wavelength.
\item[] Equation based on data for neonatal skin.
\item[] For details please check \textless{}\sphinxurl{https://omlc.org/news/jan98/skinoptics.html}\textgreater{}.
\end{DUlineblock}

\sphinxAtStartPar
\(\mu_a(\lambda) = 7.84 \times 10^7 \lambda^{-3255}\)
\begin{quote}\begin{description}
\sphinxlineitem{Parameters}
\sphinxAtStartPar
\sphinxstyleliteralstrong{\sphinxupquote{lambda0}} (\sphinxstyleliteralemphasis{\sphinxupquote{float}}\sphinxstyleliteralemphasis{\sphinxupquote{ or }}\sphinxstyleliteralemphasis{\sphinxupquote{np.ndarray}}) \textendash{} wavelength {[}nm{]}

\sphinxlineitem{Returns}
\sphinxAtStartPar
\begin{itemize}
\item {} 
\sphinxAtStartPar
\sphinxstylestrong{mua} (\sphinxstyleemphasis{float or np.ndarray}) \textendash{} absorption coefficient {[}mm\sphinxhyphen{}1{]}

\end{itemize}


\end{description}\end{quote}

\end{fulllineitems}

\index{mua\_bil\_Li() (in module skinoptics.absorption\_coefficient)@\spxentry{mua\_bil\_Li()}\spxextra{in module skinoptics.absorption\_coefficient}}

\begin{fulllineitems}
\phantomsection\label{\detokenize{03_absorption_coefficient:skinoptics.absorption_coefficient.mua_bil_Li}}
\pysigstartsignatures
\pysiglinewithargsret{\sphinxcode{\sphinxupquote{skinoptics.absorption\_coefficient.}}\sphinxbfcode{\sphinxupquote{mua\_bil\_Li}}}{\sphinxparam{\DUrole{n}{lambda0}}\sphinxparamcomma \sphinxparam{\DUrole{n}{Cmass\_bil}}\sphinxparamcomma \sphinxparam{\DUrole{n}{molar\_mass\_bil}\DUrole{o}{=}\DUrole{default_value}{585}}}{}
\pysigstopsignatures
\begin{DUlineblock}{0em}
\item[] The absorption coefficient of BILIRUBIN as a function of wavelength.
\item[] Calculated from molarext\_bil\_Li for a specific bilirubin mass concentration.
\end{DUlineblock}

\begin{DUlineblock}{0em}
\item[] wavelength range: {[}239.75 nm, 700 nm{]}
\end{DUlineblock}
\begin{quote}\begin{description}
\sphinxlineitem{Parameters}\begin{itemize}
\item {} 
\sphinxAtStartPar
\sphinxstyleliteralstrong{\sphinxupquote{lambda0}} (\sphinxstyleliteralemphasis{\sphinxupquote{float}}\sphinxstyleliteralemphasis{\sphinxupquote{ or }}\sphinxstyleliteralemphasis{\sphinxupquote{np.ndarray}}) \textendash{} wavelength {[}nm{]}

\item {} 
\sphinxAtStartPar
\sphinxstyleliteralstrong{\sphinxupquote{Cmass\_bil}} (\sphinxstyleliteralemphasis{\sphinxupquote{float}}) \textendash{} bilirubin mass concentration {[}g/L{]}

\item {} 
\sphinxAtStartPar
\sphinxstyleliteralstrong{\sphinxupquote{molarmass\_bil}} (\sphinxstyleliteralemphasis{\sphinxupquote{float}}) \textendash{} molar mass of bilirubin {[}g/mol{]} (default to 585. {[}DJV11{]})

\end{itemize}

\sphinxlineitem{Returns}
\sphinxAtStartPar
\begin{itemize}
\item {} 
\sphinxAtStartPar
\sphinxstylestrong{mua} (\sphinxstyleemphasis{float or np.ndarray}) \textendash{} absorption coefficient {[}mm\sphinxhyphen{}1{]}

\end{itemize}


\end{description}\end{quote}

\end{fulllineitems}

\index{mua\_deo\_Bosschaart() (in module skinoptics.absorption\_coefficient)@\spxentry{mua\_deo\_Bosschaart()}\spxextra{in module skinoptics.absorption\_coefficient}}

\begin{fulllineitems}
\phantomsection\label{\detokenize{03_absorption_coefficient:skinoptics.absorption_coefficient.mua_deo_Bosschaart}}
\pysigstartsignatures
\pysiglinewithargsret{\sphinxcode{\sphinxupquote{skinoptics.absorption\_coefficient.}}\sphinxbfcode{\sphinxupquote{mua\_deo\_Bosschaart}}}{\sphinxparam{\DUrole{n}{lambda0}}}{}
\pysigstopsignatures
\begin{DUlineblock}{0em}
\item[] The absorption coefficient of DEOXIGENIZED BLOOD (saturation = 0\%) as a function of wavelength.
\item[] Linear interpolation of data from Bosschaart et. al. 2014 {[}B*14{]}.
\end{DUlineblock}

\begin{DUlineblock}{0em}
\item[] wavelength range: {[}251 nm, 1995 nm{]}
\end{DUlineblock}
\begin{quote}\begin{description}
\sphinxlineitem{Parameters}
\sphinxAtStartPar
\sphinxstyleliteralstrong{\sphinxupquote{lambda0}} (\sphinxstyleliteralemphasis{\sphinxupquote{float}}\sphinxstyleliteralemphasis{\sphinxupquote{ or }}\sphinxstyleliteralemphasis{\sphinxupquote{np.ndarray}}) \textendash{} wavelength {[}nm{]}

\sphinxlineitem{Returns}
\sphinxAtStartPar
\begin{itemize}
\item {} 
\sphinxAtStartPar
\sphinxstylestrong{mua} (\sphinxstyleemphasis{float or np.ndarray}) \textendash{} absorption coefficient {[}mm\sphinxhyphen{}1{]}

\end{itemize}


\end{description}\end{quote}

\end{fulllineitems}

\index{mua\_deo\_Prahl() (in module skinoptics.absorption\_coefficient)@\spxentry{mua\_deo\_Prahl()}\spxextra{in module skinoptics.absorption\_coefficient}}

\begin{fulllineitems}
\phantomsection\label{\detokenize{03_absorption_coefficient:skinoptics.absorption_coefficient.mua_deo_Prahl}}
\pysigstartsignatures
\pysiglinewithargsret{\sphinxcode{\sphinxupquote{skinoptics.absorption\_coefficient.}}\sphinxbfcode{\sphinxupquote{mua\_deo\_Prahl}}}{\sphinxparam{\DUrole{n}{lambda0}}\sphinxparamcomma \sphinxparam{\DUrole{n}{Cmass\_deo}\DUrole{o}{=}\DUrole{default_value}{150}}\sphinxparamcomma \sphinxparam{\DUrole{n}{molar\_mass\_deo}\DUrole{o}{=}\DUrole{default_value}{64500}}}{}
\pysigstopsignatures
\begin{DUlineblock}{0em}
\item[] The absorption coefficient of DEOXY\sphinxhyphen{}HEMOGLOBIN as a function of wavelength.
\item[] Calculated from molarext\_deo\_Prahl for a specific deoxy\sphinxhyphen{}hemoglobin mass concentration.
\end{DUlineblock}

\begin{DUlineblock}{0em}
\item[] wavelength range: {[}250 nm, 1000 nm{]}
\end{DUlineblock}
\begin{quote}\begin{description}
\sphinxlineitem{Parameters}\begin{itemize}
\item {} 
\sphinxAtStartPar
\sphinxstyleliteralstrong{\sphinxupquote{lambda0}} (\sphinxstyleliteralemphasis{\sphinxupquote{float}}\sphinxstyleliteralemphasis{\sphinxupquote{ or }}\sphinxstyleliteralemphasis{\sphinxupquote{np.ndarray.}}) \textendash{} wavelength {[}nm{]}

\item {} 
\sphinxAtStartPar
\sphinxstyleliteralstrong{\sphinxupquote{Cmass\_oxy}} (\sphinxstyleliteralemphasis{\sphinxupquote{float}}) \textendash{} deoxy\sphinxhyphen{}hemoglobin mass concentration {[}g/L{]} (default to 150. {[}S*23{]})

\item {} 
\sphinxAtStartPar
\sphinxstyleliteralstrong{\sphinxupquote{molar\_mass\_deo}} (\sphinxstyleliteralemphasis{\sphinxupquote{float}}) \textendash{} molar mass of deoxy\sphinxhyphen{}hemoglobin {[}g/mol{]} (default to 64500. {[}B90{]})

\end{itemize}

\sphinxlineitem{Returns}
\sphinxAtStartPar
\begin{itemize}
\item {} 
\sphinxAtStartPar
\sphinxstylestrong{mua} (\sphinxstyleemphasis{float or np.ndarray}) \textendash{} absorption coefficient {[}mm\sphinxhyphen{}1{]}

\end{itemize}


\end{description}\end{quote}

\end{fulllineitems}

\index{mua\_eum\_Jacques() (in module skinoptics.absorption\_coefficient)@\spxentry{mua\_eum\_Jacques()}\spxextra{in module skinoptics.absorption\_coefficient}}

\begin{fulllineitems}
\phantomsection\label{\detokenize{03_absorption_coefficient:skinoptics.absorption_coefficient.mua_eum_Jacques}}
\pysigstartsignatures
\pysiglinewithargsret{\sphinxcode{\sphinxupquote{skinoptics.absorption\_coefficient.}}\sphinxbfcode{\sphinxupquote{mua\_eum\_Jacques}}}{\sphinxparam{\DUrole{n}{lambda0}}}{}
\pysigstopsignatures
\begin{DUlineblock}{0em}
\item[] The absoption coefficient of EUMELANIN as a function of wavelength.
\item[] Equation proposed by S. Jacques based on data from various sources.
\item[] For details please check \textless{}\sphinxurl{https://omlc.org/news/jan98/skinoptics.html}\textgreater{}.
\end{DUlineblock}

\sphinxAtStartPar
\(\mu_a^{eum} (\lambda) = 6.6 \times 10^{10} \times \lambda^{-3.33}\)
\begin{quote}\begin{description}
\sphinxlineitem{Parameters}
\sphinxAtStartPar
\sphinxstyleliteralstrong{\sphinxupquote{lambda0}} (\sphinxstyleliteralemphasis{\sphinxupquote{float}}\sphinxstyleliteralemphasis{\sphinxupquote{ or }}\sphinxstyleliteralemphasis{\sphinxupquote{np.ndarray}}) \textendash{} wavelength {[}nm{]}

\sphinxlineitem{Returns}
\sphinxAtStartPar
\begin{itemize}
\item {} 
\sphinxAtStartPar
\sphinxstylestrong{mua} (\sphinxstyleemphasis{float or np.ndarray}) \textendash{} absorption coefficient {[}mm\sphinxhyphen{}1{]}

\end{itemize}


\end{description}\end{quote}

\end{fulllineitems}

\index{mua\_fat\_vanVeen() (in module skinoptics.absorption\_coefficient)@\spxentry{mua\_fat\_vanVeen()}\spxextra{in module skinoptics.absorption\_coefficient}}

\begin{fulllineitems}
\phantomsection\label{\detokenize{03_absorption_coefficient:skinoptics.absorption_coefficient.mua_fat_vanVeen}}
\pysigstartsignatures
\pysiglinewithargsret{\sphinxcode{\sphinxupquote{skinoptics.absorption\_coefficient.}}\sphinxbfcode{\sphinxupquote{mua\_fat\_vanVeen}}}{\sphinxparam{\DUrole{n}{lambda0}}}{}
\pysigstopsignatures
\begin{DUlineblock}{0em}
\item[] The absorption coefficient of (pig lard) FAT as a function of wavelength.
\item[] Linear interpolation of data from van Veen et al. 2000 {[}v*00{]} collected and processed
\item[] by S. Prahl and publicly available at \textless{}\sphinxurl{https://omlc.org/spectra/fat/}\textgreater{}.
\end{DUlineblock}

\begin{DUlineblock}{0em}
\item[] wavelength range: {[}429 nm, 1098 nm{]}
\end{DUlineblock}
\begin{quote}\begin{description}
\sphinxlineitem{Parameters}
\sphinxAtStartPar
\sphinxstyleliteralstrong{\sphinxupquote{lambda0}} (\sphinxstyleliteralemphasis{\sphinxupquote{float}}\sphinxstyleliteralemphasis{\sphinxupquote{ or }}\sphinxstyleliteralemphasis{\sphinxupquote{np.ndarray}}) \textendash{} wavelength {[}nm{]}

\sphinxlineitem{Returns}
\sphinxAtStartPar
\begin{itemize}
\item {} 
\sphinxAtStartPar
\sphinxstylestrong{mua} (\sphinxstyleemphasis{float or np.ndarray}) \textendash{} absorption coefficient {[}mm\sphinxhyphen{}1{]}

\end{itemize}


\end{description}\end{quote}

\end{fulllineitems}

\index{mua\_from\_ext\_and\_Cmass() (in module skinoptics.absorption\_coefficient)@\spxentry{mua\_from\_ext\_and\_Cmass()}\spxextra{in module skinoptics.absorption\_coefficient}}

\begin{fulllineitems}
\phantomsection\label{\detokenize{03_absorption_coefficient:skinoptics.absorption_coefficient.mua_from_ext_and_Cmass}}
\pysigstartsignatures
\pysiglinewithargsret{\sphinxcode{\sphinxupquote{skinoptics.absorption\_coefficient.}}\sphinxbfcode{\sphinxupquote{mua\_from\_ext\_and\_Cmass}}}{\sphinxparam{\DUrole{n}{ext}}\sphinxparamcomma \sphinxparam{\DUrole{n}{Cmass}}}{}
\pysigstopsignatures
\begin{DUlineblock}{0em}
\item[] Calculate the absorption coefficient from the extinction coefficient
\item[] and the mass concentration.
\item[] For details please check Jacques 2013 {[}J13{]}.
\end{DUlineblock}

\sphinxAtStartPar
\(\mu_a = \mbox{ln}(10) \varepsilon_{mass} C_{mass}\)
\begin{quote}\begin{description}
\sphinxlineitem{Parameters}\begin{itemize}
\item {} 
\sphinxAtStartPar
\sphinxstyleliteralstrong{\sphinxupquote{ext}} (\sphinxstyleliteralemphasis{\sphinxupquote{float}}\sphinxstyleliteralemphasis{\sphinxupquote{ or }}\sphinxstyleliteralemphasis{\sphinxupquote{np.ndarray}}) \textendash{} extinction coefficient {[}cm\sphinxhyphen{}1 mL mg\sphinxhyphen{}1{]}

\item {} 
\sphinxAtStartPar
\sphinxstyleliteralstrong{\sphinxupquote{Cmass}} (\sphinxstyleliteralemphasis{\sphinxupquote{float}}) \textendash{} mass concentration {[}g L\sphinxhyphen{}1{]}

\end{itemize}

\sphinxlineitem{Returns}
\sphinxAtStartPar
\begin{itemize}
\item {} 
\sphinxAtStartPar
\sphinxstylestrong{mua} (\sphinxstyleemphasis{float or np.ndarray}) \textendash{} absorption coefficient {[}mm\sphinxhyphen{}1{]}

\end{itemize}


\end{description}\end{quote}

\end{fulllineitems}

\index{mua\_from\_k() (in module skinoptics.absorption\_coefficient)@\spxentry{mua\_from\_k()}\spxextra{in module skinoptics.absorption\_coefficient}}

\begin{fulllineitems}
\phantomsection\label{\detokenize{03_absorption_coefficient:skinoptics.absorption_coefficient.mua_from_k}}
\pysigstartsignatures
\pysiglinewithargsret{\sphinxcode{\sphinxupquote{skinoptics.absorption\_coefficient.}}\sphinxbfcode{\sphinxupquote{mua\_from\_k}}}{\sphinxparam{\DUrole{n}{k}}\sphinxparamcomma \sphinxparam{\DUrole{n}{lambda0}}}{}
\pysigstopsignatures
\begin{DUlineblock}{0em}
\item[] Calculate the absorption coefficient from the imaginary part of the complex refractive index
\item[] and the wavelength.
\item[] For details please check Hecht 2002 {[}H02{]}, Jacques 2013 {[}J13{]} and Griffiths 2017 {[}G17{]}.
\end{DUlineblock}

\sphinxAtStartPar
\(\mu_a(\lambda) = \frac{4\pi k(\lambda)}{\lambda}\)
\begin{quote}\begin{description}
\sphinxlineitem{Parameters}\begin{itemize}
\item {} 
\sphinxAtStartPar
\sphinxstyleliteralstrong{\sphinxupquote{k}} (\sphinxstyleliteralemphasis{\sphinxupquote{float}}\sphinxstyleliteralemphasis{\sphinxupquote{ or }}\sphinxstyleliteralemphasis{\sphinxupquote{np.ndarray}}) \textendash{} imaginary part of the complex refractive index {[}\sphinxhyphen{}{]}

\item {} 
\sphinxAtStartPar
\sphinxstyleliteralstrong{\sphinxupquote{lambda0}} (\sphinxstyleliteralemphasis{\sphinxupquote{float}}\sphinxstyleliteralemphasis{\sphinxupquote{ or }}\sphinxstyleliteralemphasis{\sphinxupquote{np.ndarray}}) \textendash{} wavelength {[}nm{]}

\end{itemize}

\sphinxlineitem{Returns}
\sphinxAtStartPar
\begin{itemize}
\item {} 
\sphinxAtStartPar
\sphinxstylestrong{mua} (\sphinxstyleemphasis{float or np.ndarray}) \textendash{} absorption coefficient {[}mm\sphinxhyphen{}1{]}

\end{itemize}


\end{description}\end{quote}

\end{fulllineitems}

\index{mua\_from\_molarext\_and\_Cmolar() (in module skinoptics.absorption\_coefficient)@\spxentry{mua\_from\_molarext\_and\_Cmolar()}\spxextra{in module skinoptics.absorption\_coefficient}}

\begin{fulllineitems}
\phantomsection\label{\detokenize{03_absorption_coefficient:skinoptics.absorption_coefficient.mua_from_molarext_and_Cmolar}}
\pysigstartsignatures
\pysiglinewithargsret{\sphinxcode{\sphinxupquote{skinoptics.absorption\_coefficient.}}\sphinxbfcode{\sphinxupquote{mua\_from\_molarext\_and\_Cmolar}}}{\sphinxparam{\DUrole{n}{molarext}}\sphinxparamcomma \sphinxparam{\DUrole{n}{Cmolar}}}{}
\pysigstopsignatures
\begin{DUlineblock}{0em}
\item[] Calculate the absorption coefficient from the molar extinction coefficient
\item[] and the molar concentration.
\item[] For details please check Jacques 2013 {[}J13{]}.
\end{DUlineblock}

\sphinxAtStartPar
\(\mu_a = \mbox{ln}(10) \varepsilon_{molar} C_{molar}\)
\begin{quote}\begin{description}
\sphinxlineitem{Parameters}\begin{itemize}
\item {} 
\sphinxAtStartPar
\sphinxstyleliteralstrong{\sphinxupquote{molarext}} (\sphinxstyleliteralemphasis{\sphinxupquote{float}}\sphinxstyleliteralemphasis{\sphinxupquote{ or }}\sphinxstyleliteralemphasis{\sphinxupquote{np.ndarray}}) \textendash{} molar extinction coefficient {[}cm\sphinxhyphen{}1 M\sphinxhyphen{}1{]}

\item {} 
\sphinxAtStartPar
\sphinxstyleliteralstrong{\sphinxupquote{Cmolar}} (\sphinxstyleliteralemphasis{\sphinxupquote{float}}) \textendash{} molar concentration {[}M{]}

\end{itemize}

\sphinxlineitem{Returns}
\sphinxAtStartPar
\begin{itemize}
\item {} 
\sphinxAtStartPar
\sphinxstylestrong{mua} (\sphinxstyleemphasis{float or np.ndarray}) \textendash{} absorption coefficient {[}mm\sphinxhyphen{}1{]}

\end{itemize}


\end{description}\end{quote}

\end{fulllineitems}

\index{mua\_iBCC\_Salomatina() (in module skinoptics.absorption\_coefficient)@\spxentry{mua\_iBCC\_Salomatina()}\spxextra{in module skinoptics.absorption\_coefficient}}

\begin{fulllineitems}
\phantomsection\label{\detokenize{03_absorption_coefficient:skinoptics.absorption_coefficient.mua_iBCC_Salomatina}}
\pysigstartsignatures
\pysiglinewithargsret{\sphinxcode{\sphinxupquote{skinoptics.absorption\_coefficient.}}\sphinxbfcode{\sphinxupquote{mua\_iBCC\_Salomatina}}}{\sphinxparam{\DUrole{n}{lambda0}}}{}
\pysigstopsignatures
\begin{DUlineblock}{0em}
\item[] The absoption coefficient of INFILTRATIVE BASAL CELL CARCINOMA as a function of wavelength.
\item[] Linear interpolation of experimental data from Salomatina et al. 2006 {[}S*06{]},
\item[] publicly available at \textless{}\sphinxurl{https://sites.uml.edu/abl/optical-properties-2/}\textgreater{}.
\end{DUlineblock}

\begin{DUlineblock}{0em}
\item[] wavelength range: {[}370 nm, 1600 nm{]}
\end{DUlineblock}
\begin{quote}\begin{description}
\sphinxlineitem{Parameters}
\sphinxAtStartPar
\sphinxstyleliteralstrong{\sphinxupquote{lambda0}} (\sphinxstyleliteralemphasis{\sphinxupquote{float}}\sphinxstyleliteralemphasis{\sphinxupquote{ or }}\sphinxstyleliteralemphasis{\sphinxupquote{np.ndarray}}) \textendash{} wavelength {[}nm{]}

\sphinxlineitem{Returns}
\sphinxAtStartPar
\begin{itemize}
\item {} 
\sphinxAtStartPar
\sphinxstylestrong{mua} (\sphinxstyleemphasis{float or np.ndarray}) \textendash{} absorption coefficient {[}mm\sphinxhyphen{}1{]}

\end{itemize}


\end{description}\end{quote}

\end{fulllineitems}

\index{mua\_nBCC\_Salomatina() (in module skinoptics.absorption\_coefficient)@\spxentry{mua\_nBCC\_Salomatina()}\spxextra{in module skinoptics.absorption\_coefficient}}

\begin{fulllineitems}
\phantomsection\label{\detokenize{03_absorption_coefficient:skinoptics.absorption_coefficient.mua_nBCC_Salomatina}}
\pysigstartsignatures
\pysiglinewithargsret{\sphinxcode{\sphinxupquote{skinoptics.absorption\_coefficient.}}\sphinxbfcode{\sphinxupquote{mua\_nBCC\_Salomatina}}}{\sphinxparam{\DUrole{n}{lambda0}}}{}
\pysigstopsignatures
\begin{DUlineblock}{0em}
\item[] The absoption coefficient of NODULAR BASAL CELL CARCINOMA as a function of wavelength.
\item[] Linear interpolation of experimental data from Salomatina et al. 2006 {[}S*06{]},
\item[] publicly available at \textless{}\sphinxurl{https://sites.uml.edu/abl/optical-properties-2/}\textgreater{}.
\end{DUlineblock}

\begin{DUlineblock}{0em}
\item[] wavelength range: {[}370 nm, 1600 nm{]}
\end{DUlineblock}
\begin{quote}\begin{description}
\sphinxlineitem{Parameters}
\sphinxAtStartPar
\sphinxstyleliteralstrong{\sphinxupquote{lambda0}} (\sphinxstyleliteralemphasis{\sphinxupquote{float}}\sphinxstyleliteralemphasis{\sphinxupquote{ or }}\sphinxstyleliteralemphasis{\sphinxupquote{np.ndarray}}) \textendash{} wavelength {[}nm{]}

\sphinxlineitem{Returns}
\sphinxAtStartPar
\begin{itemize}
\item {} 
\sphinxAtStartPar
\sphinxstylestrong{mua} (\sphinxstyleemphasis{float or np.ndarray}) \textendash{} absorption coefficient {[}mm\sphinxhyphen{}1{]}

\end{itemize}


\end{description}\end{quote}

\end{fulllineitems}

\index{mua\_oxy\_Bosschaart() (in module skinoptics.absorption\_coefficient)@\spxentry{mua\_oxy\_Bosschaart()}\spxextra{in module skinoptics.absorption\_coefficient}}

\begin{fulllineitems}
\phantomsection\label{\detokenize{03_absorption_coefficient:skinoptics.absorption_coefficient.mua_oxy_Bosschaart}}
\pysigstartsignatures
\pysiglinewithargsret{\sphinxcode{\sphinxupquote{skinoptics.absorption\_coefficient.}}\sphinxbfcode{\sphinxupquote{mua\_oxy\_Bosschaart}}}{\sphinxparam{\DUrole{n}{lambda0}}}{}
\pysigstopsignatures
\begin{DUlineblock}{0em}
\item[] The absorption coefficient of OXYGENIZED BLOOD (saturation \textgreater{} 98\%) as a function of wavelength.
\item[] Linear interpolation of data from Bosschaart et. al. 2014 {[}B*14{]}.
\end{DUlineblock}

\begin{DUlineblock}{0em}
\item[] wavelength range: {[}251 nm, 1995 nm{]}
\end{DUlineblock}
\begin{quote}\begin{description}
\sphinxlineitem{Parameters}
\sphinxAtStartPar
\sphinxstyleliteralstrong{\sphinxupquote{lambda0}} (\sphinxstyleliteralemphasis{\sphinxupquote{float}}\sphinxstyleliteralemphasis{\sphinxupquote{ or }}\sphinxstyleliteralemphasis{\sphinxupquote{np.ndarray}}) \textendash{} wavelength {[}nm{]}

\sphinxlineitem{Returns}
\sphinxAtStartPar
\begin{itemize}
\item {} 
\sphinxAtStartPar
\sphinxstylestrong{mua} (\sphinxstyleemphasis{float or np.ndarray}) \textendash{} absorption coefficient {[}mm\sphinxhyphen{}1{]}

\end{itemize}


\end{description}\end{quote}

\end{fulllineitems}

\index{mua\_oxy\_Prahl() (in module skinoptics.absorption\_coefficient)@\spxentry{mua\_oxy\_Prahl()}\spxextra{in module skinoptics.absorption\_coefficient}}

\begin{fulllineitems}
\phantomsection\label{\detokenize{03_absorption_coefficient:skinoptics.absorption_coefficient.mua_oxy_Prahl}}
\pysigstartsignatures
\pysiglinewithargsret{\sphinxcode{\sphinxupquote{skinoptics.absorption\_coefficient.}}\sphinxbfcode{\sphinxupquote{mua\_oxy\_Prahl}}}{\sphinxparam{\DUrole{n}{lambda0}}\sphinxparamcomma \sphinxparam{\DUrole{n}{Cmass\_oxy}\DUrole{o}{=}\DUrole{default_value}{150}}\sphinxparamcomma \sphinxparam{\DUrole{n}{molar\_mass\_oxy}\DUrole{o}{=}\DUrole{default_value}{64500}}}{}
\pysigstopsignatures
\begin{DUlineblock}{0em}
\item[] The absorption coefficient of OXY\sphinxhyphen{}HEMOGLOBIN as a function of wavelength.
\item[] Calculated from molarext\_oxy\_Prahl for a specific oxy\sphinxhyphen{}hemoglobin mass concentration.
\end{DUlineblock}

\begin{DUlineblock}{0em}
\item[] wavelength range: {[}250 nm, 1000 nm{]}
\end{DUlineblock}
\begin{quote}\begin{description}
\sphinxlineitem{Parameters}\begin{itemize}
\item {} 
\sphinxAtStartPar
\sphinxstyleliteralstrong{\sphinxupquote{lambda0}} (\sphinxstyleliteralemphasis{\sphinxupquote{float}}\sphinxstyleliteralemphasis{\sphinxupquote{ or }}\sphinxstyleliteralemphasis{\sphinxupquote{np.ndarray}}) \textendash{} wavelength {[}nm{]}

\item {} 
\sphinxAtStartPar
\sphinxstyleliteralstrong{\sphinxupquote{Cmass\_oxy}} (\sphinxstyleliteralemphasis{\sphinxupquote{float}}) \textendash{} oxy\sphinxhyphen{}hemoglobin mass concentration {[}g/L{]} (default to 150. {[}S*23{]})

\item {} 
\sphinxAtStartPar
\sphinxstyleliteralstrong{\sphinxupquote{molar\_mass\_oxy}} (\sphinxstyleliteralemphasis{\sphinxupquote{float}}) \textendash{} molar mass of oxy\sphinxhyphen{}hemoglobin {[}g/mol{]} (default  to 64500. {[}B90{]})

\end{itemize}

\sphinxlineitem{Returns}
\sphinxAtStartPar
\begin{itemize}
\item {} 
\sphinxAtStartPar
\sphinxstylestrong{mua} (\sphinxstyleemphasis{float or np.ndarray}) \textendash{} absorption coefficient {[}mm\sphinxhyphen{}1{]}

\end{itemize}


\end{description}\end{quote}

\end{fulllineitems}

\index{mua\_phe\_Donner() (in module skinoptics.absorption\_coefficient)@\spxentry{mua\_phe\_Donner()}\spxextra{in module skinoptics.absorption\_coefficient}}

\begin{fulllineitems}
\phantomsection\label{\detokenize{03_absorption_coefficient:skinoptics.absorption_coefficient.mua_phe_Donner}}
\pysigstartsignatures
\pysiglinewithargsret{\sphinxcode{\sphinxupquote{skinoptics.absorption\_coefficient.}}\sphinxbfcode{\sphinxupquote{mua\_phe\_Donner}}}{\sphinxparam{\DUrole{n}{lambda0}}}{}
\pysigstopsignatures
\begin{DUlineblock}{0em}
\item[] The absoption coefficient of PHEOMELANIN as a function of wavelength.
\item[] Equation proposed by Donner \& Jensen 2006 {[}DJ06{]} based on data from Sarna \& Swartz {[}SS06{]}.
\end{DUlineblock}

\sphinxAtStartPar
\(\mu_a^{phe} (\lambda) = 2.9 \times 10^{14} \times \lambda^{-4.75}\)
\begin{quote}\begin{description}
\sphinxlineitem{Parameters}
\sphinxAtStartPar
\sphinxstyleliteralstrong{\sphinxupquote{lambda0}} (\sphinxstyleliteralemphasis{\sphinxupquote{float}}\sphinxstyleliteralemphasis{\sphinxupquote{ or }}\sphinxstyleliteralemphasis{\sphinxupquote{np.ndarray}}) \textendash{} wavelength {[}nm{]}

\sphinxlineitem{Return}\begin{itemize}
\item {} 
\sphinxAtStartPar
\sphinxstylestrong{mua} (\sphinxstyleemphasis{float or np.ndarray}) \textendash{} absorption coefficient {[}mm\sphinxhyphen{}1{]}

\end{itemize}

\end{description}\end{quote}

\end{fulllineitems}

\index{mua\_wat\_Hale() (in module skinoptics.absorption\_coefficient)@\spxentry{mua\_wat\_Hale()}\spxextra{in module skinoptics.absorption\_coefficient}}

\begin{fulllineitems}
\phantomsection\label{\detokenize{03_absorption_coefficient:skinoptics.absorption_coefficient.mua_wat_Hale}}
\pysigstartsignatures
\pysiglinewithargsret{\sphinxcode{\sphinxupquote{skinoptics.absorption\_coefficient.}}\sphinxbfcode{\sphinxupquote{mua\_wat\_Hale}}}{\sphinxparam{\DUrole{n}{lambda0}}}{}
\pysigstopsignatures
\begin{DUlineblock}{0em}
\item[] The absorption coefficient of WATER as a function of wavelength.
\item[] Linear interpolation of data from Hale \& Querry 1973 {[}HQ73{]} collected and processed 
\item[] by S. Jacques and S. Prahl and publicly available at
\item[] \textless{}\sphinxurl{https://omlc.org/spectra/water/abs/index.html}\textgreater{}.
\end{DUlineblock}

\sphinxAtStartPar
wavelength range: {[}200 nm, 200 \(\mu\) m{]}
\begin{quote}\begin{description}
\sphinxlineitem{Parameters}
\sphinxAtStartPar
\sphinxstyleliteralstrong{\sphinxupquote{lambda0}} (\sphinxstyleliteralemphasis{\sphinxupquote{float}}\sphinxstyleliteralemphasis{\sphinxupquote{ or }}\sphinxstyleliteralemphasis{\sphinxupquote{np.ndarray}}) \textendash{} wavelength {[}nm{]}

\sphinxlineitem{Returns}
\sphinxAtStartPar
\begin{itemize}
\item {} 
\sphinxAtStartPar
\sphinxstylestrong{mua} (\sphinxstyleemphasis{float or np.ndarray}) \textendash{} absorption coefficient {[}mm\sphinxhyphen{}1{]}

\end{itemize}


\end{description}\end{quote}

\end{fulllineitems}

\index{mua\_wat\_Segelstein() (in module skinoptics.absorption\_coefficient)@\spxentry{mua\_wat\_Segelstein()}\spxextra{in module skinoptics.absorption\_coefficient}}

\begin{fulllineitems}
\phantomsection\label{\detokenize{03_absorption_coefficient:skinoptics.absorption_coefficient.mua_wat_Segelstein}}
\pysigstartsignatures
\pysiglinewithargsret{\sphinxcode{\sphinxupquote{skinoptics.absorption\_coefficient.}}\sphinxbfcode{\sphinxupquote{mua\_wat\_Segelstein}}}{\sphinxparam{\DUrole{n}{lambda0}}}{}
\pysigstopsignatures
\begin{DUlineblock}{0em}
\item[] The absorption coefficient of WATER as a function of wavelength.
\item[] Linear interpolation of data from D. J. Segelstein’s M.S. Thesis 1981 {[}S81{]},
\item[] collected by S. Jacques and S. Prahl and publicly available at
\item[] \textless{}\sphinxurl{https://omlc.org/spectra/water/abs/index.html}\textgreater{}.
\end{DUlineblock}

\begin{DUlineblock}{0em}
\item[] wavelength range: {[}10 nm, 10 m{]}.
\end{DUlineblock}
\begin{quote}\begin{description}
\sphinxlineitem{Parameters}
\sphinxAtStartPar
\sphinxstyleliteralstrong{\sphinxupquote{lambda0}} (\sphinxstyleliteralemphasis{\sphinxupquote{float}}\sphinxstyleliteralemphasis{\sphinxupquote{ or }}\sphinxstyleliteralemphasis{\sphinxupquote{np.ndarray}}) \textendash{} wavelength {[}nm{]}

\sphinxlineitem{Returns}
\sphinxAtStartPar
\begin{itemize}
\item {} 
\sphinxAtStartPar
\sphinxstylestrong{mua} (\sphinxstyleemphasis{float or np.ndarray}) \textendash{} absorption coefficient {[}mm\sphinxhyphen{}1{]}

\end{itemize}


\end{description}\end{quote}

\end{fulllineitems}

\index{std\_mua\_DE\_Salomatina() (in module skinoptics.absorption\_coefficient)@\spxentry{std\_mua\_DE\_Salomatina()}\spxextra{in module skinoptics.absorption\_coefficient}}

\begin{fulllineitems}
\phantomsection\label{\detokenize{03_absorption_coefficient:skinoptics.absorption_coefficient.std_mua_DE_Salomatina}}
\pysigstartsignatures
\pysiglinewithargsret{\sphinxcode{\sphinxupquote{skinoptics.absorption\_coefficient.}}\sphinxbfcode{\sphinxupquote{std\_mua\_DE\_Salomatina}}}{\sphinxparam{\DUrole{n}{lambda0}}}{}
\pysigstopsignatures
\begin{DUlineblock}{0em}
\item[] The standard deviation respective to mua\_DE\_Salomatina.
\item[] Linear interpolation of experimental data from Salomatina et al. 2006 {[}S*06{]},
\item[] publicly available at \textless{}\sphinxurl{https://sites.uml.edu/abl/optical-properties-2/}\textgreater{}.
\end{DUlineblock}

\begin{DUlineblock}{0em}
\item[] wavelength range: {[}370 nm, 1600 nm{]}
\end{DUlineblock}
\begin{quote}\begin{description}
\sphinxlineitem{Parameters}
\sphinxAtStartPar
\sphinxstyleliteralstrong{\sphinxupquote{lambda0}} (\sphinxstyleliteralemphasis{\sphinxupquote{float}}\sphinxstyleliteralemphasis{\sphinxupquote{ or }}\sphinxstyleliteralemphasis{\sphinxupquote{np.ndarray}}) \textendash{} wavelength {[}nm{]}

\sphinxlineitem{Returns}
\sphinxAtStartPar
\begin{itemize}
\item {} 
\sphinxAtStartPar
\sphinxstylestrong{std\_mua} (\sphinxstyleemphasis{float or np.ndarray}) \textendash{} standard deviation of the absorption coefficient {[}mm\sphinxhyphen{}1{]}

\end{itemize}


\end{description}\end{quote}

\end{fulllineitems}

\index{std\_mua\_EP\_Salomatina() (in module skinoptics.absorption\_coefficient)@\spxentry{std\_mua\_EP\_Salomatina()}\spxextra{in module skinoptics.absorption\_coefficient}}

\begin{fulllineitems}
\phantomsection\label{\detokenize{03_absorption_coefficient:skinoptics.absorption_coefficient.std_mua_EP_Salomatina}}
\pysigstartsignatures
\pysiglinewithargsret{\sphinxcode{\sphinxupquote{skinoptics.absorption\_coefficient.}}\sphinxbfcode{\sphinxupquote{std\_mua\_EP\_Salomatina}}}{\sphinxparam{\DUrole{n}{lambda0}}}{}
\pysigstopsignatures
\begin{DUlineblock}{0em}
\item[] The standard deviation respective to mua\_EP\_Salomatina.
\item[] Linear interpolation of experimental data from Salomatina et al. 2006 {[}S*06{]},
\item[] publicly available at \textless{}\sphinxurl{https://sites.uml.edu/abl/optical-properties-2/}\textgreater{}.
\end{DUlineblock}

\begin{DUlineblock}{0em}
\item[] wavelength range: {[}370 nm, 1600 nm{]}
\end{DUlineblock}
\begin{quote}\begin{description}
\sphinxlineitem{Parameters}
\sphinxAtStartPar
\sphinxstyleliteralstrong{\sphinxupquote{lambda0}} (\sphinxstyleliteralemphasis{\sphinxupquote{float}}\sphinxstyleliteralemphasis{\sphinxupquote{ or }}\sphinxstyleliteralemphasis{\sphinxupquote{np.ndarray}}) \textendash{} wavelength {[}nm{]}

\sphinxlineitem{Returns}
\sphinxAtStartPar
\begin{itemize}
\item {} 
\sphinxAtStartPar
\sphinxstylestrong{std\_mua} (\sphinxstyleemphasis{float or np.ndarray}) \textendash{} standard deviation of the absorption coefficient {[}mm\sphinxhyphen{}1{]}

\end{itemize}


\end{description}\end{quote}

\end{fulllineitems}

\index{std\_mua\_HY\_Salomatina() (in module skinoptics.absorption\_coefficient)@\spxentry{std\_mua\_HY\_Salomatina()}\spxextra{in module skinoptics.absorption\_coefficient}}

\begin{fulllineitems}
\phantomsection\label{\detokenize{03_absorption_coefficient:skinoptics.absorption_coefficient.std_mua_HY_Salomatina}}
\pysigstartsignatures
\pysiglinewithargsret{\sphinxcode{\sphinxupquote{skinoptics.absorption\_coefficient.}}\sphinxbfcode{\sphinxupquote{std\_mua\_HY\_Salomatina}}}{\sphinxparam{\DUrole{n}{lambda0}}}{}
\pysigstopsignatures
\begin{DUlineblock}{0em}
\item[] The standard deviation respective to mua\_HY\_Salomatina.
\item[] Linear interpolation of experimental data from Salomatina et al. 2006 {[}S*06{]},
\item[] publicly available at \textless{}\sphinxurl{https://sites.uml.edu/abl/optical-properties-2/}\textgreater{}.
\end{DUlineblock}

\begin{DUlineblock}{0em}
\item[] wavelength range: {[}370 nm, 1600 nm{]}
\end{DUlineblock}
\begin{quote}\begin{description}
\sphinxlineitem{Parameters}
\sphinxAtStartPar
\sphinxstyleliteralstrong{\sphinxupquote{lambda0}} (\sphinxstyleliteralemphasis{\sphinxupquote{float}}\sphinxstyleliteralemphasis{\sphinxupquote{ or }}\sphinxstyleliteralemphasis{\sphinxupquote{np.ndarray}}) \textendash{} wavelength {[}nm{]}

\sphinxlineitem{Returns}
\sphinxAtStartPar
\begin{itemize}
\item {} 
\sphinxAtStartPar
\sphinxstylestrong{std\_mua} (\sphinxstyleemphasis{float or np.ndarray}) \textendash{} standard deviation of the absorption coefficient {[}mm\sphinxhyphen{}1{]}

\end{itemize}


\end{description}\end{quote}

\end{fulllineitems}


\sphinxstepscope


\subsection{skinoptics.scattering\_coefficient module}
\label{\detokenize{04_scattering_coefficient:module-skinoptics.scattering_coefficient}}\label{\detokenize{04_scattering_coefficient:skinoptics-scattering-coefficient-module}}\label{\detokenize{04_scattering_coefficient::doc}}\index{module@\spxentry{module}!skinoptics.scattering\_coefficient@\spxentry{skinoptics.scattering\_coefficient}}\index{skinoptics.scattering\_coefficient@\spxentry{skinoptics.scattering\_coefficient}!module@\spxentry{module}}
\sphinxAtStartPar
Copyright (C) 2024 Victor Lima
\begin{quote}

\begin{DUlineblock}{0em}
\item[] This program is free software: you can redistribute it and/or modify
\item[] it under the terms of the GNU General Public License as published by
\item[] the Free Software Foundation, either version 3 of the License, or
\item[] (at your option) any later version.
\end{DUlineblock}

\begin{DUlineblock}{0em}
\item[] This program is distributed in the hope that it will be useful,
\item[] but WITHOUT ANY WARRANTY; without even the implied warranty of
\item[] MERCHANTABILITY or FITNESS FOR A PARTICULAR PURPOSE.  See the
\item[] GNU General Public License for more details.
\end{DUlineblock}

\begin{DUlineblock}{0em}
\item[] You should have received a copy of the GNU General Public License
\item[] along with this program.  If not, see \textless{}\sphinxurl{https://www.gnu.org/licenses/}\textgreater{}.
\end{DUlineblock}
\end{quote}

\begin{DUlineblock}{0em}
\item[] Victor Lima
\item[] victorporto@ifsc.usp.br
\item[] victor.lima@ufscar.br
\end{DUlineblock}

\begin{DUlineblock}{0em}
\item[] Release Date:
\item[] August 2024
\item[] Last Modification:
\item[] August 2024
\end{DUlineblock}

\begin{DUlineblock}{0em}
\item[] References:
\end{DUlineblock}

\begin{DUlineblock}{0em}
\item[] {[}SJT95{]} Saidi, Jacques \& Tittel 1995.
\item[] Mie and Rayleigh modeling of visible\sphinxhyphen{}light scattering in neonatal skin.
\item[] \sphinxurl{https://doi.org/10.1364/AO.34.007410}
\end{DUlineblock}

\begin{DUlineblock}{0em}
\item[] {[}S*06{]} Salomatina, Jiang, Novak \& Yaroslavsky 2006.
\item[] Optical properties of normal and cancerous human skin in the visible and near\sphinxhyphen{}infrared spectral range.
\item[] \sphinxurl{https://doi.org/10.1117/1.2398928}
\end{DUlineblock}

\begin{DUlineblock}{0em}
\item[] {[}J13{]} Jacques 2013.
\item[] Optical properties of biological tissues: a review.
\item[] \sphinxurl{https://doi.org/10.1088/0031-9155/58/14/5007}
\end{DUlineblock}

\begin{DUlineblock}{0em}
\item[] {[}N19{]} Niemz 2019.
\item[] Laser\sphinxhyphen{}Tissue Interactions: Fundamentals and Applications (4th edition).
\item[] \sphinxurl{https://doi.org/10.1007/978-3-030-11917-1}
\end{DUlineblock}
\index{albedo() (in module skinoptics.scattering\_coefficient)@\spxentry{albedo()}\spxextra{in module skinoptics.scattering\_coefficient}}

\begin{fulllineitems}
\phantomsection\label{\detokenize{04_scattering_coefficient:skinoptics.scattering_coefficient.albedo}}
\pysigstartsignatures
\pysiglinewithargsret{\sphinxcode{\sphinxupquote{skinoptics.scattering\_coefficient.}}\sphinxbfcode{\sphinxupquote{albedo}}}{\sphinxparam{\DUrole{n}{mua}}\sphinxparamcomma \sphinxparam{\DUrole{n}{mus}}}{}
\pysigstopsignatures
\begin{DUlineblock}{0em}
\item[] Calculate the optical albedo from the absorption coefficient and the scattering coefficient.
\item[] For details please check section 2.4 from Niemz 2019 {[}N19{]}.
\end{DUlineblock}

\sphinxAtStartPar
\(a = \frac{\mu_s}{\mu_a + \mu_s}\)
\begin{quote}\begin{description}
\sphinxlineitem{Parameters}\begin{itemize}
\item {} 
\sphinxAtStartPar
\sphinxstyleliteralstrong{\sphinxupquote{mua}} (\sphinxstyleliteralemphasis{\sphinxupquote{float}}\sphinxstyleliteralemphasis{\sphinxupquote{ or }}\sphinxstyleliteralemphasis{\sphinxupquote{np.ndarray}}) \textendash{} absorption coefficient {[}mm\sphinxhyphen{}1{]}

\item {} 
\sphinxAtStartPar
\sphinxstyleliteralstrong{\sphinxupquote{mus}} (\sphinxstyleliteralemphasis{\sphinxupquote{float}}\sphinxstyleliteralemphasis{\sphinxupquote{ or }}\sphinxstyleliteralemphasis{\sphinxupquote{np.ndarray}}) \textendash{} scattering coefficient {[}mm\sphinxhyphen{}1{]}

\end{itemize}

\sphinxlineitem{Returns}
\sphinxAtStartPar
\begin{itemize}
\item {} 
\sphinxAtStartPar
\sphinxstylestrong{albedo} (\sphinxstyleemphasis{float or np.ndarray}) \textendash{} optical albedo {[}\sphinxhyphen{}{]}

\end{itemize}


\end{description}\end{quote}

\end{fulllineitems}

\index{mus\_from\_rmus() (in module skinoptics.scattering\_coefficient)@\spxentry{mus\_from\_rmus()}\spxextra{in module skinoptics.scattering\_coefficient}}

\begin{fulllineitems}
\phantomsection\label{\detokenize{04_scattering_coefficient:skinoptics.scattering_coefficient.mus_from_rmus}}
\pysigstartsignatures
\pysiglinewithargsret{\sphinxcode{\sphinxupquote{skinoptics.scattering\_coefficient.}}\sphinxbfcode{\sphinxupquote{mus\_from\_rmus}}}{\sphinxparam{\DUrole{n}{rmus}}\sphinxparamcomma \sphinxparam{\DUrole{n}{g}}}{}
\pysigstopsignatures
\begin{DUlineblock}{0em}
\item[] Calculate the scattering coefficient from the reduced scattering coefficient and the 
\item[] anisotropy factor.
\end{DUlineblock}

\sphinxAtStartPar
\(\mu_s = \frac{\mu_s'}{1-g}\)
\begin{quote}\begin{description}
\sphinxlineitem{Parameters}\begin{itemize}
\item {} 
\sphinxAtStartPar
\sphinxstyleliteralstrong{\sphinxupquote{rmus}} (\sphinxstyleliteralemphasis{\sphinxupquote{float}}\sphinxstyleliteralemphasis{\sphinxupquote{ or }}\sphinxstyleliteralemphasis{\sphinxupquote{np.ndarray}}) \textendash{} reduced scattering coefficient {[}mm\sphinxhyphen{}1{]}

\item {} 
\sphinxAtStartPar
\sphinxstyleliteralstrong{\sphinxupquote{g}} (\sphinxstyleliteralemphasis{\sphinxupquote{float}}) \textendash{} anisotropy factor {[}\sphinxhyphen{}{]} (must be in the range {[}\sphinxhyphen{}1, 1{]})

\end{itemize}

\sphinxlineitem{Returns}
\sphinxAtStartPar
\begin{itemize}
\item {} 
\sphinxAtStartPar
\sphinxstylestrong{mus} (\sphinxstyleemphasis{float or np.ndarray}) \textendash{} scattering coefficient {[}mm\sphinxhyphen{}1{]}

\end{itemize}


\end{description}\end{quote}

\end{fulllineitems}

\index{rmus\_DE\_Salomatina() (in module skinoptics.scattering\_coefficient)@\spxentry{rmus\_DE\_Salomatina()}\spxextra{in module skinoptics.scattering\_coefficient}}

\begin{fulllineitems}
\phantomsection\label{\detokenize{04_scattering_coefficient:skinoptics.scattering_coefficient.rmus_DE_Salomatina}}
\pysigstartsignatures
\pysiglinewithargsret{\sphinxcode{\sphinxupquote{skinoptics.scattering\_coefficient.}}\sphinxbfcode{\sphinxupquote{rmus\_DE\_Salomatina}}}{\sphinxparam{\DUrole{n}{lambda0}}}{}
\pysigstopsignatures
\begin{DUlineblock}{0em}
\item[] The reduced scattering coefficient of human DERMIS as a function of wavelength.
\item[] Linear interpolation of experimental data from Salomatina et al. 2006 {[}S*06{]},
\item[] publicly available at \textless{}\sphinxurl{https://sites.uml.edu/abl/optical-properties-2/}\textgreater{}.
\end{DUlineblock}

\begin{DUlineblock}{0em}
\item[] wavelength range: {[}370 nm, 1600 nm{]}
\end{DUlineblock}
\begin{quote}\begin{description}
\sphinxlineitem{Parameters}
\sphinxAtStartPar
\sphinxstyleliteralstrong{\sphinxupquote{lambda0}} (\sphinxstyleliteralemphasis{\sphinxupquote{float}}\sphinxstyleliteralemphasis{\sphinxupquote{ or }}\sphinxstyleliteralemphasis{\sphinxupquote{np.ndarray}}) \textendash{} wavelength {[}nm{]}

\sphinxlineitem{Returns}
\sphinxAtStartPar
\begin{itemize}
\item {} 
\sphinxAtStartPar
\sphinxstylestrong{rmus} (\sphinxstyleemphasis{float or np.ndarray}) \textendash{} reduced scattering coefficient {[}mm\sphinxhyphen{}1{]}

\end{itemize}


\end{description}\end{quote}

\end{fulllineitems}

\index{rmus\_EP\_Salomatina() (in module skinoptics.scattering\_coefficient)@\spxentry{rmus\_EP\_Salomatina()}\spxextra{in module skinoptics.scattering\_coefficient}}

\begin{fulllineitems}
\phantomsection\label{\detokenize{04_scattering_coefficient:skinoptics.scattering_coefficient.rmus_EP_Salomatina}}
\pysigstartsignatures
\pysiglinewithargsret{\sphinxcode{\sphinxupquote{skinoptics.scattering\_coefficient.}}\sphinxbfcode{\sphinxupquote{rmus\_EP\_Salomatina}}}{\sphinxparam{\DUrole{n}{lambda0}}}{}
\pysigstopsignatures
\begin{DUlineblock}{0em}
\item[] The reduced scattering coefficient of human EPIDERMIS as a function of wavelength.
\item[] Linear interpolation of experimental data from Salomatina et al. 2006 {[}S*06{]},
\item[] publicly available at \textless{}\sphinxurl{https://sites.uml.edu/abl/optical-properties-2}\textgreater{}.
\end{DUlineblock}

\begin{DUlineblock}{0em}
\item[] wavelength range: {[}370 nm, 1600 nm{]}
\end{DUlineblock}
\begin{quote}\begin{description}
\sphinxlineitem{Parameters}
\sphinxAtStartPar
\sphinxstyleliteralstrong{\sphinxupquote{lambda0}} (\sphinxstyleliteralemphasis{\sphinxupquote{float}}\sphinxstyleliteralemphasis{\sphinxupquote{ or }}\sphinxstyleliteralemphasis{\sphinxupquote{np.ndarray}}) \textendash{} wavelength {[}nm{]}

\sphinxlineitem{Returns}
\sphinxAtStartPar
\begin{itemize}
\item {} 
\sphinxAtStartPar
\sphinxstylestrong{rmus} (\sphinxstyleemphasis{float or np.ndarray}) \textendash{} reduced scattering coefficient {[}mm\sphinxhyphen{}1{]}

\end{itemize}


\end{description}\end{quote}

\end{fulllineitems}

\index{rmus\_HY\_Salomatina() (in module skinoptics.scattering\_coefficient)@\spxentry{rmus\_HY\_Salomatina()}\spxextra{in module skinoptics.scattering\_coefficient}}

\begin{fulllineitems}
\phantomsection\label{\detokenize{04_scattering_coefficient:skinoptics.scattering_coefficient.rmus_HY_Salomatina}}
\pysigstartsignatures
\pysiglinewithargsret{\sphinxcode{\sphinxupquote{skinoptics.scattering\_coefficient.}}\sphinxbfcode{\sphinxupquote{rmus\_HY\_Salomatina}}}{\sphinxparam{\DUrole{n}{lambda0}}}{}
\pysigstopsignatures
\begin{DUlineblock}{0em}
\item[] The reduced scattering coefficient of human HYPODERMIS as a function of wavelength.
\item[] Linear interpolation of experimental data from Salomatina et al. 2006 {[}S*06{]},
\item[] publicly available at \textless{}\sphinxurl{https://sites.uml.edu/abl/optical-properties-2/}\textgreater{}.
\end{DUlineblock}

\begin{DUlineblock}{0em}
\item[] wavelength range: {[}370 nm, 1600 nm{]}
\end{DUlineblock}
\begin{quote}\begin{description}
\sphinxlineitem{Parameters}
\sphinxAtStartPar
\sphinxstyleliteralstrong{\sphinxupquote{lambda0}} (\sphinxstyleliteralemphasis{\sphinxupquote{float}}\sphinxstyleliteralemphasis{\sphinxupquote{ or }}\sphinxstyleliteralemphasis{\sphinxupquote{np.ndarray}}) \textendash{} wavelength {[}nm{]}

\sphinxlineitem{Returns}
\sphinxAtStartPar
\begin{itemize}
\item {} 
\sphinxAtStartPar
\sphinxstylestrong{rmus} (\sphinxstyleemphasis{float or np.ndarray}) \textendash{} reduced scattering coefficient {[}mm\sphinxhyphen{}1{]}

\end{itemize}


\end{description}\end{quote}

\end{fulllineitems}

\index{rmus\_Jacques() (in module skinoptics.scattering\_coefficient)@\spxentry{rmus\_Jacques()}\spxextra{in module skinoptics.scattering\_coefficient}}

\begin{fulllineitems}
\phantomsection\label{\detokenize{04_scattering_coefficient:skinoptics.scattering_coefficient.rmus_Jacques}}
\pysigstartsignatures
\pysiglinewithargsret{\sphinxcode{\sphinxupquote{skinoptics.scattering\_coefficient.}}\sphinxbfcode{\sphinxupquote{rmus\_Jacques}}}{\sphinxparam{\DUrole{n}{lambda0}}\sphinxparamcomma \sphinxparam{\DUrole{n}{a}}\sphinxparamcomma \sphinxparam{\DUrole{n}{f\_Ray}}\sphinxparamcomma \sphinxparam{\DUrole{n}{b\_Mie}}}{}
\pysigstopsignatures
\begin{DUlineblock}{0em}
\item[] The reduced scattering coefficient as a function of wavelength, assuming contributions from
\item[] both Rayleigh and Mie scattering.
\item[] For details please check Jacques 2013 {[}J13{]}.
\end{DUlineblock}

\sphinxAtStartPar
\(\mu_s'(\lambda) = a\left[f_{Ray}\left(\frac{\lambda}{500}\right)^{-4} + (1-f_{Ray})\left(\frac{\lambda}{500}\right)^{-b_{Mie}} \right]\)
\begin{quote}\begin{description}
\sphinxlineitem{Parameters}\begin{itemize}
\item {} 
\sphinxAtStartPar
\sphinxstyleliteralstrong{\sphinxupquote{lambda0}} (\sphinxstyleliteralemphasis{\sphinxupquote{float}}\sphinxstyleliteralemphasis{\sphinxupquote{ or }}\sphinxstyleliteralemphasis{\sphinxupquote{np.ndarray}}) \textendash{} wavelength {[}nm{]}

\item {} 
\sphinxAtStartPar
\sphinxstyleliteralstrong{\sphinxupquote{a}} (\sphinxstyleliteralemphasis{\sphinxupquote{float}}) \textendash{} parameter a {[}mm\sphinxhyphen{}1{]}

\item {} 
\sphinxAtStartPar
\sphinxstyleliteralstrong{\sphinxupquote{f\_Ray}} (\sphinxstyleliteralemphasis{\sphinxupquote{float}}) \textendash{} fraction of Rayleigh scattering contribution {[}\sphinxhyphen{}{]} (must be in the range {[}0, 1{]})

\item {} 
\sphinxAtStartPar
\sphinxstyleliteralstrong{\sphinxupquote{b\_Mie}} (\sphinxstyleliteralemphasis{\sphinxupquote{float}}) \textendash{} Mie scattering power {[}\sphinxhyphen{}{]} (must be nonnegative)

\end{itemize}

\sphinxlineitem{Returns}
\sphinxAtStartPar
\begin{itemize}
\item {} 
\sphinxAtStartPar
\sphinxstylestrong{rmus} (\sphinxstyleemphasis{float or np.ndarray}) \textendash{} reduced scattering coefficient {[}mm\sphinxhyphen{}1{]}

\end{itemize}


\end{description}\end{quote}

\end{fulllineitems}

\index{rmus\_Mie() (in module skinoptics.scattering\_coefficient)@\spxentry{rmus\_Mie()}\spxextra{in module skinoptics.scattering\_coefficient}}

\begin{fulllineitems}
\phantomsection\label{\detokenize{04_scattering_coefficient:skinoptics.scattering_coefficient.rmus_Mie}}
\pysigstartsignatures
\pysiglinewithargsret{\sphinxcode{\sphinxupquote{skinoptics.scattering\_coefficient.}}\sphinxbfcode{\sphinxupquote{rmus\_Mie}}}{\sphinxparam{\DUrole{n}{lambda0}}\sphinxparamcomma \sphinxparam{\DUrole{n}{B}}\sphinxparamcomma \sphinxparam{\DUrole{n}{b}}}{}
\pysigstopsignatures
\begin{DUlineblock}{0em}
\item[] The reduced scattering coefficient as a function of wavelength, Mie scattering only.
\item[] For details please check Jacques 2013 {[}J13{]}.
\end{DUlineblock}

\sphinxAtStartPar
\(\mu_s'(\lambda) = B \lambda^{-b}\)
\begin{quote}\begin{description}
\sphinxlineitem{Parameters}\begin{itemize}
\item {} 
\sphinxAtStartPar
\sphinxstyleliteralstrong{\sphinxupquote{lambda0}} (\sphinxstyleliteralemphasis{\sphinxupquote{float}}\sphinxstyleliteralemphasis{\sphinxupquote{ or }}\sphinxstyleliteralemphasis{\sphinxupquote{np.ndarray}}) \textendash{} wavelength {[}nm{]}

\item {} 
\sphinxAtStartPar
\sphinxstyleliteralstrong{\sphinxupquote{B}} (\sphinxstyleliteralemphasis{\sphinxupquote{float}}) \textendash{} parameter B {[}mm\sphinxhyphen{}1 nmb{]} (must be nonnegative)

\item {} 
\sphinxAtStartPar
\sphinxstyleliteralstrong{\sphinxupquote{b}} (\sphinxstyleliteralemphasis{\sphinxupquote{float}}) \textendash{} Mie scattering power {[}\sphinxhyphen{}{]} (must be nonnegative)

\end{itemize}

\sphinxlineitem{Returns}
\sphinxAtStartPar
\begin{itemize}
\item {} 
\sphinxAtStartPar
\sphinxstylestrong{rmus} (\sphinxstyleemphasis{float or np.ndarray}) \textendash{} reduced scattering coefficient {[}mm\sphinxhyphen{}1{]}

\end{itemize}


\end{description}\end{quote}

\end{fulllineitems}

\index{rmus\_Ray() (in module skinoptics.scattering\_coefficient)@\spxentry{rmus\_Ray()}\spxextra{in module skinoptics.scattering\_coefficient}}

\begin{fulllineitems}
\phantomsection\label{\detokenize{04_scattering_coefficient:skinoptics.scattering_coefficient.rmus_Ray}}
\pysigstartsignatures
\pysiglinewithargsret{\sphinxcode{\sphinxupquote{skinoptics.scattering\_coefficient.}}\sphinxbfcode{\sphinxupquote{rmus\_Ray}}}{\sphinxparam{\DUrole{n}{lambda0}}\sphinxparamcomma \sphinxparam{\DUrole{n}{A}}}{}
\pysigstopsignatures
\begin{DUlineblock}{0em}
\item[] The reduced scattering coefficient as a function of wavelength, Rayleigh scattering only.
\item[] For details please check Jacques 2013 {[}J13{]}.
\end{DUlineblock}

\sphinxAtStartPar
\(\mu_s'(\lambda) = A \lambda^{-4}\)
\begin{quote}\begin{description}
\sphinxlineitem{Parameters}\begin{itemize}
\item {} 
\sphinxAtStartPar
\sphinxstyleliteralstrong{\sphinxupquote{lambda0}} (\sphinxstyleliteralemphasis{\sphinxupquote{float}}\sphinxstyleliteralemphasis{\sphinxupquote{ or }}\sphinxstyleliteralemphasis{\sphinxupquote{np.ndarray}}) \textendash{} wavelength {[}nm{]}

\item {} 
\sphinxAtStartPar
\sphinxstyleliteralstrong{\sphinxupquote{A}} (\sphinxstyleliteralemphasis{\sphinxupquote{float}}) \textendash{} parameter A {[}mm\sphinxhyphen{}1 nm4{]} (must be nonnegative)

\end{itemize}

\sphinxlineitem{Returns}
\sphinxAtStartPar
\begin{itemize}
\item {} 
\sphinxAtStartPar
\sphinxstylestrong{rmus} (\sphinxstyleemphasis{float or np.ndarray}) \textendash{} reduced scattering coefficient {[}mm\sphinxhyphen{}1{]}

\end{itemize}


\end{description}\end{quote}

\end{fulllineitems}

\index{rmus\_SCC\_Salomatina() (in module skinoptics.scattering\_coefficient)@\spxentry{rmus\_SCC\_Salomatina()}\spxextra{in module skinoptics.scattering\_coefficient}}

\begin{fulllineitems}
\phantomsection\label{\detokenize{04_scattering_coefficient:skinoptics.scattering_coefficient.rmus_SCC_Salomatina}}
\pysigstartsignatures
\pysiglinewithargsret{\sphinxcode{\sphinxupquote{skinoptics.scattering\_coefficient.}}\sphinxbfcode{\sphinxupquote{rmus\_SCC\_Salomatina}}}{\sphinxparam{\DUrole{n}{lambda0}}}{}
\pysigstopsignatures
\begin{DUlineblock}{0em}
\item[] The reduced scattering coefficient of human SQUAMOUS CELL CARCINOMA (SCC)
\item[] as a function of wavelength.
\item[] Linear interpolation of experimental data from Salomatina et al. 2006 {[}S*06{]},
\item[] publicly available at \textless{}\sphinxurl{https://sites.uml.edu/abl/optical-properties-2}\textgreater{}.
\end{DUlineblock}

\begin{DUlineblock}{0em}
\item[] wavelength range: {[}370 nm, 1600 nm{]}
\end{DUlineblock}
\begin{quote}\begin{description}
\sphinxlineitem{Parameters}
\sphinxAtStartPar
\sphinxstyleliteralstrong{\sphinxupquote{lambda0}} (\sphinxstyleliteralemphasis{\sphinxupquote{float}}\sphinxstyleliteralemphasis{\sphinxupquote{ or }}\sphinxstyleliteralemphasis{\sphinxupquote{np.ndarray}}) \textendash{} wavelength {[}nm{]}

\sphinxlineitem{Returns}
\sphinxAtStartPar
\begin{itemize}
\item {} 
\sphinxAtStartPar
\sphinxstylestrong{rmus} (\sphinxstyleemphasis{float or np.ndarray}) \textendash{} reduced scattering coefficient {[}mm\sphinxhyphen{}1{]}

\end{itemize}


\end{description}\end{quote}

\end{fulllineitems}

\index{rmus\_Saidi() (in module skinoptics.scattering\_coefficient)@\spxentry{rmus\_Saidi()}\spxextra{in module skinoptics.scattering\_coefficient}}

\begin{fulllineitems}
\phantomsection\label{\detokenize{04_scattering_coefficient:skinoptics.scattering_coefficient.rmus_Saidi}}
\pysigstartsignatures
\pysiglinewithargsret{\sphinxcode{\sphinxupquote{skinoptics.scattering\_coefficient.}}\sphinxbfcode{\sphinxupquote{rmus\_Saidi}}}{\sphinxparam{\DUrole{n}{lambda0}}\sphinxparamcomma \sphinxparam{\DUrole{n}{A\_Mie}}\sphinxparamcomma \sphinxparam{\DUrole{n}{B\_Ray}}}{}
\pysigstopsignatures
\begin{DUlineblock}{0em}
\item[] The reduced scattering coefficient of NEONATAL SKIN as a function of wavelength.
\item[] Saidi, Jacques \& Tittel 1995 {[}SJT95{]} fit to their own experimental data.
\end{DUlineblock}

\sphinxAtStartPar
\(\mu_s'(\lambda) = A_{Mie} ( 9.843 \times 10^{-7} \lambda^2 - 1.745 \times 10^{-3} \lambda + 1) + B_{Ray} \lambda^{-4}\)

\begin{DUlineblock}{0em}
\item[] wavelength range: {[}450 nm, 750 nm{]}
\item[] gestational ages between 19 and 52 weeks
\end{DUlineblock}
\begin{quote}\begin{description}
\sphinxlineitem{Parameters}\begin{itemize}
\item {} 
\sphinxAtStartPar
\sphinxstyleliteralstrong{\sphinxupquote{lambda0}} (\sphinxstyleliteralemphasis{\sphinxupquote{float}}\sphinxstyleliteralemphasis{\sphinxupquote{ or }}\sphinxstyleliteralemphasis{\sphinxupquote{np.ndarray}}) \textendash{} wavelength {[}nm{]}

\item {} 
\sphinxAtStartPar
\sphinxstyleliteralstrong{\sphinxupquote{A\_Mie}} (\sphinxstyleliteralemphasis{\sphinxupquote{float}}) \textendash{} parameter A\_Mie {[}mm\sphinxhyphen{}1{]} (must be nonnegative)

\item {} 
\sphinxAtStartPar
\sphinxstyleliteralstrong{\sphinxupquote{B\_Ray}} (\sphinxstyleliteralemphasis{\sphinxupquote{float}}) \textendash{} parameter B\_Ray {[}mm\sphinxhyphen{}1 nm4{]} (must be nonnegative)

\end{itemize}

\sphinxlineitem{Returns}
\sphinxAtStartPar
\begin{itemize}
\item {} 
\sphinxAtStartPar
\sphinxstylestrong{rmus} (\sphinxstyleemphasis{float or np.ndarray}) \textendash{} reduced scattering coefficient {[}mm\sphinxhyphen{}1{]}

\end{itemize}


\end{description}\end{quote}

\end{fulllineitems}

\index{rmus\_from\_mus() (in module skinoptics.scattering\_coefficient)@\spxentry{rmus\_from\_mus()}\spxextra{in module skinoptics.scattering\_coefficient}}

\begin{fulllineitems}
\phantomsection\label{\detokenize{04_scattering_coefficient:skinoptics.scattering_coefficient.rmus_from_mus}}
\pysigstartsignatures
\pysiglinewithargsret{\sphinxcode{\sphinxupquote{skinoptics.scattering\_coefficient.}}\sphinxbfcode{\sphinxupquote{rmus\_from\_mus}}}{\sphinxparam{\DUrole{n}{mus}}\sphinxparamcomma \sphinxparam{\DUrole{n}{g}}}{}
\pysigstopsignatures
\begin{DUlineblock}{0em}
\item[] Calculate the reduced scattering coefficient from the scattering coefficient and the 
\item[] anisotropy factor.
\end{DUlineblock}

\sphinxAtStartPar
\(\mu_s' = \mu_s (1-g)\)
\begin{quote}\begin{description}
\sphinxlineitem{Parameters}\begin{itemize}
\item {} 
\sphinxAtStartPar
\sphinxstyleliteralstrong{\sphinxupquote{mus}} (\sphinxstyleliteralemphasis{\sphinxupquote{float}}\sphinxstyleliteralemphasis{\sphinxupquote{ or }}\sphinxstyleliteralemphasis{\sphinxupquote{np.ndarray}}) \textendash{} scattering coefficient {[}mm\sphinxhyphen{}1{]}

\item {} 
\sphinxAtStartPar
\sphinxstyleliteralstrong{\sphinxupquote{g}} (\sphinxstyleliteralemphasis{\sphinxupquote{float}}) \textendash{} anisotropy factor {[}\sphinxhyphen{}{]} (must be in the range {[}\sphinxhyphen{}1, 1{]})

\end{itemize}

\sphinxlineitem{Returns}
\sphinxAtStartPar
\begin{itemize}
\item {} 
\sphinxAtStartPar
\sphinxstylestrong{rmus} (\sphinxstyleemphasis{float or np.ndarray}) \textendash{} reduced scattering coefficient {[}mm\sphinxhyphen{}1{]}

\end{itemize}


\end{description}\end{quote}

\end{fulllineitems}

\index{rmus\_iBCC\_Salomatina() (in module skinoptics.scattering\_coefficient)@\spxentry{rmus\_iBCC\_Salomatina()}\spxextra{in module skinoptics.scattering\_coefficient}}

\begin{fulllineitems}
\phantomsection\label{\detokenize{04_scattering_coefficient:skinoptics.scattering_coefficient.rmus_iBCC_Salomatina}}
\pysigstartsignatures
\pysiglinewithargsret{\sphinxcode{\sphinxupquote{skinoptics.scattering\_coefficient.}}\sphinxbfcode{\sphinxupquote{rmus\_iBCC\_Salomatina}}}{\sphinxparam{\DUrole{n}{lambda0}}}{}
\pysigstopsignatures
\begin{DUlineblock}{0em}
\item[] The reduced scattering coefficient of human INFILTRATIVE BASAL CELL CARCINOMA (iBCC)
\item[] as a function of wavelength.
\item[] Linear interpolation of experimental data from Salomatina et al. 2006 {[}S*06{]},
\item[] publicly available at \textless{}\sphinxurl{https://sites.uml.edu/abl/optical-properties-2}\textgreater{}.
\end{DUlineblock}

\begin{DUlineblock}{0em}
\item[] wavelength range: {[}370 nm, 1600 nm{]}
\end{DUlineblock}
\begin{quote}\begin{description}
\sphinxlineitem{Parameters}
\sphinxAtStartPar
\sphinxstyleliteralstrong{\sphinxupquote{lambda0}} (\sphinxstyleliteralemphasis{\sphinxupquote{float}}\sphinxstyleliteralemphasis{\sphinxupquote{ or }}\sphinxstyleliteralemphasis{\sphinxupquote{np.ndarray}}) \textendash{} wavelength {[}nm{]}

\sphinxlineitem{Returns}
\sphinxAtStartPar
\begin{itemize}
\item {} 
\sphinxAtStartPar
\sphinxstylestrong{rmus} (\sphinxstyleemphasis{float or np.ndarray}) \textendash{} reduced scattering coefficient {[}mm\sphinxhyphen{}1{]}

\end{itemize}


\end{description}\end{quote}

\end{fulllineitems}

\index{rmus\_nBCC\_Salomatina() (in module skinoptics.scattering\_coefficient)@\spxentry{rmus\_nBCC\_Salomatina()}\spxextra{in module skinoptics.scattering\_coefficient}}

\begin{fulllineitems}
\phantomsection\label{\detokenize{04_scattering_coefficient:skinoptics.scattering_coefficient.rmus_nBCC_Salomatina}}
\pysigstartsignatures
\pysiglinewithargsret{\sphinxcode{\sphinxupquote{skinoptics.scattering\_coefficient.}}\sphinxbfcode{\sphinxupquote{rmus\_nBCC\_Salomatina}}}{\sphinxparam{\DUrole{n}{lambda0}}}{}
\pysigstopsignatures
\begin{DUlineblock}{0em}
\item[] The reduced scattering coefficient of human NODULAR BASAL CELL CARCINOMA (nBCC)
\item[] as a function of wavelength.
\item[] Linear interpolation of experimental data from Salomatina et al. 2006 {[}S*06{]},
\item[] publicly available at \textless{}\sphinxurl{https://sites.uml.edu/abl/optical-properties-2}\textgreater{}.
\end{DUlineblock}

\begin{DUlineblock}{0em}
\item[] wavelength range: {[}370 nm, 1600 nm{]}
\end{DUlineblock}
\begin{quote}\begin{description}
\sphinxlineitem{Parameters}
\sphinxAtStartPar
\sphinxstyleliteralstrong{\sphinxupquote{lambda0}} (\sphinxstyleliteralemphasis{\sphinxupquote{float}}\sphinxstyleliteralemphasis{\sphinxupquote{ or }}\sphinxstyleliteralemphasis{\sphinxupquote{np.ndarray}}) \textendash{} wavelength {[}nm{]}

\sphinxlineitem{Returns}
\sphinxAtStartPar
\begin{itemize}
\item {} 
\sphinxAtStartPar
\sphinxstylestrong{rmus} (\sphinxstyleemphasis{float or np.ndarray}) \textendash{} reduced scattering coefficient {[}mm\sphinxhyphen{}1{]}

\end{itemize}


\end{description}\end{quote}

\end{fulllineitems}

\index{std\_rmus\_DE\_Salomatina() (in module skinoptics.scattering\_coefficient)@\spxentry{std\_rmus\_DE\_Salomatina()}\spxextra{in module skinoptics.scattering\_coefficient}}

\begin{fulllineitems}
\phantomsection\label{\detokenize{04_scattering_coefficient:skinoptics.scattering_coefficient.std_rmus_DE_Salomatina}}
\pysigstartsignatures
\pysiglinewithargsret{\sphinxcode{\sphinxupquote{skinoptics.scattering\_coefficient.}}\sphinxbfcode{\sphinxupquote{std\_rmus\_DE\_Salomatina}}}{\sphinxparam{\DUrole{n}{lambda0}}}{}
\pysigstopsignatures
\begin{DUlineblock}{0em}
\item[] The standard deviation respective to rmus\_DE\_Salomatina.
\item[] Linear interpolation of experimental data from Salomatina et al. 2006 {[}S*06{]},
\item[] publicly available at \textless{}\sphinxurl{https://sites.uml.edu/abl/optical-properties-2}\textgreater{}.
\end{DUlineblock}

\begin{DUlineblock}{0em}
\item[] wavelength range: {[}370 nm, 1600 nm{]}
\end{DUlineblock}
\begin{quote}\begin{description}
\sphinxlineitem{Parameters}
\sphinxAtStartPar
\sphinxstyleliteralstrong{\sphinxupquote{lambda0}} (\sphinxstyleliteralemphasis{\sphinxupquote{float}}\sphinxstyleliteralemphasis{\sphinxupquote{ or }}\sphinxstyleliteralemphasis{\sphinxupquote{np.ndarray}}) \textendash{} wavelength {[}nm{]}

\sphinxlineitem{Returns}
\sphinxAtStartPar
\begin{itemize}
\item {} 
\sphinxAtStartPar
\sphinxstylestrong{std\_rmus} (\sphinxstyleemphasis{float or np.ndarray}) \textendash{} standard deviation of reduced scattering coefficient {[}mm\sphinxhyphen{}1{]}

\end{itemize}


\end{description}\end{quote}

\end{fulllineitems}

\index{std\_rmus\_EP\_Salomatina() (in module skinoptics.scattering\_coefficient)@\spxentry{std\_rmus\_EP\_Salomatina()}\spxextra{in module skinoptics.scattering\_coefficient}}

\begin{fulllineitems}
\phantomsection\label{\detokenize{04_scattering_coefficient:skinoptics.scattering_coefficient.std_rmus_EP_Salomatina}}
\pysigstartsignatures
\pysiglinewithargsret{\sphinxcode{\sphinxupquote{skinoptics.scattering\_coefficient.}}\sphinxbfcode{\sphinxupquote{std\_rmus\_EP\_Salomatina}}}{\sphinxparam{\DUrole{n}{lambda0}}}{}
\pysigstopsignatures
\begin{DUlineblock}{0em}
\item[] The standard deviation respective to rmus\_EP\_Salomatina.
\item[] Linear interpolation of experimental data from Salomatina et al. 2006 {[}S*06{]},
\item[] publicly available at \textless{}\sphinxurl{https://sites.uml.edu/abl/optical-properties-2}\textgreater{}.
\end{DUlineblock}

\begin{DUlineblock}{0em}
\item[] wavelength range: {[}370 nm, 1600 nm{]}
\end{DUlineblock}
\begin{quote}\begin{description}
\sphinxlineitem{Parameters}
\sphinxAtStartPar
\sphinxstyleliteralstrong{\sphinxupquote{lambda0}} (\sphinxstyleliteralemphasis{\sphinxupquote{float}}\sphinxstyleliteralemphasis{\sphinxupquote{ or }}\sphinxstyleliteralemphasis{\sphinxupquote{np.ndarray}}) \textendash{} wavelength {[}nm{]}

\sphinxlineitem{Returns}
\sphinxAtStartPar
\begin{itemize}
\item {} 
\sphinxAtStartPar
\sphinxstylestrong{std\_rmus} (\sphinxstyleemphasis{float or np.ndarray}) \textendash{} standard deviation of reduced scattering coefficient {[}mm\sphinxhyphen{}1{]}

\end{itemize}


\end{description}\end{quote}

\end{fulllineitems}

\index{std\_rmus\_HY\_Salomatina() (in module skinoptics.scattering\_coefficient)@\spxentry{std\_rmus\_HY\_Salomatina()}\spxextra{in module skinoptics.scattering\_coefficient}}

\begin{fulllineitems}
\phantomsection\label{\detokenize{04_scattering_coefficient:skinoptics.scattering_coefficient.std_rmus_HY_Salomatina}}
\pysigstartsignatures
\pysiglinewithargsret{\sphinxcode{\sphinxupquote{skinoptics.scattering\_coefficient.}}\sphinxbfcode{\sphinxupquote{std\_rmus\_HY\_Salomatina}}}{\sphinxparam{\DUrole{n}{lambda0}}}{}
\pysigstopsignatures
\begin{DUlineblock}{0em}
\item[] The standard deviation respective to rmus\_HY\_Salomatina.
\item[] Linear interpolation of experimental data from Salomatina et al. 2006 {[}S*06{]},
\item[] publicly available at \textless{}\sphinxurl{https://sites.uml.edu/abl/optical-properties-2}\textgreater{}.
\end{DUlineblock}

\begin{DUlineblock}{0em}
\item[] wavelength range: {[}370 nm, 1600 nm{]}
\end{DUlineblock}
\begin{quote}\begin{description}
\sphinxlineitem{Parameters}
\sphinxAtStartPar
\sphinxstyleliteralstrong{\sphinxupquote{lambda0}} (\sphinxstyleliteralemphasis{\sphinxupquote{float}}\sphinxstyleliteralemphasis{\sphinxupquote{ or }}\sphinxstyleliteralemphasis{\sphinxupquote{np.ndarray}}) \textendash{} wavelength {[}nm{]}

\sphinxlineitem{Returns}
\sphinxAtStartPar
\begin{itemize}
\item {} 
\sphinxAtStartPar
\sphinxstylestrong{std\_rmus} (\sphinxstyleemphasis{float or np.ndarray}) \textendash{} standard deviation of reduced scattering coefficient {[}mm\sphinxhyphen{}1{]}

\end{itemize}


\end{description}\end{quote}

\end{fulllineitems}


\sphinxstepscope


\subsection{skinoptics.refractive\_index}
\label{\detokenize{05_refractive_index:module-skinoptics.refractive_index}}\label{\detokenize{05_refractive_index:skinoptics-refractive-index}}\label{\detokenize{05_refractive_index::doc}}\index{module@\spxentry{module}!skinoptics.refractive\_index@\spxentry{skinoptics.refractive\_index}}\index{skinoptics.refractive\_index@\spxentry{skinoptics.refractive\_index}!module@\spxentry{module}}
\sphinxAtStartPar
Copyright (C) 2024 Victor Lima
\begin{quote}

\begin{DUlineblock}{0em}
\item[] This program is free software: you can redistribute it and/or modify
\item[] it under the terms of the GNU General Public License as published by
\item[] the Free Software Foundation, either version 3 of the License, or
\item[] (at your option) any later version.
\end{DUlineblock}

\begin{DUlineblock}{0em}
\item[] This program is distributed in the hope that it will be useful,
\item[] but WITHOUT ANY WARRANTY; without even the implied warranty of
\item[] MERCHANTABILITY or FITNESS FOR A PARTICULAR PURPOSE.  See the
\item[] GNU General Public License for more details.
\end{DUlineblock}

\begin{DUlineblock}{0em}
\item[] You should have received a copy of the GNU General Public License
\item[] along with this program.  If not, see \textless{}\sphinxurl{https://www.gnu.org/licenses/}\textgreater{}.
\end{DUlineblock}
\end{quote}

\begin{DUlineblock}{0em}
\item[] Victor Lima
\item[] victorporto@ifsc.usp.br
\item[] victor.lima@ufscar.br
\end{DUlineblock}

\begin{DUlineblock}{0em}
\item[] Release Date:
\item[] August 2024
\item[] Last Modification:
\item[] August 2024
\end{DUlineblock}

\begin{DUlineblock}{0em}
\item[] References:
\end{DUlineblock}

\begin{DUlineblock}{0em}
\item[] {[}HQ73{]} Hale \& Querry 1973.
\item[] Optical Constants of Water in the 200\sphinxhyphen{}nm to 200\sphinxhyphen{}μm Wavelength Region.
\item[] \sphinxurl{https://doi.org/10.1364/AO.12.000555}
\end{DUlineblock}

\begin{DUlineblock}{0em}
\item[] {[}S81{]} Segelstein’s M.S. Thesis 1981.
\item[] The complex refractive index of water.
\item[] \sphinxurl{https://mospace.umsystem.edu/xmlui/handle/10355/11599}
\end{DUlineblock}

\begin{DUlineblock}{0em}
\item[] {[}LLX00{]} Li, Lin \& Xie 2000.
\item[] Refractive index of human whole blood with different types in the visible and near\sphinxhyphen{}infrared ranges.
\item[] \sphinxurl{https://doi.org/10.1117/12.388073}
\end{DUlineblock}

\begin{DUlineblock}{0em}
\item[] {[}D*06{]} Ding, Lu, Wooden, Kragel \& Hu 2006.
\item[] Refractive indices of human skin tissues at eight wavelengths and estimated dispersion relations
\item[] between 300 and 1600 nm.
\item[] \sphinxurl{https://doi.org/10.1088/0031-9155/51/6/008}
\end{DUlineblock}

\begin{DUlineblock}{0em}
\item[] {[}FM06{]} Friebel \& Meinke 2006
\item[] Model function to calculate the refractive index of native hemoglobin in the wavelength range of
\item[] 250\textendash{}1100 nm dependent on concentration.
\item[] \sphinxurl{https://doi.org/10.1364/AO.45.002838}
\end{DUlineblock}

\begin{DUlineblock}{0em}
\item[] {[}YLT18{]} Yanina, Lazareva \& Tuchin 2018.
\item[] Refractive index of adipose tissue and lipid droplet measured in wide spectral and temperature ranges.
\item[] \sphinxurl{https://doi.org/10.1364/AO.57.004839}
\end{DUlineblock}

\begin{DUlineblock}{0em}
\item[] {[}M*21{]} Matiatou, Giannios, Koutsoumpos, Toutouzas, Zografos \& Moutzouris 2021.
\item[] Data on the refractive index of freshly\sphinxhyphen{}excised human tissues in the visible and near\sphinxhyphen{}infrared
\item[] spectral range.
\item[] \sphinxurl{https://doi.org/10.1016/j.rinp.2021.103833}
\end{DUlineblock}
\index{beta\_oxy\_Friebel() (in module skinoptics.refractive\_index)@\spxentry{beta\_oxy\_Friebel()}\spxextra{in module skinoptics.refractive\_index}}

\begin{fulllineitems}
\phantomsection\label{\detokenize{05_refractive_index:skinoptics.refractive_index.beta_oxy_Friebel}}
\pysigstartsignatures
\pysiglinewithargsret{\sphinxcode{\sphinxupquote{skinoptics.refractive\_index.}}\sphinxbfcode{\sphinxupquote{beta\_oxy\_Friebel}}}{\sphinxparam{\DUrole{n}{lambda0}}}{}
\pysigstopsignatures
\begin{DUlineblock}{0em}
\item[] The specific refractive increment of OXY\sphinxhyphen{}HEMOGLOBIN SOLUTIONS as a function of wavelength.
\item[] Linear interpolation of experimental data from Friebel \& Meinke 2006 {[}FM06{]}.
\end{DUlineblock}

\begin{DUlineblock}{0em}
\item[] wavelength range: {[}250 nm, 1100 nm{]}
\end{DUlineblock}
\begin{quote}\begin{description}
\sphinxlineitem{Parameters}
\sphinxAtStartPar
\sphinxstyleliteralstrong{\sphinxupquote{lambda0}} (\sphinxstyleliteralemphasis{\sphinxupquote{float}}\sphinxstyleliteralemphasis{\sphinxupquote{ or }}\sphinxstyleliteralemphasis{\sphinxupquote{np.ndarray}}) \textendash{} wavelength {[}nm{]} (must be in the range {[}250., 1100.{]})

\sphinxlineitem{Returns}
\sphinxAtStartPar
\begin{itemize}
\item {} 
\sphinxAtStartPar
\sphinxstylestrong{beta} (\sphinxstyleemphasis{float or np.ndarray}) \textendash{} specific refractive increment {[}dL/g{]}

\end{itemize}


\end{description}\end{quote}

\end{fulllineitems}

\index{n\_AT\_Matiatou() (in module skinoptics.refractive\_index)@\spxentry{n\_AT\_Matiatou()}\spxextra{in module skinoptics.refractive\_index}}

\begin{fulllineitems}
\phantomsection\label{\detokenize{05_refractive_index:skinoptics.refractive_index.n_AT_Matiatou}}
\pysigstartsignatures
\pysiglinewithargsret{\sphinxcode{\sphinxupquote{skinoptics.refractive\_index.}}\sphinxbfcode{\sphinxupquote{n\_AT\_Matiatou}}}{\sphinxparam{\DUrole{n}{lambda0}}}{}
\pysigstopsignatures
\begin{DUlineblock}{0em}
\item[] The refractive index of the human ADIPOSE TISSUE as a function of wavelength.
\item[] Matiatou et al. 2021 {[}M*21{]}’s fit to their own experimental data.
\end{DUlineblock}

\sphinxAtStartPar
\(n(\lambda) = 1.44933 + \frac{4908.37}{\lambda^2}\)

\begin{DUlineblock}{0em}
\item[] wavelength range: {[}450 nm, 1551 nm{]}
\item[] temperature: 25 ºC
\item[] body location: abdominen
\end{DUlineblock}
\begin{quote}\begin{description}
\sphinxlineitem{Parameters}
\sphinxAtStartPar
\sphinxstyleliteralstrong{\sphinxupquote{lambda0}} (\sphinxstyleliteralemphasis{\sphinxupquote{float}}\sphinxstyleliteralemphasis{\sphinxupquote{ or }}\sphinxstyleliteralemphasis{\sphinxupquote{np.ndarray}}) \textendash{} wavelength {[}nm{]}

\sphinxlineitem{Returns}
\sphinxAtStartPar
\begin{itemize}
\item {} 
\sphinxAtStartPar
\sphinxstylestrong{n} (\sphinxstyleemphasis{float or np.ndarray}) \textendash{} refractive index {[}\sphinxhyphen{}{]}

\end{itemize}


\end{description}\end{quote}

\end{fulllineitems}

\index{n\_AT\_Yanina() (in module skinoptics.refractive\_index)@\spxentry{n\_AT\_Yanina()}\spxextra{in module skinoptics.refractive\_index}}

\begin{fulllineitems}
\phantomsection\label{\detokenize{05_refractive_index:skinoptics.refractive_index.n_AT_Yanina}}
\pysigstartsignatures
\pysiglinewithargsret{\sphinxcode{\sphinxupquote{skinoptics.refractive\_index.}}\sphinxbfcode{\sphinxupquote{n\_AT\_Yanina}}}{\sphinxparam{\DUrole{n}{lambda0}}}{}
\pysigstopsignatures
\begin{DUlineblock}{0em}
\item[] The refractive index of the human ADIPOSE TISSUE as a function of wavelength.
\item[] Yanina, Lazareva \& Tuchin 2018 {[}YLT18{]}’s fit to their own experimental data.
\end{DUlineblock}

\sphinxAtStartPar
\(n(\lambda) = \sqrt{1 + \frac{1.1236\lambda^2}{\lambda^2-10556.6963} + \frac{0.2725\lambda^2}{\lambda^2-1.8867\times 10^7}}\)

\begin{DUlineblock}{0em}
\item[] wavelength range: {[}480 nm, 1550 nm{]}
\item[] temperature: 23 ºC
\item[] body location: abdomen
\item[] volunteers info: 10 biopsies, 5 men, 40 \textendash{} 50 years old, 70 \textendash{} 80 kg
\end{DUlineblock}
\begin{quote}\begin{description}
\sphinxlineitem{Parameters}
\sphinxAtStartPar
\sphinxstyleliteralstrong{\sphinxupquote{lambda0}} (\sphinxstyleliteralemphasis{\sphinxupquote{float}}\sphinxstyleliteralemphasis{\sphinxupquote{ or }}\sphinxstyleliteralemphasis{\sphinxupquote{np.ndarray}}) \textendash{} wavelength {[}nm{]}

\sphinxlineitem{Returns}
\sphinxAtStartPar
\begin{itemize}
\item {} 
\sphinxAtStartPar
\sphinxstylestrong{n} (\sphinxstyleemphasis{float or np.ndarray}) \textendash{} refractive index {[}\sphinxhyphen{}{]}

\end{itemize}


\end{description}\end{quote}

\end{fulllineitems}

\index{n\_Cauchy() (in module skinoptics.refractive\_index)@\spxentry{n\_Cauchy()}\spxextra{in module skinoptics.refractive\_index}}

\begin{fulllineitems}
\phantomsection\label{\detokenize{05_refractive_index:skinoptics.refractive_index.n_Cauchy}}
\pysigstartsignatures
\pysiglinewithargsret{\sphinxcode{\sphinxupquote{skinoptics.refractive\_index.}}\sphinxbfcode{\sphinxupquote{n\_Cauchy}}}{\sphinxparam{\DUrole{n}{lambda0}}\sphinxparamcomma \sphinxparam{\DUrole{n}{A0}}\sphinxparamcomma \sphinxparam{\DUrole{n}{A1}}\sphinxparamcomma \sphinxparam{\DUrole{n}{A2}}\sphinxparamcomma \sphinxparam{\DUrole{n}{A3}}}{}
\pysigstopsignatures
\sphinxAtStartPar
The Cauchy’s equation.

\sphinxAtStartPar
\(n(\lambda) = A_0 + \frac{A_1}{\lambda^2} + \frac{A_2}{\lambda^4} + \frac{A_3}{\lambda^6}\)
\begin{quote}\begin{description}
\sphinxlineitem{Parameters}\begin{itemize}
\item {} 
\sphinxAtStartPar
\sphinxstyleliteralstrong{\sphinxupquote{lambda0}} (\sphinxstyleliteralemphasis{\sphinxupquote{float}}\sphinxstyleliteralemphasis{\sphinxupquote{ or }}\sphinxstyleliteralemphasis{\sphinxupquote{np.ndarray}}) \textendash{} wavelength {[}nm{]}

\item {} 
\sphinxAtStartPar
\sphinxstyleliteralstrong{\sphinxupquote{A0}} (\sphinxstyleliteralemphasis{\sphinxupquote{float}}) \textendash{} coefficient \(A_0\) {[}\sphinxhyphen{}{]}

\item {} 
\sphinxAtStartPar
\sphinxstyleliteralstrong{\sphinxupquote{A1}} (\sphinxstyleliteralemphasis{\sphinxupquote{float}}) \textendash{} coefficient \(A_1\) {[}nm2{]}

\item {} 
\sphinxAtStartPar
\sphinxstyleliteralstrong{\sphinxupquote{A2}} (\sphinxstyleliteralemphasis{\sphinxupquote{float}}) \textendash{} coefficient \(A_2\) {[}nm4{]}

\item {} 
\sphinxAtStartPar
\sphinxstyleliteralstrong{\sphinxupquote{A3}} (\sphinxstyleliteralemphasis{\sphinxupquote{float}}) \textendash{} coefficient \(A_3\) {[}nm6{]}

\end{itemize}

\sphinxlineitem{Returns}
\sphinxAtStartPar
\begin{itemize}
\item {} 
\sphinxAtStartPar
\sphinxstylestrong{n} (\sphinxstyleemphasis{float or np.ndarray}) \textendash{} refractive index {[}\sphinxhyphen{}{]}

\end{itemize}


\end{description}\end{quote}

\end{fulllineitems}

\index{n\_Conrady() (in module skinoptics.refractive\_index)@\spxentry{n\_Conrady()}\spxextra{in module skinoptics.refractive\_index}}

\begin{fulllineitems}
\phantomsection\label{\detokenize{05_refractive_index:skinoptics.refractive_index.n_Conrady}}
\pysigstartsignatures
\pysiglinewithargsret{\sphinxcode{\sphinxupquote{skinoptics.refractive\_index.}}\sphinxbfcode{\sphinxupquote{n\_Conrady}}}{\sphinxparam{\DUrole{n}{lambda0}}\sphinxparamcomma \sphinxparam{\DUrole{n}{A}}\sphinxparamcomma \sphinxparam{\DUrole{n}{B}}\sphinxparamcomma \sphinxparam{\DUrole{n}{C}}}{}
\pysigstopsignatures
\sphinxAtStartPar
The Conrady’s equation.

\sphinxAtStartPar
\(n(\lambda) = A + \frac{B}{\lambda} + \frac{C}{\lambda^{3.5}}\)
\begin{quote}\begin{description}
\sphinxlineitem{Parameters}\begin{itemize}
\item {} 
\sphinxAtStartPar
\sphinxstyleliteralstrong{\sphinxupquote{lambda0}} (\sphinxstyleliteralemphasis{\sphinxupquote{float}}\sphinxstyleliteralemphasis{\sphinxupquote{ or }}\sphinxstyleliteralemphasis{\sphinxupquote{np.ndarray}}) \textendash{} wavelength {[}nm{]}

\item {} 
\sphinxAtStartPar
\sphinxstyleliteralstrong{\sphinxupquote{A}} (\sphinxstyleliteralemphasis{\sphinxupquote{float}}) \textendash{} coefficient A {[}\sphinxhyphen{}{]}

\item {} 
\sphinxAtStartPar
\sphinxstyleliteralstrong{\sphinxupquote{B}} (\sphinxstyleliteralemphasis{\sphinxupquote{float}}) \textendash{} coefficient B {[}nm{]}

\item {} 
\sphinxAtStartPar
\sphinxstyleliteralstrong{\sphinxupquote{C}} (\sphinxstyleliteralemphasis{\sphinxupquote{float}}) \textendash{} coefficient C {[}nm:math:\sphinxtitleref{\textasciicircum{}\{3.5\}}{]}

\end{itemize}

\sphinxlineitem{Returns}
\sphinxAtStartPar
\begin{itemize}
\item {} 
\sphinxAtStartPar
\sphinxstylestrong{n} (\sphinxstyleemphasis{float or np.ndarray}) \textendash{} refractive index {[}\sphinxhyphen{}{]}

\end{itemize}


\end{description}\end{quote}

\end{fulllineitems}

\index{n\_Cornu() (in module skinoptics.refractive\_index)@\spxentry{n\_Cornu()}\spxextra{in module skinoptics.refractive\_index}}

\begin{fulllineitems}
\phantomsection\label{\detokenize{05_refractive_index:skinoptics.refractive_index.n_Cornu}}
\pysigstartsignatures
\pysiglinewithargsret{\sphinxcode{\sphinxupquote{skinoptics.refractive\_index.}}\sphinxbfcode{\sphinxupquote{n\_Cornu}}}{\sphinxparam{\DUrole{n}{lambda0}}\sphinxparamcomma \sphinxparam{\DUrole{n}{A}}\sphinxparamcomma \sphinxparam{\DUrole{n}{B}}\sphinxparamcomma \sphinxparam{\DUrole{n}{C}}}{}
\pysigstopsignatures
\sphinxAtStartPar
The Cornu’s equation.

\sphinxAtStartPar
\(n(\lambda) = A + \frac{B}{(\lambda - C)}\)
\begin{quote}\begin{description}
\sphinxlineitem{Parameters}\begin{itemize}
\item {} 
\sphinxAtStartPar
\sphinxstyleliteralstrong{\sphinxupquote{lambda0}} (\sphinxstyleliteralemphasis{\sphinxupquote{float}}\sphinxstyleliteralemphasis{\sphinxupquote{ or }}\sphinxstyleliteralemphasis{\sphinxupquote{np.ndarray}}) \textendash{} wavelength {[}nm{]}

\item {} 
\sphinxAtStartPar
\sphinxstyleliteralstrong{\sphinxupquote{A}} (\sphinxstyleliteralemphasis{\sphinxupquote{float}}) \textendash{} coefficient A {[}\sphinxhyphen{}{]}

\item {} 
\sphinxAtStartPar
\sphinxstyleliteralstrong{\sphinxupquote{B}} (\sphinxstyleliteralemphasis{\sphinxupquote{float}}) \textendash{} coefficient B {[}nm{]}

\item {} 
\sphinxAtStartPar
\sphinxstyleliteralstrong{\sphinxupquote{C}} (\sphinxstyleliteralemphasis{\sphinxupquote{float}}) \textendash{} coefficient C {[}nm{]}

\end{itemize}

\sphinxlineitem{Returns}
\sphinxAtStartPar
\begin{itemize}
\item {} 
\sphinxAtStartPar
\sphinxstylestrong{n} (\sphinxstyleemphasis{float or np.ndarray}) \textendash{} refractive index {[}\sphinxhyphen{}{]}

\end{itemize}


\end{description}\end{quote}

\end{fulllineitems}

\index{n\_DE\_Ding() (in module skinoptics.refractive\_index)@\spxentry{n\_DE\_Ding()}\spxextra{in module skinoptics.refractive\_index}}

\begin{fulllineitems}
\phantomsection\label{\detokenize{05_refractive_index:skinoptics.refractive_index.n_DE_Ding}}
\pysigstartsignatures
\pysiglinewithargsret{\sphinxcode{\sphinxupquote{skinoptics.refractive\_index.}}\sphinxbfcode{\sphinxupquote{n\_DE\_Ding}}}{\sphinxparam{\DUrole{n}{lambda0}}\sphinxparamcomma \sphinxparam{\DUrole{n}{model}\DUrole{o}{=}\DUrole{default_value}{\textquotesingle{}Cauchy\textquotesingle{}}}}{}
\pysigstopsignatures
\begin{DUlineblock}{0em}
\item[] The refractive index of the human DERMIS as a function of wavelength.
\item[] Ding et al. 2006 {[}D*06{]}’s fits to their own experimental data.
\item[] Complementary data publicly available at \textless{}bmlaser.physics.ecu.edu/literature/lit.htm\textgreater{}.
\end{DUlineblock}

\begin{DUlineblock}{0em}
\item[] wavelength range: {[}325 nm, 1557 nm{]}
\item[] temperature: 22 ºC
\item[] body location: abdomen and arm
\item[] volunteers info: 12 female patients (10 caucasians and 2 african americans),
\item[] phototypes I\sphinxhyphen{}III and V, 27\sphinxhyphen{}63 years old
\end{DUlineblock}
\begin{quote}\begin{description}
\sphinxlineitem{Parameters}\begin{itemize}
\item {} 
\sphinxAtStartPar
\sphinxstyleliteralstrong{\sphinxupquote{lambda0}} (\sphinxstyleliteralemphasis{\sphinxupquote{float}}\sphinxstyleliteralemphasis{\sphinxupquote{ or }}\sphinxstyleliteralemphasis{\sphinxupquote{np.ndarray}}) \textendash{} wavelength {[}nm{]}

\item {} 
\sphinxAtStartPar
\sphinxstyleliteralstrong{\sphinxupquote{model}} \textendash{} the user can choose one of the following… ‘Cauchy’, ‘Cornu’ or ‘Conrady’ (default to ‘Cauchy’)

\end{itemize}

\sphinxlineitem{Returns}
\sphinxAtStartPar
\begin{itemize}
\item {} 
\sphinxAtStartPar
\sphinxstylestrong{n} (\sphinxstyleemphasis{float or np.ndarray}) \textendash{} refractive index {[}\sphinxhyphen{}{]}

\end{itemize}


\end{description}\end{quote}

\end{fulllineitems}

\index{n\_EP\_Ding() (in module skinoptics.refractive\_index)@\spxentry{n\_EP\_Ding()}\spxextra{in module skinoptics.refractive\_index}}

\begin{fulllineitems}
\phantomsection\label{\detokenize{05_refractive_index:skinoptics.refractive_index.n_EP_Ding}}
\pysigstartsignatures
\pysiglinewithargsret{\sphinxcode{\sphinxupquote{skinoptics.refractive\_index.}}\sphinxbfcode{\sphinxupquote{n\_EP\_Ding}}}{\sphinxparam{\DUrole{n}{lambda0}}\sphinxparamcomma \sphinxparam{\DUrole{n}{model}\DUrole{o}{=}\DUrole{default_value}{\textquotesingle{}Cauchy\textquotesingle{}}}}{}
\pysigstopsignatures
\begin{DUlineblock}{0em}
\item[] The refractive index of the human EPIDERMIS as a function of wavelength.
\item[] Ding et al. 2006 {[}D*06{]}’s fits to their own experimental data.
\item[] Complementary data publicly available at \textless{}bmlaser.physics.ecu.edu/literature/lit.htm\textgreater{}.
\end{DUlineblock}

\begin{DUlineblock}{0em}
\item[] wavelength range: {[}325 nm, 1557 nm{]}
\item[] temperature: 22 ºC
\item[] body location: abdomen and arm
\item[] volunteers info: 12 female patients (10 caucasians and 2 african americans),
\item[] phototypes I\sphinxhyphen{}III and V, 27\sphinxhyphen{}63 years old
\end{DUlineblock}
\begin{quote}\begin{description}
\sphinxlineitem{Parameters}\begin{itemize}
\item {} 
\sphinxAtStartPar
\sphinxstyleliteralstrong{\sphinxupquote{lambda0}} (\sphinxstyleliteralemphasis{\sphinxupquote{float}}\sphinxstyleliteralemphasis{\sphinxupquote{ or }}\sphinxstyleliteralemphasis{\sphinxupquote{np.ndarray}}) \textendash{} wavelength {[}nm{]}

\item {} 
\sphinxAtStartPar
\sphinxstyleliteralstrong{\sphinxupquote{model}} \textendash{} the user can choose one of the following… ‘Cauchy’, ‘Cornu’ or ‘Conrady’ (default to ‘Cauchy’)

\end{itemize}

\sphinxlineitem{Returns}
\sphinxAtStartPar
\begin{itemize}
\item {} 
\sphinxAtStartPar
\sphinxstylestrong{n} (\sphinxstyleemphasis{float or np.ndarray}) \textendash{} refractive index {[}\sphinxhyphen{}{]}

\end{itemize}


\end{description}\end{quote}

\end{fulllineitems}

\index{n\_HY\_Matiatou() (in module skinoptics.refractive\_index)@\spxentry{n\_HY\_Matiatou()}\spxextra{in module skinoptics.refractive\_index}}

\begin{fulllineitems}
\phantomsection\label{\detokenize{05_refractive_index:skinoptics.refractive_index.n_HY_Matiatou}}
\pysigstartsignatures
\pysiglinewithargsret{\sphinxcode{\sphinxupquote{skinoptics.refractive\_index.}}\sphinxbfcode{\sphinxupquote{n\_HY\_Matiatou}}}{\sphinxparam{\DUrole{n}{lambda0}}}{}
\pysigstopsignatures
\begin{DUlineblock}{0em}
\item[] The refractive index of the human HYPODERMIS as a function of wavelength.
\item[] Matiatou et al. 2021 {[}M*21{]}’s fit to their own experimental data.
\end{DUlineblock}

\sphinxAtStartPar
\(n(\lambda) = 1.44909 + \frac{5099.42}{\lambda^2}\)

\begin{DUlineblock}{0em}
\item[] wavelength range: {[}450 nm, 1551 nm{]}
\item[] temperature: 25 ºC
\end{DUlineblock}
\begin{quote}\begin{description}
\sphinxlineitem{Parameters}
\sphinxAtStartPar
\sphinxstyleliteralstrong{\sphinxupquote{lambda0}} (\sphinxstyleliteralemphasis{\sphinxupquote{float}}\sphinxstyleliteralemphasis{\sphinxupquote{ or }}\sphinxstyleliteralemphasis{\sphinxupquote{np.ndarray}}) \textendash{} wavelength {[}nm{]}

\sphinxlineitem{Returns}
\sphinxAtStartPar
\begin{itemize}
\item {} 
\sphinxAtStartPar
\sphinxstylestrong{n} (\sphinxstyleemphasis{float or np.ndarray}) \textendash{} refractive index {[}\sphinxhyphen{}{]}

\end{itemize}


\end{description}\end{quote}

\end{fulllineitems}

\index{n\_Sellmeier() (in module skinoptics.refractive\_index)@\spxentry{n\_Sellmeier()}\spxextra{in module skinoptics.refractive\_index}}

\begin{fulllineitems}
\phantomsection\label{\detokenize{05_refractive_index:skinoptics.refractive_index.n_Sellmeier}}
\pysigstartsignatures
\pysiglinewithargsret{\sphinxcode{\sphinxupquote{skinoptics.refractive\_index.}}\sphinxbfcode{\sphinxupquote{n\_Sellmeier}}}{\sphinxparam{\DUrole{n}{lambda0}}\sphinxparamcomma \sphinxparam{\DUrole{n}{A1}}\sphinxparamcomma \sphinxparam{\DUrole{n}{B1}}\sphinxparamcomma \sphinxparam{\DUrole{n}{A2}}\sphinxparamcomma \sphinxparam{\DUrole{n}{B2}}}{}
\pysigstopsignatures
\sphinxAtStartPar
The Sellmeier’s equation.

\sphinxAtStartPar
\(n(\lambda) = \sqrt{1 + \frac{A_1\lambda^2}{\lambda^2 - B_1} + \frac{A_2\lambda^2}{\lambda^2 - B_2}}\)
\begin{quote}\begin{description}
\sphinxlineitem{Parameters}\begin{itemize}
\item {} 
\sphinxAtStartPar
\sphinxstyleliteralstrong{\sphinxupquote{lambda0}} (\sphinxstyleliteralemphasis{\sphinxupquote{float}}\sphinxstyleliteralemphasis{\sphinxupquote{ or }}\sphinxstyleliteralemphasis{\sphinxupquote{np.ndarray}}) \textendash{} wavelength {[}nm{]}

\item {} 
\sphinxAtStartPar
\sphinxstyleliteralstrong{\sphinxupquote{A1}} (\sphinxstyleliteralemphasis{\sphinxupquote{float}}) \textendash{} coefficient \(A_1\) {[}\sphinxhyphen{}{]}

\item {} 
\sphinxAtStartPar
\sphinxstyleliteralstrong{\sphinxupquote{B1}} (\sphinxstyleliteralemphasis{\sphinxupquote{float}}) \textendash{} coefficient \(B_1\) {[}nm2{]}

\item {} 
\sphinxAtStartPar
\sphinxstyleliteralstrong{\sphinxupquote{A2}} (\sphinxstyleliteralemphasis{\sphinxupquote{float}}) \textendash{} coefficient \(A_2\) {[}\sphinxhyphen{}{]}

\item {} 
\sphinxAtStartPar
\sphinxstyleliteralstrong{\sphinxupquote{B2}} (\sphinxstyleliteralemphasis{\sphinxupquote{float}}) \textendash{} coefficient \(B_2\) {[}nm2{]}

\end{itemize}

\sphinxlineitem{Returns}
\sphinxAtStartPar
\begin{itemize}
\item {} 
\sphinxAtStartPar
\sphinxstylestrong{n} (\sphinxstyleemphasis{float or np.ndarray}) \textendash{} refractive index {[}\sphinxhyphen{}{]}

\end{itemize}


\end{description}\end{quote}

\end{fulllineitems}

\index{n\_blo\_Li() (in module skinoptics.refractive\_index)@\spxentry{n\_blo\_Li()}\spxextra{in module skinoptics.refractive\_index}}

\begin{fulllineitems}
\phantomsection\label{\detokenize{05_refractive_index:skinoptics.refractive_index.n_blo_Li}}
\pysigstartsignatures
\pysiglinewithargsret{\sphinxcode{\sphinxupquote{skinoptics.refractive\_index.}}\sphinxbfcode{\sphinxupquote{n\_blo\_Li}}}{\sphinxparam{\DUrole{n}{lambda0}}}{}
\pysigstopsignatures
\begin{DUlineblock}{0em}
\item[] The refractive index of the human BLOOD as a function of wavelength.
\item[] Li, Lin \& Xie 2000 {[}LLX00{]}’s fit to their own experimental data.
\end{DUlineblock}

\sphinxAtStartPar
\(n(\lambda) = 1.357 + \frac{6.9 \times 10^3}{\lambda^2} + \frac{7.6 \times 10^8}{\lambda^4}\)

\begin{DUlineblock}{0em}
\item[] wavelength range: {[}370 nm, 850 nm{]}
\item[] temperature: 27\sphinxhyphen{}28 ºC
\item[] blood types: A, B and O
\item[] volunteers info: 9 healthy volunteers, chinese, male and female
\end{DUlineblock}
\begin{quote}\begin{description}
\sphinxlineitem{Parameters}
\sphinxAtStartPar
\sphinxstyleliteralstrong{\sphinxupquote{lambda0}} (\sphinxstyleliteralemphasis{\sphinxupquote{float}}\sphinxstyleliteralemphasis{\sphinxupquote{ or }}\sphinxstyleliteralemphasis{\sphinxupquote{np.ndarray}}) \textendash{} wavelength {[}nm{]}

\sphinxlineitem{Returns}
\sphinxAtStartPar
\begin{itemize}
\item {} 
\sphinxAtStartPar
\sphinxstylestrong{n} (\sphinxstyleemphasis{float or np.ndarray}) \textendash{} refractive index {[}\sphinxhyphen{}{]}

\end{itemize}


\end{description}\end{quote}

\end{fulllineitems}

\index{n\_wat\_Hale() (in module skinoptics.refractive\_index)@\spxentry{n\_wat\_Hale()}\spxextra{in module skinoptics.refractive\_index}}

\begin{fulllineitems}
\phantomsection\label{\detokenize{05_refractive_index:skinoptics.refractive_index.n_wat_Hale}}
\pysigstartsignatures
\pysiglinewithargsret{\sphinxcode{\sphinxupquote{skinoptics.refractive\_index.}}\sphinxbfcode{\sphinxupquote{n\_wat\_Hale}}}{\sphinxparam{\DUrole{n}{lambda0}}}{}
\pysigstopsignatures
\begin{DUlineblock}{0em}
\item[] The refractive index of WATER as a function of wavelength.
\item[] Linear interpolation of data from Hale \& Querry 1973 {[}HQ73{]}.
\end{DUlineblock}

\begin{DUlineblock}{0em}
\item[] wavelength range: {[}200 nm, 200 \(\mu\) m{]}
\item[] temperature: 25 ºC
\end{DUlineblock}
\begin{quote}\begin{description}
\sphinxlineitem{Parameters}
\sphinxAtStartPar
\sphinxstyleliteralstrong{\sphinxupquote{lambda0}} (\sphinxstyleliteralemphasis{\sphinxupquote{float}}\sphinxstyleliteralemphasis{\sphinxupquote{ or }}\sphinxstyleliteralemphasis{\sphinxupquote{np.ndarray}}) \textendash{} wavelength {[}nm{]}

\sphinxlineitem{Returns}
\sphinxAtStartPar
\begin{itemize}
\item {} 
\sphinxAtStartPar
\sphinxstylestrong{n} (\sphinxstyleemphasis{float or np.ndarray}) \textendash{} refractive index {[}\sphinxhyphen{}{]}

\end{itemize}


\end{description}\end{quote}

\end{fulllineitems}

\index{n\_wat\_Segelstein() (in module skinoptics.refractive\_index)@\spxentry{n\_wat\_Segelstein()}\spxextra{in module skinoptics.refractive\_index}}

\begin{fulllineitems}
\phantomsection\label{\detokenize{05_refractive_index:skinoptics.refractive_index.n_wat_Segelstein}}
\pysigstartsignatures
\pysiglinewithargsret{\sphinxcode{\sphinxupquote{skinoptics.refractive\_index.}}\sphinxbfcode{\sphinxupquote{n\_wat\_Segelstein}}}{\sphinxparam{\DUrole{n}{lambda0}}}{}
\pysigstopsignatures
\begin{DUlineblock}{0em}
\item[] The refractive index of WATER as a function of wavelength.
\item[] Linear interpolation of data from D. J. Segelstein’s M.S. Thesis 1981 {[}S81{]} collected
\item[] by S. Prahl and publicly available at \textless{}\sphinxurl{https://omlc.org/spectra/water/abs/index.html}\textgreater{}.
\end{DUlineblock}

\begin{DUlineblock}{0em}
\item[] wavelength range: {[}10 nm, 10 m{]}
\end{DUlineblock}
\begin{quote}\begin{description}
\sphinxlineitem{Parameters}
\sphinxAtStartPar
\sphinxstyleliteralstrong{\sphinxupquote{lambda0}} (\sphinxstyleliteralemphasis{\sphinxupquote{float}}\sphinxstyleliteralemphasis{\sphinxupquote{ or }}\sphinxstyleliteralemphasis{\sphinxupquote{np.ndarray}}) \textendash{} wavelength {[}nm{]}

\sphinxlineitem{Returns}
\sphinxAtStartPar
\begin{itemize}
\item {} 
\sphinxAtStartPar
\sphinxstylestrong{n} (\sphinxstyleemphasis{float or np.ndarray}) \textendash{} refractive index {[}\sphinxhyphen{}{]}

\end{itemize}


\end{description}\end{quote}

\end{fulllineitems}


\sphinxstepscope


\subsection{skinoptics.anisotropy\_factor module}
\label{\detokenize{06_anisotropy_factor:module-skinoptics.anisotropy_factor}}\label{\detokenize{06_anisotropy_factor:skinoptics-anisotropy-factor-module}}\label{\detokenize{06_anisotropy_factor::doc}}\index{module@\spxentry{module}!skinoptics.anisotropy\_factor@\spxentry{skinoptics.anisotropy\_factor}}\index{skinoptics.anisotropy\_factor@\spxentry{skinoptics.anisotropy\_factor}!module@\spxentry{module}}
\sphinxAtStartPar
Copyright (C) 2024 Victor Lima
\begin{quote}

\begin{DUlineblock}{0em}
\item[] This program is free software: you can redistribute it and/or modify
\item[] it under the terms of the GNU General Public License as published by
\item[] the Free Software Foundation, either version 3 of the License, or
\item[] (at your option) any later version.
\end{DUlineblock}

\begin{DUlineblock}{0em}
\item[] This program is distributed in the hope that it will be useful,
\item[] but WITHOUT ANY WARRANTY; without even the implied warranty of
\item[] MERCHANTABILITY or FITNESS FOR A PARTICULAR PURPOSE.  See the
\item[] GNU General Public License for more details.
\end{DUlineblock}

\begin{DUlineblock}{0em}
\item[] You should have received a copy of the GNU General Public License
\item[] along with this program.  If not, see \textless{}\sphinxurl{https://www.gnu.org/licenses/}\textgreater{}.
\end{DUlineblock}
\end{quote}

\begin{DUlineblock}{0em}
\item[] Victor Lima
\item[] victorporto@ifsc.usp.br
\item[] victor.lima@ufscar.br
\end{DUlineblock}

\begin{DUlineblock}{0em}
\item[] Release Date:
\item[] August 2024
\item[] Last Modification:
\item[] August 2024
\end{DUlineblock}

\begin{DUlineblock}{0em}
\item[] References:
\end{DUlineblock}

\begin{DUlineblock}{0em}
\item[] {[}HG41{]} Henyey \& Greenstein 1941.
\item[] Diffuse radiation in the Galaxy.
\item[] \sphinxurl{https://doi.org/10.1086/144246}
\end{DUlineblock}

\begin{DUlineblock}{0em}
\item[] {[}RM80{]} Reynolds \& McCormick 1980.
\item[] Approximate two\sphinxhyphen{}parameter phase function for light scattering.
\item[] \sphinxurl{https://doi.org/10.1364/JOSA.70.001206}
\end{DUlineblock}

\begin{DUlineblock}{0em}
\item[] {[}Bv84{]} Bruls \& van der Leun 1984.
\item[] Forward scattering properties of human epidermal layers.
\item[] \sphinxurl{https://doi.org/10.1111/j.1751-1097.1984.tb04581.x}
\end{DUlineblock}

\begin{DUlineblock}{0em}
\item[] {[}JAP87{]} Jacques, Alter \& Prahl 1987.
\item[] Angular Dependence of HeNe Laser Light Scattering by Human Dermis.
\item[] \sphinxurl{https://omlc.org/~prahl/pubs/pdfx/jacques87b.pdf}
\end{DUlineblock}

\begin{DUlineblock}{0em}
\item[] {[}Y*87{]} Yoon, Welch, Motamedi \& van Gemert 1987.
\item[] Development and Application of Three\sphinxhyphen{}Dimensional Light Distribution Model for Laser Irradiated Tissue.
\item[] \sphinxurl{https://doi.org/10.1109/JQE.1987.1073224}
\end{DUlineblock}

\begin{DUlineblock}{0em}
\item[] {[}v*89{]} van Gemert, Jacques, Sterenborg \& Star 1989.
\item[] Skin Optics.
\item[] \sphinxurl{https://doi.org/10.1109/10.42108}
\end{DUlineblock}

\begin{DUlineblock}{0em}
\item[] {[}CS92{]} Cornette \& Shanks 1992.
\item[] Physically reasonable analytic expression for the single\sphinxhyphen{}scattering phase function
\item[] \sphinxurl{https://doi.org/10.1364/AO.31.003152}
\end{DUlineblock}

\begin{DUlineblock}{0em}
\item[] {[}WJ92{]} Wang \& Jacques 1992.
\item[] Monte Carlo Modeling of Light Transport in Multi\sphinxhyphen{}layered Tissues in Standard C.
\item[] \sphinxurl{https://omlc.org/software/mc/mcml/MCman.pdf}
\end{DUlineblock}

\begin{DUlineblock}{0em}
\item[] {[}D03{]} Draine 2003.
\item[] Scattering by Interstellar Dust Grains. I. Optical and Ultraviolet.
\item[] \sphinxurl{https://doi.org/10.48550/arXiv.astro-ph/0304060}
\end{DUlineblock}

\begin{DUlineblock}{0em}
\item[] {[}F11{]} Frisvad 2011.
\item[] Importance sampling the Rayleigh phase function.
\item[] \sphinxurl{https://doi.org/10.1364/JOSAA.28.002436}
\end{DUlineblock}

\begin{DUlineblock}{0em}
\item[] {[}B*14{]} Bosschaart, Edelman, Aalders, van Leeuwen \& Faber 2014.
\item[] A literature review and novel theoretical approach on the optical properties of whole blood.
\item[] \sphinxurl{https://doi.org/10.1007/s10103-013-1446-7}
\end{DUlineblock}

\begin{DUlineblock}{0em}
\item[] {[}BCK22{]} Baes, Camps \& Kapoor 2022.
\item[] A new analytical scattering phase function for interstellar dust.
\item[] \sphinxurl{https://doi.org/10.1051/0004-6361/202142437}
\end{DUlineblock}

\begin{DUlineblock}{0em}
\item[] {[}JM23{]} Jacques \& McCormick 2023.
\item[] Two\sphinxhyphen{}term scattering phase function for photon transport to model subdiffuse reflectance
\item[] in superficial tissues.
\item[] \sphinxurl{https://doi.org/10.1364/BOE.476461}
\end{DUlineblock}
\index{costheta\_CS() (in module skinoptics.anisotropy\_factor)@\spxentry{costheta\_CS()}\spxextra{in module skinoptics.anisotropy\_factor}}

\begin{fulllineitems}
\phantomsection\label{\detokenize{06_anisotropy_factor:skinoptics.anisotropy_factor.costheta_CS}}
\pysigstartsignatures
\pysiglinewithargsret{\sphinxcode{\sphinxupquote{skinoptics.anisotropy\_factor.}}\sphinxbfcode{\sphinxupquote{costheta\_CS}}}{\sphinxparam{\DUrole{n}{g}}}{}
\pysigstopsignatures
\begin{DUlineblock}{0em}
\item[] The anisotropy factor as a function of the parameter g, assuming the Cornette\sphinxhyphen{}Shanks
\item[] scattering phase function.
\item[] For details please check Cornette \& Shanks 1992 {[}CS92{]}.
\end{DUlineblock}

\sphinxAtStartPar
\(\langle \cos\theta \rangle_{CS}(g) = g\frac{3(4 + g^2)}{5(2 + g^2)}\)
\begin{quote}\begin{description}
\sphinxlineitem{Parameters}
\sphinxAtStartPar
\sphinxstyleliteralstrong{\sphinxupquote{g}} (\sphinxstyleliteralemphasis{\sphinxupquote{float}}) \textendash{} parameter g {[}\sphinxhyphen{}{]} (must be in the range {[}\sphinxhyphen{}1, 1{]})

\sphinxlineitem{Returns}
\sphinxAtStartPar
\begin{itemize}
\item {} 
\sphinxAtStartPar
\sphinxstylestrong{costheta} (\sphinxstyleemphasis{np.ndarray}) \textendash{} anisotropy factor {[}\sphinxhyphen{}{]}

\end{itemize}


\end{description}\end{quote}

\end{fulllineitems}

\index{costheta\_D() (in module skinoptics.anisotropy\_factor)@\spxentry{costheta\_D()}\spxextra{in module skinoptics.anisotropy\_factor}}

\begin{fulllineitems}
\phantomsection\label{\detokenize{06_anisotropy_factor:skinoptics.anisotropy_factor.costheta_D}}
\pysigstartsignatures
\pysiglinewithargsret{\sphinxcode{\sphinxupquote{skinoptics.anisotropy\_factor.}}\sphinxbfcode{\sphinxupquote{costheta\_D}}}{\sphinxparam{\DUrole{n}{g}}\sphinxparamcomma \sphinxparam{\DUrole{n}{alpha}}}{}
\pysigstopsignatures
\begin{DUlineblock}{0em}
\item[] The anisotropy factor as a function of the parameters g and aplha, assuming the Draine
\item[] scattering phase function.
\item[] For details please check Draine 2003 {[}D03{]}.
\end{DUlineblock}

\sphinxAtStartPar
\(\langle \cos\theta \rangle_{D}(g, \alpha) = g\frac{1 + \alpha(3 + 2g^2)/5}{1 + \alpha(1 + 2g^2)/3}\)
\begin{quote}\begin{description}
\sphinxlineitem{Parameters}\begin{itemize}
\item {} 
\sphinxAtStartPar
\sphinxstyleliteralstrong{\sphinxupquote{g}} (\sphinxstyleliteralemphasis{\sphinxupquote{float}}) \textendash{} parameter g {[}\sphinxhyphen{}{]} (must be in the range {[}\sphinxhyphen{}1, 1{]})

\item {} 
\sphinxAtStartPar
\sphinxstyleliteralstrong{\sphinxupquote{alpha}} \textendash{} parameter alpha {[}\sphinxhyphen{}{]}

\end{itemize}

\sphinxlineitem{Float alpha}
\sphinxAtStartPar
float or np.ndarray

\sphinxlineitem{Returns}
\sphinxAtStartPar
\begin{itemize}
\item {} 
\sphinxAtStartPar
\sphinxstylestrong{costheta} (\sphinxstyleemphasis{np.ndarray}) \textendash{} anisotropy factor {[}\sphinxhyphen{}{]}

\end{itemize}


\end{description}\end{quote}

\end{fulllineitems}

\index{costheta\_HGIT() (in module skinoptics.anisotropy\_factor)@\spxentry{costheta\_HGIT()}\spxextra{in module skinoptics.anisotropy\_factor}}

\begin{fulllineitems}
\phantomsection\label{\detokenize{06_anisotropy_factor:skinoptics.anisotropy_factor.costheta_HGIT}}
\pysigstartsignatures
\pysiglinewithargsret{\sphinxcode{\sphinxupquote{skinoptics.anisotropy\_factor.}}\sphinxbfcode{\sphinxupquote{costheta\_HGIT}}}{\sphinxparam{\DUrole{n}{g}}\sphinxparamcomma \sphinxparam{\DUrole{n}{gamma}}}{}
\pysigstopsignatures
\begin{DUlineblock}{0em}
\item[] The anisotropy factor as a function of the parameters g and gamma, assuming the
\item[] Henyey\sphinxhyphen{}Greenstein scattering phase function with an isotropic term.
\item[] For details please check Jacques, Alter \& Prahl 1987 {[}JAP87{]} and Yoon et al. {[}Y*87{]}.
\end{DUlineblock}

\sphinxAtStartPar
\(\langle \cos\theta \rangle_{HGIT}(g, \gamma) = (1 - \gamma) g\)
\begin{quote}\begin{description}
\sphinxlineitem{Parameters}\begin{itemize}
\item {} 
\sphinxAtStartPar
\sphinxstyleliteralstrong{\sphinxupquote{g}} (\sphinxstyleliteralemphasis{\sphinxupquote{float}}) \textendash{} parameter g {[}\sphinxhyphen{}{]} (must be in the range {[}\sphinxhyphen{}1, 1{]})

\item {} 
\sphinxAtStartPar
\sphinxstyleliteralstrong{\sphinxupquote{gamma}} (\sphinxstyleliteralemphasis{\sphinxupquote{float}}) \textendash{} fraction of isotropic term contribution {[}\sphinxhyphen{}{]} (must be in the range {[}0, 1{]})

\end{itemize}

\sphinxlineitem{Returns}
\sphinxAtStartPar
\begin{itemize}
\item {} 
\sphinxAtStartPar
\sphinxstylestrong{costheta} (\sphinxstyleemphasis{np.ndarray}) \textendash{} anisotropy factor {[}\sphinxhyphen{}{]}

\end{itemize}


\end{description}\end{quote}

\end{fulllineitems}

\index{costheta\_RM() (in module skinoptics.anisotropy\_factor)@\spxentry{costheta\_RM()}\spxextra{in module skinoptics.anisotropy\_factor}}

\begin{fulllineitems}
\phantomsection\label{\detokenize{06_anisotropy_factor:skinoptics.anisotropy_factor.costheta_RM}}
\pysigstartsignatures
\pysiglinewithargsret{\sphinxcode{\sphinxupquote{skinoptics.anisotropy\_factor.}}\sphinxbfcode{\sphinxupquote{costheta\_RM}}}{\sphinxparam{\DUrole{n}{g}}\sphinxparamcomma \sphinxparam{\DUrole{n}{alpha}}}{}
\pysigstopsignatures
\begin{DUlineblock}{0em}
\item[] The anisotropy factor as a function of the parameters g and alpha, assuming the
\item[] Reynolds\sphinxhyphen{}McCormick scattering phase function.
\item[] For details please check Reynolds \& McCormick 1980 {[}RM80{]} and Jacques \& McCormick 2023 {[}JM23{]}.
\end{DUlineblock}

\begin{DUlineblock}{0em}
\item[] \(\langle \cos\theta \rangle_{RM}(g, \alpha) = \frac{2 \alpha g L - (1+g^2)}{2g(\alpha - 1)}\)
\item[] with
\item[] \(L = \frac{(1+g)^{2\alpha} + (1-g)^{2\alpha}}{(1+g)^{2\alpha} - (1-g)^{2\alpha}}\)
\end{DUlineblock}
\begin{quote}\begin{description}
\sphinxlineitem{Parameters}\begin{itemize}
\item {} 
\sphinxAtStartPar
\sphinxstyleliteralstrong{\sphinxupquote{g}} (\sphinxstyleliteralemphasis{\sphinxupquote{float}}) \textendash{} parameter g {[}\sphinxhyphen{}{]} (must be in the range {[}\sphinxhyphen{}1, 1{]})

\item {} 
\sphinxAtStartPar
\sphinxstyleliteralstrong{\sphinxupquote{alpha}} \textendash{} parameter alpha {[}\sphinxhyphen{}{]} (must be greater than \sphinxhyphen{}0.5)

\end{itemize}

\sphinxlineitem{Float alpha}
\sphinxAtStartPar
float or np.ndarray

\sphinxlineitem{Returns}
\sphinxAtStartPar
\begin{itemize}
\item {} 
\sphinxAtStartPar
\sphinxstylestrong{costheta} (\sphinxstyleemphasis{np.ndarray}) \textendash{} anisotropy factor {[}\sphinxhyphen{}{]}

\end{itemize}


\end{description}\end{quote}

\end{fulllineitems}

\index{costheta\_TTHG() (in module skinoptics.anisotropy\_factor)@\spxentry{costheta\_TTHG()}\spxextra{in module skinoptics.anisotropy\_factor}}

\begin{fulllineitems}
\phantomsection\label{\detokenize{06_anisotropy_factor:skinoptics.anisotropy_factor.costheta_TTHG}}
\pysigstartsignatures
\pysiglinewithargsret{\sphinxcode{\sphinxupquote{skinoptics.anisotropy\_factor.}}\sphinxbfcode{\sphinxupquote{costheta\_TTHG}}}{\sphinxparam{\DUrole{n}{g1}}\sphinxparamcomma \sphinxparam{\DUrole{n}{g2}}\sphinxparamcomma \sphinxparam{\DUrole{n}{gamma}}}{}
\pysigstopsignatures
\begin{DUlineblock}{0em}
\item[] The anisotropy factor as a function of the parameters g1, g2 and gamma, assuming the
\item[] two\sphinxhyphen{}term Henyey\sphinxhyphen{}Greenstein scattering phase function.
\item[] For details please check Baes, Camps \& Kapoor 2022 {[}BCK22{]}.
\end{DUlineblock}

\sphinxAtStartPar
\(\langle \cos\theta \rangle_{TTHG}(g_1, g_2, \gamma) = \gamma g_1 + (1 - \gamma) g_2\)

\sphinxAtStartPar
\(g_1\) characterises the shape and the strength of the forward scattering peak

\sphinxAtStartPar
\(g_2\) characterises the shape and the strength of the backward scattering peak
\begin{quote}\begin{description}
\sphinxlineitem{Parameters}\begin{itemize}
\item {} 
\sphinxAtStartPar
\sphinxstyleliteralstrong{\sphinxupquote{g1}} (\sphinxstyleliteralemphasis{\sphinxupquote{float}}) \textendash{} parameter g1 {[}\sphinxhyphen{}{]} (must be in the range {[}0, 1{]})

\item {} 
\sphinxAtStartPar
\sphinxstyleliteralstrong{\sphinxupquote{g2}} (\sphinxstyleliteralemphasis{\sphinxupquote{float}}) \textendash{} parameter g2 {[}\sphinxhyphen{}{]} (must be in the range {[}\sphinxhyphen{}1, 0{]})

\item {} 
\sphinxAtStartPar
\sphinxstyleliteralstrong{\sphinxupquote{gamma}} (\sphinxstyleliteralemphasis{\sphinxupquote{float}}) \textendash{} relative weight of the forward scattering component {[}\sphinxhyphen{}{]} (must be in the range {[}0, 1{]})

\end{itemize}

\sphinxlineitem{Returns}
\sphinxAtStartPar
\begin{itemize}
\item {} 
\sphinxAtStartPar
\sphinxstylestrong{costheta} (\sphinxstyleemphasis{np.ndarray}) \textendash{} anisotropy factor {[}\sphinxhyphen{}{]}

\end{itemize}


\end{description}\end{quote}

\end{fulllineitems}

\index{costheta\_TTRM() (in module skinoptics.anisotropy\_factor)@\spxentry{costheta\_TTRM()}\spxextra{in module skinoptics.anisotropy\_factor}}

\begin{fulllineitems}
\phantomsection\label{\detokenize{06_anisotropy_factor:skinoptics.anisotropy_factor.costheta_TTRM}}
\pysigstartsignatures
\pysiglinewithargsret{\sphinxcode{\sphinxupquote{skinoptics.anisotropy\_factor.}}\sphinxbfcode{\sphinxupquote{costheta\_TTRM}}}{\sphinxparam{\DUrole{n}{g1}}\sphinxparamcomma \sphinxparam{\DUrole{n}{g2}}\sphinxparamcomma \sphinxparam{\DUrole{n}{alpha1}}\sphinxparamcomma \sphinxparam{\DUrole{n}{alpha2}}\sphinxparamcomma \sphinxparam{\DUrole{n}{gamma}}}{}
\pysigstopsignatures
\begin{DUlineblock}{0em}
\item[] The anisotropy factor as a function of the parameters g1, g2, alpha1, alpha2 and gamma,
\item[] assuming the two\sphinxhyphen{}term Reynolds\sphinxhyphen{}McCormick scattering phase function.
\item[] For details please check Reynolds \& McCormick 1980 {[}RM80{]} and Jacques \& McCormick 2023 {[}JM23{]}.
\end{DUlineblock}

\sphinxAtStartPar
\(\langle \cos\theta \rangle_{TTRM}(g_1, g_2, \alpha_1, \alpha_2, \gamma) = \gamma \langle \cos\theta \rangle_{RM}(g_1, \alpha_1) + (1 - \gamma) \langle \cos\theta \rangle_{RM}(g_2, \alpha_2)\)

\sphinxAtStartPar
\(g_1\) characterises the shape and the strength of the forward scattering peak

\sphinxAtStartPar
\(g_2\) characterises the shape and the strength of the backward scattering peak
\begin{quote}\begin{description}
\sphinxlineitem{Parameters}\begin{itemize}
\item {} 
\sphinxAtStartPar
\sphinxstyleliteralstrong{\sphinxupquote{g1}} (\sphinxstyleliteralemphasis{\sphinxupquote{float}}) \textendash{} parameter g1 {[}\sphinxhyphen{}{]} (must be in the range {[}0, 1{]})

\item {} 
\sphinxAtStartPar
\sphinxstyleliteralstrong{\sphinxupquote{g2}} (\sphinxstyleliteralemphasis{\sphinxupquote{float}}) \textendash{} parameter g2 {[}\sphinxhyphen{}{]} (must be in the range {[}\sphinxhyphen{}1, 0{]})

\item {} 
\sphinxAtStartPar
\sphinxstyleliteralstrong{\sphinxupquote{alpha1}} (\sphinxstyleliteralemphasis{\sphinxupquote{float}}) \textendash{} parameter alpha1 {[}\sphinxhyphen{}{]} (must be greater than \sphinxhyphen{}0.5)

\item {} 
\sphinxAtStartPar
\sphinxstyleliteralstrong{\sphinxupquote{alpha2}} (\sphinxstyleliteralemphasis{\sphinxupquote{float}}) \textendash{} parameter alpha2 {[}\sphinxhyphen{}{]} (must be greater than \sphinxhyphen{}0.5)

\item {} 
\sphinxAtStartPar
\sphinxstyleliteralstrong{\sphinxupquote{gamma}} (\sphinxstyleliteralemphasis{\sphinxupquote{float}}) \textendash{} relative weight of the forward scattering component {[}\sphinxhyphen{}{]} (must be in the range {[}0, 1{]})

\end{itemize}

\sphinxlineitem{Returns}
\sphinxAtStartPar
\begin{itemize}
\item {} 
\sphinxAtStartPar
\sphinxstylestrong{costheta} (\sphinxstyleemphasis{np.ndarray}) \textendash{} anisotropy factor {[}\sphinxhyphen{}{]}

\end{itemize}


\end{description}\end{quote}

\end{fulllineitems}

\index{costheta\_TTU2() (in module skinoptics.anisotropy\_factor)@\spxentry{costheta\_TTU2()}\spxextra{in module skinoptics.anisotropy\_factor}}

\begin{fulllineitems}
\phantomsection\label{\detokenize{06_anisotropy_factor:skinoptics.anisotropy_factor.costheta_TTU2}}
\pysigstartsignatures
\pysiglinewithargsret{\sphinxcode{\sphinxupquote{skinoptics.anisotropy\_factor.}}\sphinxbfcode{\sphinxupquote{costheta\_TTU2}}}{\sphinxparam{\DUrole{n}{g1}}\sphinxparamcomma \sphinxparam{\DUrole{n}{g2}}\sphinxparamcomma \sphinxparam{\DUrole{n}{gamma}}}{}
\pysigstopsignatures
\begin{DUlineblock}{0em}
\item[] The anisotropy factor as a function of the parameters g1, g2 and gamma, assuming the
\item[] two\sphinxhyphen{}term Ultraspherical\sphinxhyphen{}2 scattering phase function.
\item[] For details please check Baes, Camps \& Kapoor 2022 {[}BCK22{]}.
\end{DUlineblock}

\sphinxAtStartPar
\(\langle \cos\theta \rangle_{TTU2}(g_1, g_2, \gamma) = \gamma \langle \cos\theta \rangle_{U2}(g_1) + (1 - \gamma) \langle \cos\theta \rangle_{U2}(g_2)\)

\sphinxAtStartPar
\(g_1\) characterises the shape and the strength of the forward scattering peak

\sphinxAtStartPar
\(g_2\) characterises the shape and the strength of the backward scattering peak
\begin{quote}\begin{description}
\sphinxlineitem{Parameters}\begin{itemize}
\item {} 
\sphinxAtStartPar
\sphinxstyleliteralstrong{\sphinxupquote{g1}} (\sphinxstyleliteralemphasis{\sphinxupquote{float}}) \textendash{} parameter g1 {[}\sphinxhyphen{}{]} (must be in the range {[}0, 1{]})

\item {} 
\sphinxAtStartPar
\sphinxstyleliteralstrong{\sphinxupquote{g2}} (\sphinxstyleliteralemphasis{\sphinxupquote{float}}) \textendash{} parameter g2 {[}\sphinxhyphen{}{]} (must be in the range {[}\sphinxhyphen{}1, 0{]})

\item {} 
\sphinxAtStartPar
\sphinxstyleliteralstrong{\sphinxupquote{gamma}} (\sphinxstyleliteralemphasis{\sphinxupquote{float}}) \textendash{} relative weight of the forward scattering component {[}\sphinxhyphen{}{]} (must be in the range {[}0, 1{]})

\end{itemize}

\sphinxlineitem{Returns}
\sphinxAtStartPar
\begin{itemize}
\item {} 
\sphinxAtStartPar
\sphinxstylestrong{costheta} (\sphinxstyleemphasis{np.ndarray}) \textendash{} anisotropy factor {[}\sphinxhyphen{}{]}

\end{itemize}


\end{description}\end{quote}

\end{fulllineitems}

\index{costheta\_U2() (in module skinoptics.anisotropy\_factor)@\spxentry{costheta\_U2()}\spxextra{in module skinoptics.anisotropy\_factor}}

\begin{fulllineitems}
\phantomsection\label{\detokenize{06_anisotropy_factor:skinoptics.anisotropy_factor.costheta_U2}}
\pysigstartsignatures
\pysiglinewithargsret{\sphinxcode{\sphinxupquote{skinoptics.anisotropy\_factor.}}\sphinxbfcode{\sphinxupquote{costheta\_U2}}}{\sphinxparam{\DUrole{n}{g}}}{}
\pysigstopsignatures
\begin{DUlineblock}{0em}
\item[] The anisotropy factor as a function of the parameter g, assuming the Ultraspherical\sphinxhyphen{}2
\item[] scattering phase function.
\item[] For details please check Baes, Camps \& Kapoor 2022 {[}BCK22{]}.
\end{DUlineblock}

\sphinxAtStartPar
\(\langle \cos\theta \rangle_{U2}(g) = \frac{1+g^2}{2g} + \left(\frac{1-g^2}{2g}\right)^2 ln\left(\frac{1-g}{1+g}\right)\)
\begin{quote}\begin{description}
\sphinxlineitem{Parameters}
\sphinxAtStartPar
\sphinxstyleliteralstrong{\sphinxupquote{g}} (\sphinxstyleliteralemphasis{\sphinxupquote{float}}) \textendash{} parameter g {[}\sphinxhyphen{}{]} (must be in the range {[}\sphinxhyphen{}1, 1{]})

\sphinxlineitem{Returns}
\sphinxAtStartPar
\begin{itemize}
\item {} 
\sphinxAtStartPar
\sphinxstylestrong{costheta} (\sphinxstyleemphasis{np.ndarray}) \textendash{} anisotropy factor {[}\sphinxhyphen{}{]}

\end{itemize}


\end{description}\end{quote}

\end{fulllineitems}

\index{g\_Bosschaart() (in module skinoptics.anisotropy\_factor)@\spxentry{g\_Bosschaart()}\spxextra{in module skinoptics.anisotropy\_factor}}

\begin{fulllineitems}
\phantomsection\label{\detokenize{06_anisotropy_factor:skinoptics.anisotropy_factor.g_Bosschaart}}
\pysigstartsignatures
\pysiglinewithargsret{\sphinxcode{\sphinxupquote{skinoptics.anisotropy\_factor.}}\sphinxbfcode{\sphinxupquote{g\_Bosschaart}}}{\sphinxparam{\DUrole{n}{lambda0}}}{}
\pysigstopsignatures
\begin{DUlineblock}{0em}
\item[] The anisotropy factor of the human OXYGENATED BLOOD as a function of wavelength.
\item[] Linear interpolation of experimental data compiled by Bosschaart et al. 2014 {[}B*14{]}.
\end{DUlineblock}

\sphinxAtStartPar
wavelength range: {[}251 nm, 1000 nm{]}
\begin{quote}\begin{description}
\sphinxlineitem{Parameters}
\sphinxAtStartPar
\sphinxstyleliteralstrong{\sphinxupquote{lambda0}} (\sphinxstyleliteralemphasis{\sphinxupquote{float}}\sphinxstyleliteralemphasis{\sphinxupquote{ or }}\sphinxstyleliteralemphasis{\sphinxupquote{np.ndarray}}) \textendash{} wavelength {[}nm{]}

\sphinxlineitem{Returns}
\sphinxAtStartPar
\begin{itemize}
\item {} 
\sphinxAtStartPar
\sphinxstylestrong{g} (\sphinxstyleemphasis{float or np.ndarray}) \textendash{} anisotropy factor {[}\sphinxhyphen{}{]}

\end{itemize}


\end{description}\end{quote}

\end{fulllineitems}

\index{g\_vanGemert() (in module skinoptics.anisotropy\_factor)@\spxentry{g\_vanGemert()}\spxextra{in module skinoptics.anisotropy\_factor}}

\begin{fulllineitems}
\phantomsection\label{\detokenize{06_anisotropy_factor:skinoptics.anisotropy_factor.g_vanGemert}}
\pysigstartsignatures
\pysiglinewithargsret{\sphinxcode{\sphinxupquote{skinoptics.anisotropy\_factor.}}\sphinxbfcode{\sphinxupquote{g\_vanGemert}}}{\sphinxparam{\DUrole{n}{lambda0}}}{}
\pysigstopsignatures
\begin{DUlineblock}{0em}
\item[] The anisotropy factor of the human EPIDERMIS or DERMIS as a function of wavelength.
\item[] van Gemert et al. 1989 {[}v*89{]}’s fit for experimental data from Bruls \& van der Leun 1984
\item[] {[}Bv84{]} (epidermis, {[}302, 365, 436, 546{]} nm) and Jacques, Alter \& Prahl 1987 {[}JAP87{]}
\item[] (dermis, 633 nm).
\end{DUlineblock}

\sphinxAtStartPar
\(g(\lambda) = 0.29 \times 10^{-3} \lambda + 0.62\)

\begin{DUlineblock}{0em}
\item[] wavelength range: {[}302 nm, 633 nm{]}
\end{DUlineblock}
\begin{quote}\begin{description}
\sphinxlineitem{Parameters}
\sphinxAtStartPar
\sphinxstyleliteralstrong{\sphinxupquote{lambda0}} (\sphinxstyleliteralemphasis{\sphinxupquote{float}}\sphinxstyleliteralemphasis{\sphinxupquote{ or }}\sphinxstyleliteralemphasis{\sphinxupquote{np.ndarray}}) \textendash{} wavelength {[}nm{]}

\sphinxlineitem{Returns}
\sphinxAtStartPar
\begin{itemize}
\item {} 
\sphinxAtStartPar
\sphinxstylestrong{g} (\sphinxstyleemphasis{float or np.ndarray}) \textendash{} anisotropy factor {[}\sphinxhyphen{}{]}

\end{itemize}


\end{description}\end{quote}

\end{fulllineitems}

\index{ptheta\_CS() (in module skinoptics.anisotropy\_factor)@\spxentry{ptheta\_CS()}\spxextra{in module skinoptics.anisotropy\_factor}}

\begin{fulllineitems}
\phantomsection\label{\detokenize{06_anisotropy_factor:skinoptics.anisotropy_factor.ptheta_CS}}
\pysigstartsignatures
\pysiglinewithargsret{\sphinxcode{\sphinxupquote{skinoptics.anisotropy\_factor.}}\sphinxbfcode{\sphinxupquote{ptheta\_CS}}}{\sphinxparam{\DUrole{n}{theta}}\sphinxparamcomma \sphinxparam{\DUrole{n}{g}}}{}
\pysigstopsignatures
\begin{DUlineblock}{0em}
\item[] The Cornette\sphinxhyphen{}Shanks scattering phase function.
\item[] For details please check Cornette \& Shanks 1992 {[}CS92{]}.
\end{DUlineblock}

\sphinxAtStartPar
\(p_{CS}(\theta, g) = \frac{3}{2}\frac{1 + \cos^2\theta}{2 + g^2)} p_{HG}(\theta, g)\)

\begin{DUlineblock}{0em}
\item[] In this model \(g\) is NOT the anisotropy factor.
\end{DUlineblock}
\begin{quote}\begin{description}
\sphinxlineitem{Parameters}\begin{itemize}
\item {} 
\sphinxAtStartPar
\sphinxstyleliteralstrong{\sphinxupquote{theta}} (\sphinxstyleliteralemphasis{\sphinxupquote{float}}\sphinxstyleliteralemphasis{\sphinxupquote{ or }}\sphinxstyleliteralemphasis{\sphinxupquote{np.ndarray}}) \textendash{} scattering angle {[}degrees{]}

\item {} 
\sphinxAtStartPar
\sphinxstyleliteralstrong{\sphinxupquote{g}} (\sphinxstyleliteralemphasis{\sphinxupquote{float}}) \textendash{} parameter g {[}\sphinxhyphen{}{]} (must be in the range {[}\sphinxhyphen{}1, 1{]})

\end{itemize}

\sphinxlineitem{Returns}
\sphinxAtStartPar
\begin{itemize}
\item {} 
\sphinxAtStartPar
\sphinxstylestrong{ptheta} (\sphinxstyleemphasis{float or np.ndarray}) \textendash{} scattering phase function {[}\sphinxhyphen{}{]}

\end{itemize}


\end{description}\end{quote}

\end{fulllineitems}

\index{ptheta\_D() (in module skinoptics.anisotropy\_factor)@\spxentry{ptheta\_D()}\spxextra{in module skinoptics.anisotropy\_factor}}

\begin{fulllineitems}
\phantomsection\label{\detokenize{06_anisotropy_factor:skinoptics.anisotropy_factor.ptheta_D}}
\pysigstartsignatures
\pysiglinewithargsret{\sphinxcode{\sphinxupquote{skinoptics.anisotropy\_factor.}}\sphinxbfcode{\sphinxupquote{ptheta\_D}}}{\sphinxparam{\DUrole{n}{theta}}\sphinxparamcomma \sphinxparam{\DUrole{n}{g}}\sphinxparamcomma \sphinxparam{\DUrole{n}{alpha}}}{}
\pysigstopsignatures
\begin{DUlineblock}{0em}
\item[] The Draine scattering phase function.
\item[] For details please check Draine 2003 {[}D03{]}.
\end{DUlineblock}

\sphinxAtStartPar
\(p_{D}(\theta, g, \alpha) = 3\frac{1 + \alpha \cos^2\theta}{3 + \alpha (1 + 2g^2)} p_{HG}(\theta, g)\)

\begin{DUlineblock}{0em}
\item[] For \(\alpha = 1\) and \(g = 0\) it reduces to the Rayleigh phase function.
\item[] For \(\alpha = 0\) it reduces to the Henyey\sphinxhyphen{}Greenstein scattering phase function.
\item[] For \(\alpha = 1\) it reduces to the Cornette\sphinxhyphen{}Shanks scattering phase function
\end{DUlineblock}

\begin{DUlineblock}{0em}
\item[] In this model \(g\) is NOT the anisotropy factor.
\end{DUlineblock}
\begin{quote}\begin{description}
\sphinxlineitem{Parameters}\begin{itemize}
\item {} 
\sphinxAtStartPar
\sphinxstyleliteralstrong{\sphinxupquote{theta}} (\sphinxstyleliteralemphasis{\sphinxupquote{float}}\sphinxstyleliteralemphasis{\sphinxupquote{ or }}\sphinxstyleliteralemphasis{\sphinxupquote{np.ndarray}}) \textendash{} scattering angle {[}degrees{]}

\item {} 
\sphinxAtStartPar
\sphinxstyleliteralstrong{\sphinxupquote{g}} (\sphinxstyleliteralemphasis{\sphinxupquote{float}}) \textendash{} parameter g {[}\sphinxhyphen{}{]} (must be in the range {[}\sphinxhyphen{}1, 1{]})

\item {} 
\sphinxAtStartPar
\sphinxstyleliteralstrong{\sphinxupquote{alpha}} (\sphinxstyleliteralemphasis{\sphinxupquote{float}}) \textendash{} parameter alpha {[}\sphinxhyphen{}{]}

\end{itemize}

\sphinxlineitem{Returns}
\sphinxAtStartPar
\begin{itemize}
\item {} 
\sphinxAtStartPar
\sphinxstylestrong{ptheta} (\sphinxstyleemphasis{float or np.ndarray}) \textendash{} scattering phase function {[}\sphinxhyphen{}{]}

\end{itemize}


\end{description}\end{quote}

\end{fulllineitems}

\index{ptheta\_HG() (in module skinoptics.anisotropy\_factor)@\spxentry{ptheta\_HG()}\spxextra{in module skinoptics.anisotropy\_factor}}

\begin{fulllineitems}
\phantomsection\label{\detokenize{06_anisotropy_factor:skinoptics.anisotropy_factor.ptheta_HG}}
\pysigstartsignatures
\pysiglinewithargsret{\sphinxcode{\sphinxupquote{skinoptics.anisotropy\_factor.}}\sphinxbfcode{\sphinxupquote{ptheta\_HG}}}{\sphinxparam{\DUrole{n}{theta}}\sphinxparamcomma \sphinxparam{\DUrole{n}{g}}}{}
\pysigstopsignatures
\begin{DUlineblock}{0em}
\item[] The Henyey\sphinxhyphen{}Greenstein scattering phase function.
\item[] For details please check Henyey \& Greenstein 1941 {[}HG41{]}.
\end{DUlineblock}

\sphinxAtStartPar
\(p_{HG}(\theta, g) = \frac{1}{2}\frac{1 - g^2}{(1 + g^2 - 2g \cos(\theta))^{3/2}}\)

\sphinxAtStartPar
In this particular model \(g\) is the anisotropy factor.
\begin{quote}\begin{description}
\sphinxlineitem{Parameters}\begin{itemize}
\item {} 
\sphinxAtStartPar
\sphinxstyleliteralstrong{\sphinxupquote{theta}} (\sphinxstyleliteralemphasis{\sphinxupquote{float}}\sphinxstyleliteralemphasis{\sphinxupquote{ or }}\sphinxstyleliteralemphasis{\sphinxupquote{np.ndarray}}) \textendash{} scattering angle {[}degrees{]}

\item {} 
\sphinxAtStartPar
\sphinxstyleliteralstrong{\sphinxupquote{g}} (\sphinxstyleliteralemphasis{\sphinxupquote{float}}) \textendash{} parameter g {[}\sphinxhyphen{}{]} (must be in the range {[}\sphinxhyphen{}1, 1{]})

\end{itemize}

\sphinxlineitem{Returns}
\sphinxAtStartPar
\begin{itemize}
\item {} 
\sphinxAtStartPar
\sphinxstylestrong{ptheta} (\sphinxstyleemphasis{float or np.ndarray}) \textendash{} scattering phase function {[}\sphinxhyphen{}{]}

\end{itemize}


\end{description}\end{quote}

\end{fulllineitems}

\index{ptheta\_HGIT() (in module skinoptics.anisotropy\_factor)@\spxentry{ptheta\_HGIT()}\spxextra{in module skinoptics.anisotropy\_factor}}

\begin{fulllineitems}
\phantomsection\label{\detokenize{06_anisotropy_factor:skinoptics.anisotropy_factor.ptheta_HGIT}}
\pysigstartsignatures
\pysiglinewithargsret{\sphinxcode{\sphinxupquote{skinoptics.anisotropy\_factor.}}\sphinxbfcode{\sphinxupquote{ptheta\_HGIT}}}{\sphinxparam{\DUrole{n}{theta}}\sphinxparamcomma \sphinxparam{\DUrole{n}{g}}\sphinxparamcomma \sphinxparam{\DUrole{n}{gamma}}}{}
\pysigstopsignatures
\begin{DUlineblock}{0em}
\item[] The Henyey\sphinxhyphen{}Greenstein scattering phase function with an isotropic term.
\item[] For details please check Jacques, Alter \& Prahl 1987 {[}JAP87{]} and Yoon et al. {[}Y*87{]}.
\end{DUlineblock}

\sphinxAtStartPar
\(p_{HGIT}(\theta, g, \gamma) = \frac{1}{2}\gamma + (1-\gamma) p_{HG}(\theta, g)\)

\sphinxAtStartPar
In this model \(g\) is NOT the anisotropy factor.
\begin{quote}\begin{description}
\sphinxlineitem{Parameters}\begin{itemize}
\item {} 
\sphinxAtStartPar
\sphinxstyleliteralstrong{\sphinxupquote{theta}} (\sphinxstyleliteralemphasis{\sphinxupquote{float}}\sphinxstyleliteralemphasis{\sphinxupquote{ or }}\sphinxstyleliteralemphasis{\sphinxupquote{np.ndarray}}) \textendash{} scattering angle {[}degrees{]}

\item {} 
\sphinxAtStartPar
\sphinxstyleliteralstrong{\sphinxupquote{g}} (\sphinxstyleliteralemphasis{\sphinxupquote{float}}) \textendash{} parameter g {[}\sphinxhyphen{}{]} (must be in the range {[}\sphinxhyphen{}1, 1{]})

\item {} 
\sphinxAtStartPar
\sphinxstyleliteralstrong{\sphinxupquote{gamma}} (\sphinxstyleliteralemphasis{\sphinxupquote{float}}) \textendash{} relative weight of the isotropic term component {[}\sphinxhyphen{}{]} (must be in the range {[}0, 1{]})

\end{itemize}

\sphinxlineitem{Returns}
\sphinxAtStartPar
\begin{itemize}
\item {} 
\sphinxAtStartPar
\sphinxstylestrong{ptheta} (\sphinxstyleemphasis{float or np.ndarray}) \textendash{} scattering phase function {[}\sphinxhyphen{}{]}

\end{itemize}


\end{description}\end{quote}

\end{fulllineitems}

\index{ptheta\_R() (in module skinoptics.anisotropy\_factor)@\spxentry{ptheta\_R()}\spxextra{in module skinoptics.anisotropy\_factor}}

\begin{fulllineitems}
\phantomsection\label{\detokenize{06_anisotropy_factor:skinoptics.anisotropy_factor.ptheta_R}}
\pysigstartsignatures
\pysiglinewithargsret{\sphinxcode{\sphinxupquote{skinoptics.anisotropy\_factor.}}\sphinxbfcode{\sphinxupquote{ptheta\_R}}}{\sphinxparam{\DUrole{n}{theta}}}{}
\pysigstopsignatures
\begin{DUlineblock}{0em}
\item[] The Rayleigh scattering phase function.
\item[] For details please check Frisvad 2011 {[}F11{]}.
\end{DUlineblock}

\sphinxAtStartPar
\(p_{R}(\theta) = \frac{3}{8}(1 + \cos^2\theta)\)
\begin{quote}\begin{description}
\sphinxlineitem{Parameters}
\sphinxAtStartPar
\sphinxstyleliteralstrong{\sphinxupquote{theta}} (\sphinxstyleliteralemphasis{\sphinxupquote{float}}\sphinxstyleliteralemphasis{\sphinxupquote{ or }}\sphinxstyleliteralemphasis{\sphinxupquote{np.ndarray}}) \textendash{} scattering angle {[}degrees{]}

\sphinxlineitem{Returns}
\sphinxAtStartPar
\begin{itemize}
\item {} 
\sphinxAtStartPar
\sphinxstylestrong{ptheta} (\sphinxstyleemphasis{float or np.ndarray}) \textendash{} scattering phase function {[}\sphinxhyphen{}{]}

\end{itemize}


\end{description}\end{quote}

\end{fulllineitems}

\index{ptheta\_RM() (in module skinoptics.anisotropy\_factor)@\spxentry{ptheta\_RM()}\spxextra{in module skinoptics.anisotropy\_factor}}

\begin{fulllineitems}
\phantomsection\label{\detokenize{06_anisotropy_factor:skinoptics.anisotropy_factor.ptheta_RM}}
\pysigstartsignatures
\pysiglinewithargsret{\sphinxcode{\sphinxupquote{skinoptics.anisotropy\_factor.}}\sphinxbfcode{\sphinxupquote{ptheta\_RM}}}{\sphinxparam{\DUrole{n}{theta}}\sphinxparamcomma \sphinxparam{\DUrole{n}{g}}\sphinxparamcomma \sphinxparam{\DUrole{n}{alpha}}}{}
\pysigstopsignatures
\begin{DUlineblock}{0em}
\item[] The Reynolds\sphinxhyphen{}McCormick scattering phase function.
\item[] For details please check Reynolds \& McCormick 1980 {[}RM80{]} and Jacques \& McCormick 2023 {[}JM23{]}.
\end{DUlineblock}

\sphinxAtStartPar
\(p_{RM}(\theta, g, \alpha) = 2 \frac{\alpha g}{(1 + g)^{2\alpha} - (1 - g)^{2\alpha}}\frac{(1 - g^2)^{2\alpha}}{(1 + g^2 - 2g\cos\theta)^{\alpha + 1}}\)

\begin{DUlineblock}{0em}
\item[] For \(\alpha = 1/2\) it reduces to the Henyey\sphinxhyphen{}Greenstein scattering phase function.
\item[] For \(\alpha = 1\) it reduces to the Ultraspherical\sphinxhyphen{}2 scattering phase function.
\end{DUlineblock}

\begin{DUlineblock}{0em}
\item[] In this model \(g\) is the anisotropy factor only when \(\alpha = 1/2\).
\end{DUlineblock}
\begin{quote}\begin{description}
\sphinxlineitem{Parameters}\begin{itemize}
\item {} 
\sphinxAtStartPar
\sphinxstyleliteralstrong{\sphinxupquote{theta}} (\sphinxstyleliteralemphasis{\sphinxupquote{float}}\sphinxstyleliteralemphasis{\sphinxupquote{ or }}\sphinxstyleliteralemphasis{\sphinxupquote{np.ndarray}}) \textendash{} scattering angle {[}degrees{]}

\item {} 
\sphinxAtStartPar
\sphinxstyleliteralstrong{\sphinxupquote{g}} (\sphinxstyleliteralemphasis{\sphinxupquote{float}}) \textendash{} parameter g {[}\sphinxhyphen{}{]} (must be in the range {[}\sphinxhyphen{}1, 1{]})

\item {} 
\sphinxAtStartPar
\sphinxstyleliteralstrong{\sphinxupquote{alpha}} (\sphinxstyleliteralemphasis{\sphinxupquote{float}}) \textendash{} parameter alpha {[}\sphinxhyphen{}{]} (must be greater than \sphinxhyphen{}0.5)

\end{itemize}

\sphinxlineitem{Returns}
\sphinxAtStartPar
\begin{itemize}
\item {} 
\sphinxAtStartPar
\sphinxstylestrong{ptheta} (\sphinxstyleemphasis{float or np.ndarray}) \textendash{} scattering phase function {[}\sphinxhyphen{}{]}

\end{itemize}


\end{description}\end{quote}

\end{fulllineitems}

\index{ptheta\_TTHG() (in module skinoptics.anisotropy\_factor)@\spxentry{ptheta\_TTHG()}\spxextra{in module skinoptics.anisotropy\_factor}}

\begin{fulllineitems}
\phantomsection\label{\detokenize{06_anisotropy_factor:skinoptics.anisotropy_factor.ptheta_TTHG}}
\pysigstartsignatures
\pysiglinewithargsret{\sphinxcode{\sphinxupquote{skinoptics.anisotropy\_factor.}}\sphinxbfcode{\sphinxupquote{ptheta\_TTHG}}}{\sphinxparam{\DUrole{n}{theta}}\sphinxparamcomma \sphinxparam{\DUrole{n}{g1}}\sphinxparamcomma \sphinxparam{\DUrole{n}{g2}}\sphinxparamcomma \sphinxparam{\DUrole{n}{gamma}}}{}
\pysigstopsignatures
\begin{DUlineblock}{0em}
\item[] The two\sphinxhyphen{}term Henyey\sphinxhyphen{}Greenstein scattering phase function.
\item[] For details please check Baes, Camps \& Kapoor 2022 {[}BCK22{]}.
\end{DUlineblock}

\sphinxAtStartPar
\(p_{TTHG}(\theta, g_1, g_2, \gamma) = \gamma p_{HG}(\theta, g_1) + (1 - \gamma) p_{HG}(\theta, g_2)\)

\sphinxAtStartPar
\(g_1\) characterises the shape and the strength of the forward scattering peak

\sphinxAtStartPar
\(g_2\) characterises the shape and the strength of the backward scattering peak
\begin{quote}\begin{description}
\sphinxlineitem{Parameters}\begin{itemize}
\item {} 
\sphinxAtStartPar
\sphinxstyleliteralstrong{\sphinxupquote{theta}} (\sphinxstyleliteralemphasis{\sphinxupquote{float}}\sphinxstyleliteralemphasis{\sphinxupquote{ or }}\sphinxstyleliteralemphasis{\sphinxupquote{np.ndarray}}) \textendash{} scattering angle {[}degrees{]}

\item {} 
\sphinxAtStartPar
\sphinxstyleliteralstrong{\sphinxupquote{g1}} (\sphinxstyleliteralemphasis{\sphinxupquote{float}}) \textendash{} parameter g1 {[}\sphinxhyphen{}{]} (must be in the range {[}0, 1{]})

\item {} 
\sphinxAtStartPar
\sphinxstyleliteralstrong{\sphinxupquote{g2}} (\sphinxstyleliteralemphasis{\sphinxupquote{float}}) \textendash{} parameter g2 {[}\sphinxhyphen{}{]} (must be in the range {[}\sphinxhyphen{}1, 0{]})

\item {} 
\sphinxAtStartPar
\sphinxstyleliteralstrong{\sphinxupquote{gamma}} (\sphinxstyleliteralemphasis{\sphinxupquote{float}}) \textendash{} relative weight of the forward scattering component {[}\sphinxhyphen{}{]} (must be in the range {[}0, 1{]})

\end{itemize}

\sphinxlineitem{Returns}
\sphinxAtStartPar
\begin{itemize}
\item {} 
\sphinxAtStartPar
\sphinxstylestrong{ptheta} (\sphinxstyleemphasis{float or np.ndarray}) \textendash{} scattering phase function {[}\sphinxhyphen{}{]}

\end{itemize}


\end{description}\end{quote}

\end{fulllineitems}

\index{ptheta\_TTRM() (in module skinoptics.anisotropy\_factor)@\spxentry{ptheta\_TTRM()}\spxextra{in module skinoptics.anisotropy\_factor}}

\begin{fulllineitems}
\phantomsection\label{\detokenize{06_anisotropy_factor:skinoptics.anisotropy_factor.ptheta_TTRM}}
\pysigstartsignatures
\pysiglinewithargsret{\sphinxcode{\sphinxupquote{skinoptics.anisotropy\_factor.}}\sphinxbfcode{\sphinxupquote{ptheta\_TTRM}}}{\sphinxparam{\DUrole{n}{theta}}\sphinxparamcomma \sphinxparam{\DUrole{n}{g1}}\sphinxparamcomma \sphinxparam{\DUrole{n}{g2}}\sphinxparamcomma \sphinxparam{\DUrole{n}{alpha1}}\sphinxparamcomma \sphinxparam{\DUrole{n}{alpha2}}\sphinxparamcomma \sphinxparam{\DUrole{n}{gamma}}}{}
\pysigstopsignatures
\begin{DUlineblock}{0em}
\item[] The two\sphinxhyphen{}term Reynolds\sphinxhyphen{}McCormick scattering phase function.
\item[] For details please check  Reynolds \& McCormick 1980 {[}RM80{]} and Jacques \& McCormick 2023 {[}JM23{]}.
\end{DUlineblock}

\sphinxAtStartPar
\(p_{TTRM}(\theta, g_1, g_2, \alpha_1, \alpha_2, \gamma) =  \gamma p_{RM}(\theta, g_1, \alpha_1) + (1 - \gamma) p_{RM}(\theta, g_2, \alpha_2)\)

\sphinxAtStartPar
\(g_1\) characterises the shape and the strength of the forward scattering peak

\sphinxAtStartPar
\(g_2\) characterises the shape and the strength of the backward scattering peak
\begin{quote}\begin{description}
\sphinxlineitem{Parameters}\begin{itemize}
\item {} 
\sphinxAtStartPar
\sphinxstyleliteralstrong{\sphinxupquote{theta}} (\sphinxstyleliteralemphasis{\sphinxupquote{float}}\sphinxstyleliteralemphasis{\sphinxupquote{ or }}\sphinxstyleliteralemphasis{\sphinxupquote{np.ndarray}}) \textendash{} scattering angle {[}degrees{]}

\item {} 
\sphinxAtStartPar
\sphinxstyleliteralstrong{\sphinxupquote{g1}} (\sphinxstyleliteralemphasis{\sphinxupquote{float}}) \textendash{} parameter g1 {[}\sphinxhyphen{}{]} (must be in the range {[}0, 1{]})

\item {} 
\sphinxAtStartPar
\sphinxstyleliteralstrong{\sphinxupquote{g2}} (\sphinxstyleliteralemphasis{\sphinxupquote{float}}) \textendash{} parameter g2 {[}\sphinxhyphen{}{]} (must be in the range {[}\sphinxhyphen{}1, 0{]})

\item {} 
\sphinxAtStartPar
\sphinxstyleliteralstrong{\sphinxupquote{alpha1}} (\sphinxstyleliteralemphasis{\sphinxupquote{float}}) \textendash{} parameter alpha1 {[}\sphinxhyphen{}{]} (must be greater than \sphinxhyphen{}1/2)

\item {} 
\sphinxAtStartPar
\sphinxstyleliteralstrong{\sphinxupquote{alpha2}} (\sphinxstyleliteralemphasis{\sphinxupquote{float}}) \textendash{} parameter alpha2 {[}\sphinxhyphen{}{]} (must be greater than \sphinxhyphen{}1/2)

\item {} 
\sphinxAtStartPar
\sphinxstyleliteralstrong{\sphinxupquote{gamma}} (\sphinxstyleliteralemphasis{\sphinxupquote{float}}) \textendash{} relative weight of the forward scattering component {[}\sphinxhyphen{}{]} (must be in the range {[}0, 1{]})

\end{itemize}

\sphinxlineitem{Returns}
\sphinxAtStartPar
\begin{itemize}
\item {} 
\sphinxAtStartPar
\sphinxstylestrong{ptheta} (\sphinxstyleemphasis{float or np.ndarray}) \textendash{} scattering phase function {[}\sphinxhyphen{}{]}

\end{itemize}


\end{description}\end{quote}

\end{fulllineitems}

\index{ptheta\_TTU2() (in module skinoptics.anisotropy\_factor)@\spxentry{ptheta\_TTU2()}\spxextra{in module skinoptics.anisotropy\_factor}}

\begin{fulllineitems}
\phantomsection\label{\detokenize{06_anisotropy_factor:skinoptics.anisotropy_factor.ptheta_TTU2}}
\pysigstartsignatures
\pysiglinewithargsret{\sphinxcode{\sphinxupquote{skinoptics.anisotropy\_factor.}}\sphinxbfcode{\sphinxupquote{ptheta\_TTU2}}}{\sphinxparam{\DUrole{n}{theta}}\sphinxparamcomma \sphinxparam{\DUrole{n}{g1}}\sphinxparamcomma \sphinxparam{\DUrole{n}{g2}}\sphinxparamcomma \sphinxparam{\DUrole{n}{gamma}}}{}
\pysigstopsignatures
\begin{DUlineblock}{0em}
\item[] The two\sphinxhyphen{}term Ultraspherical\sphinxhyphen{}2 scattering phase function.
\item[] For details please check Baes, Camps \& Kapoor 2022 {[}BCK22{]}.
\end{DUlineblock}

\sphinxAtStartPar
\(p_{TTU2}(\theta, g_1, g_2, \gamma) = \gamma p_{U2}(\theta, g_1) + (1 - \gamma) p_{U2}(\theta, g_2)\)

\sphinxAtStartPar
\(g_1\) characterises the shape and the strength of the forward scattering peak

\sphinxAtStartPar
\(g_2\) characterises the shape and the strength of the backward scattering peak
\begin{quote}\begin{description}
\sphinxlineitem{Parameters}\begin{itemize}
\item {} 
\sphinxAtStartPar
\sphinxstyleliteralstrong{\sphinxupquote{theta}} (\sphinxstyleliteralemphasis{\sphinxupquote{float}}\sphinxstyleliteralemphasis{\sphinxupquote{ or }}\sphinxstyleliteralemphasis{\sphinxupquote{np.ndarray}}) \textendash{} scattering angle {[}degrees{]}

\item {} 
\sphinxAtStartPar
\sphinxstyleliteralstrong{\sphinxupquote{g1}} (\sphinxstyleliteralemphasis{\sphinxupquote{float}}) \textendash{} parameter g1 {[}\sphinxhyphen{}{]} (must be in the range {[}0, 1{]})

\item {} 
\sphinxAtStartPar
\sphinxstyleliteralstrong{\sphinxupquote{g2}} (\sphinxstyleliteralemphasis{\sphinxupquote{float}}) \textendash{} parameter g2 {[}\sphinxhyphen{}{]} (must be in the range {[}\sphinxhyphen{}1, 0{]})

\item {} 
\sphinxAtStartPar
\sphinxstyleliteralstrong{\sphinxupquote{gamma}} (\sphinxstyleliteralemphasis{\sphinxupquote{float}}) \textendash{} relative weight of the forward scattering component {[}\sphinxhyphen{}{]} (must be in the range {[}0, 1{]})

\end{itemize}

\sphinxlineitem{Returns}
\sphinxAtStartPar
\begin{itemize}
\item {} 
\sphinxAtStartPar
\sphinxstylestrong{ptheta} (\sphinxstyleemphasis{float or np.ndarray}) \textendash{} scattering phase function {[}\sphinxhyphen{}{]}

\end{itemize}


\end{description}\end{quote}

\end{fulllineitems}

\index{ptheta\_U2() (in module skinoptics.anisotropy\_factor)@\spxentry{ptheta\_U2()}\spxextra{in module skinoptics.anisotropy\_factor}}

\begin{fulllineitems}
\phantomsection\label{\detokenize{06_anisotropy_factor:skinoptics.anisotropy_factor.ptheta_U2}}
\pysigstartsignatures
\pysiglinewithargsret{\sphinxcode{\sphinxupquote{skinoptics.anisotropy\_factor.}}\sphinxbfcode{\sphinxupquote{ptheta\_U2}}}{\sphinxparam{\DUrole{n}{theta}}\sphinxparamcomma \sphinxparam{\DUrole{n}{g}}}{}
\pysigstopsignatures
\begin{DUlineblock}{0em}
\item[] The Ultraspherical\sphinxhyphen{}2 scattering phase function.
\item[] For details please check Baes, Camps \& Kapoor 2022 {[}BCK22{]}.
\end{DUlineblock}

\sphinxAtStartPar
\(p_{U2}(\theta, g) = \frac{1}{2}\frac{(1 - g^2)^2}{(1 + g^2 - 2g \cos \theta)^2}\)

\begin{DUlineblock}{0em}
\item[] In this model \(g\) is NOT the anisotropy factor.
\end{DUlineblock}
\begin{quote}\begin{description}
\sphinxlineitem{Parameters}\begin{itemize}
\item {} 
\sphinxAtStartPar
\sphinxstyleliteralstrong{\sphinxupquote{theta}} (\sphinxstyleliteralemphasis{\sphinxupquote{float}}\sphinxstyleliteralemphasis{\sphinxupquote{ or }}\sphinxstyleliteralemphasis{\sphinxupquote{np.ndarray}}) \textendash{} scattering angle {[}degrees{]}

\item {} 
\sphinxAtStartPar
\sphinxstyleliteralstrong{\sphinxupquote{g}} (\sphinxstyleliteralemphasis{\sphinxupquote{float}}) \textendash{} parameter g {[}\sphinxhyphen{}{]} (must be in the range {[}\sphinxhyphen{}1, 1{]})

\end{itemize}

\sphinxlineitem{Returns}
\sphinxAtStartPar
\begin{itemize}
\item {} 
\sphinxAtStartPar
\sphinxstylestrong{ptheta} (\sphinxstyleemphasis{float or np.ndarray}) \textendash{} scattering phase function {[}\sphinxhyphen{}{]}

\end{itemize}


\end{description}\end{quote}

\end{fulllineitems}

\index{theta\_HG\_from\_RND() (in module skinoptics.anisotropy\_factor)@\spxentry{theta\_HG\_from\_RND()}\spxextra{in module skinoptics.anisotropy\_factor}}

\begin{fulllineitems}
\phantomsection\label{\detokenize{06_anisotropy_factor:skinoptics.anisotropy_factor.theta_HG_from_RND}}
\pysigstartsignatures
\pysiglinewithargsret{\sphinxcode{\sphinxupquote{skinoptics.anisotropy\_factor.}}\sphinxbfcode{\sphinxupquote{theta\_HG\_from\_RND}}}{\sphinxparam{\DUrole{n}{g}}\sphinxparamcomma \sphinxparam{\DUrole{n}{n\_RND}\DUrole{o}{=}\DUrole{default_value}{1000000}}}{}
\pysigstopsignatures
\begin{DUlineblock}{0em}
\item[] The scattering angle distribution as a function of the anisotropy factor and a set of random
\item[] numbers uniformly distributed over the interval {[}0, 1), assuming the Henyey\sphinxhyphen{}Greenstein
\item[] scattering phase function.
\item[] For details please check section 3.5 from Wang \& Jacques 1992 {[}WJ92{]}.
\end{DUlineblock}

\sphinxAtStartPar
\(\theta_{HG} =  
\left \{ \begin{matrix}
\mbox{arccos}(2RND - 1) , & \mbox{if } g = 0 \\
\mbox{arccos}\left\{\frac{1}{2g} \left[1 + g^2 - \left(\frac{1 - g^2}{1 - g + 2gRND}\right)^2\right]\right\}, & \mbox{if } g \ne 0
\end{matrix} \right.\)

\sphinxAtStartPar
In this particular model \(g\) is the anisotropy factor.
\begin{quote}\begin{description}
\sphinxlineitem{Parameters}\begin{itemize}
\item {} 
\sphinxAtStartPar
\sphinxstyleliteralstrong{\sphinxupquote{g}} (\sphinxstyleliteralemphasis{\sphinxupquote{float}}) \textendash{} parameter g {[}\sphinxhyphen{}{]} (must be in the range {[}\sphinxhyphen{}1, 1{]})

\item {} 
\sphinxAtStartPar
\sphinxstyleliteralstrong{\sphinxupquote{n\_RND}} (\sphinxstyleliteralemphasis{\sphinxupquote{int}}) \textendash{} number of random numbers {[}\sphinxhyphen{}{]} (default to int(1E6))

\end{itemize}

\sphinxlineitem{Returns}
\sphinxAtStartPar
\begin{itemize}
\item {} 
\sphinxAtStartPar
\sphinxstylestrong{theta} (\sphinxstyleemphasis{np.ndarray}) \textendash{} scattering angle {[}degrees{]}

\end{itemize}


\end{description}\end{quote}

\end{fulllineitems}

\index{theta\_R\_from\_RND() (in module skinoptics.anisotropy\_factor)@\spxentry{theta\_R\_from\_RND()}\spxextra{in module skinoptics.anisotropy\_factor}}

\begin{fulllineitems}
\phantomsection\label{\detokenize{06_anisotropy_factor:skinoptics.anisotropy_factor.theta_R_from_RND}}
\pysigstartsignatures
\pysiglinewithargsret{\sphinxcode{\sphinxupquote{skinoptics.anisotropy\_factor.}}\sphinxbfcode{\sphinxupquote{theta\_R\_from\_RND}}}{\sphinxparam{\DUrole{n}{n\_RND}\DUrole{o}{=}\DUrole{default_value}{1000000}}}{}
\pysigstopsignatures
\begin{DUlineblock}{0em}
\item[] The scattering angle distribution as a function of a set of random numbers uniformly
\item[] distributed over the interval {[}0, 1), assuming the Rayleigh scattering phase function.
\item[] For details please check section 3.B from Frisvad 2011 {[}F11{]}.
\end{DUlineblock}

\begin{DUlineblock}{0em}
\item[] \(\theta_{R} = \mbox{arccos}(\sqrt[3]{u + v} + \sqrt[3]{u - v})\)
\item[] with
\item[] \(u = -2(2RND - 1)\)
\item[] \(v = \sqrt{4(2RND - 1)^2 + 1}\)
\end{DUlineblock}
\begin{quote}\begin{description}
\sphinxlineitem{Parameters}
\sphinxAtStartPar
\sphinxstyleliteralstrong{\sphinxupquote{n\_RND}} (\sphinxstyleliteralemphasis{\sphinxupquote{int}}) \textendash{} number of random numbers {[}\sphinxhyphen{}{]} (default to int(1E6))

\sphinxlineitem{Returns}
\sphinxAtStartPar
\begin{itemize}
\item {} 
\sphinxAtStartPar
\sphinxstylestrong{theta} (\sphinxstyleemphasis{np.ndarray}) \textendash{} scattering angle {[}degrees{]}

\end{itemize}


\end{description}\end{quote}

\end{fulllineitems}

\index{theta\_U2\_from\_RND() (in module skinoptics.anisotropy\_factor)@\spxentry{theta\_U2\_from\_RND()}\spxextra{in module skinoptics.anisotropy\_factor}}

\begin{fulllineitems}
\phantomsection\label{\detokenize{06_anisotropy_factor:skinoptics.anisotropy_factor.theta_U2_from_RND}}
\pysigstartsignatures
\pysiglinewithargsret{\sphinxcode{\sphinxupquote{skinoptics.anisotropy\_factor.}}\sphinxbfcode{\sphinxupquote{theta\_U2\_from\_RND}}}{\sphinxparam{\DUrole{n}{g}}\sphinxparamcomma \sphinxparam{\DUrole{n}{n\_RND}\DUrole{o}{=}\DUrole{default_value}{1000000}}}{}
\pysigstopsignatures
\begin{DUlineblock}{0em}
\item[] The scattering angle distribution as a function of the g parameter and a set of random
\item[] numbers uniformly distributed over the interval {[}0, 1), assuming the Ultraspherical\sphinxhyphen{}2
\item[] scattering phase function.
\item[] For details please check section 4.4.2 from Baes, Camps \& Kapoor 2022 {[}BCK22{]}.
\end{DUlineblock}

\sphinxAtStartPar
\(\theta_{U2} = arccos\left[\frac{(1 + g)^2 - 2RND(1 + g^2)}{(1 + g)^2 - 4gRND}\right]\)

\begin{DUlineblock}{0em}
\item[] In this model \(g\) is NOT the anisotropy factor.
\end{DUlineblock}
\begin{quote}\begin{description}
\sphinxlineitem{Parameters}\begin{itemize}
\item {} 
\sphinxAtStartPar
\sphinxstyleliteralstrong{\sphinxupquote{g}} (\sphinxstyleliteralemphasis{\sphinxupquote{float}}) \textendash{} parameter g {[}\sphinxhyphen{}{]} (must be in the range {[}\sphinxhyphen{}1, 1{]})

\item {} 
\sphinxAtStartPar
\sphinxstyleliteralstrong{\sphinxupquote{n\_RND}} (\sphinxstyleliteralemphasis{\sphinxupquote{int}}) \textendash{} number of random numbers {[}\sphinxhyphen{}{]} (default to int(1E6))

\end{itemize}

\sphinxlineitem{Returns}
\sphinxAtStartPar
\begin{itemize}
\item {} 
\sphinxAtStartPar
\sphinxstylestrong{theta} (\sphinxstyleemphasis{np.ndarray}) \textendash{} scattering angle {[}degrees{]}

\end{itemize}


\end{description}\end{quote}

\end{fulllineitems}


\sphinxstepscope


\subsection{skinoptics.colors module}
\label{\detokenize{07_colors:module-skinoptics.colors}}\label{\detokenize{07_colors:skinoptics-colors-module}}\label{\detokenize{07_colors::doc}}\index{module@\spxentry{module}!skinoptics.colors@\spxentry{skinoptics.colors}}\index{skinoptics.colors@\spxentry{skinoptics.colors}!module@\spxentry{module}}
\sphinxAtStartPar
Copyright (C) 2024 Victor Lima
\begin{quote}

\begin{DUlineblock}{0em}
\item[] This program is free software: you can redistribute it and/or modify
\item[] it under the terms of the GNU General Public License as published by
\item[] the Free Software Foundation, either version 3 of the License, or
\item[] (at your option) any later version.
\end{DUlineblock}

\begin{DUlineblock}{0em}
\item[] This program is distributed in the hope that it will be useful,
\item[] but WITHOUT ANY WARRANTY; without even the implied warranty of
\item[] MERCHANTABILITY or FITNESS FOR A PARTICULAR PURPOSE.  See the
\item[] GNU General Public License for more details.
\end{DUlineblock}

\begin{DUlineblock}{0em}
\item[] You should have received a copy of the GNU General Public License
\item[] along with this program.  If not, see \textless{}\sphinxurl{https://www.gnu.org/licenses/}\textgreater{}.
\end{DUlineblock}
\end{quote}

\begin{DUlineblock}{0em}
\item[] Victor Lima
\item[] victorporto@ifsc.usp.br
\item[] victor.lima@ufscar.br
\end{DUlineblock}

\begin{DUlineblock}{0em}
\item[] Release Date:
\item[] August 2024
\item[] Last Modification:
\item[] August 2024
\end{DUlineblock}

\begin{DUlineblock}{0em}
\item[] References:
\end{DUlineblock}

\begin{DUlineblock}{0em}
\item[] {[}CCH91{]} Chardon, Cretois \& Hourseau 1991.
\item[] Skin colour typology and suntanning pathways.
\item[] \sphinxurl{https://doi.org/10.1111/j.1467-2494.1991.tb00561.x}
\end{DUlineblock}

\begin{DUlineblock}{0em}
\item[] {[}T*94{]} Takiwaki, Shirai, Kanno, Watanabe \& Arase 1994.
\item[] Quantification of erythema and pigmentation using a videomicroscope and a computer.
\item[] \sphinxurl{https://doi.org/10.1111/j.1365-2133.1994.tb08462.x}
\end{DUlineblock}

\begin{DUlineblock}{0em}
\item[] {[}F*96{]} Fullerton, Fischer, Lahti, Wilhelm, Takiwaki \& Serup 1996.
\item[] Guidetines for measurement of skin colour and erythema: A report from the Standardization Group of the European Society of Contact Dermatitis.
\item[] \sphinxurl{https://doi.org/10.1111/j.1600-0536.1996.tb02258.x}
\end{DUlineblock}

\begin{DUlineblock}{0em}
\item[] {[}S*96{]} Stokes, Anderson, Chandrasekar \& Motta 1996.
\item[] A Standard Default Color Space for the Internet \sphinxhyphen{} sRGB.
\item[] \sphinxurl{https://www.w3.org/Graphics/Color/sRGB.html}
\end{DUlineblock}

\begin{DUlineblock}{0em}
\item[] {[}IEC99{]} IEC 1999.
\item[] Multimedia systems and equipment \sphinxhyphen{} Colour measurement and management \sphinxhyphen{} Part 2\sphinxhyphen{}1: Colour management \sphinxhyphen{} Default RGB colour space \sphinxhyphen{} sRGB.
\item[] IEC 61966\sphinxhyphen{}2\sphinxhyphen{}1:1999
\end{DUlineblock}

\begin{DUlineblock}{0em}
\item[] {[}CIE04{]} CIE 2004.
\item[] Colorimetry, 3rd edition.
\item[] CIE 15:2004
\end{DUlineblock}

\begin{DUlineblock}{0em}
\item[] {[}D*06{]} Del Bino, Sok, Bessac \& Bernerd 2006.
\item[] Relationship between skin response to ultraviolet exposure and skin color type.
\item[] \sphinxurl{https://doi.org/10.1111/j.1600-0749.2006.00338.x}
\end{DUlineblock}

\begin{DUlineblock}{0em}
\item[] {[}S07{]} Schanda (editor) 2007.
\item[] Colorimetry: Understanding the CIE System.
\item[] \sphinxurl{http://dx.doi.org/10.1002/9780470175637}
\end{DUlineblock}

\begin{DUlineblock}{0em}
\item[] {[}HP11{]} Hunt \& Pointer 2011.
\item[] Measuring Colour.
\item[] \sphinxurl{https://doi.org/10.1002/9781119975595}
\end{DUlineblock}

\begin{DUlineblock}{0em}
\item[] {[}DB13{]} Del Bino \& Bernerd 2013.
\item[] Variations in skin colour and the biological consequences of ultraviolet radiation exposure.
\item[] \sphinxurl{https://doi.org/10.1111/bjd.12529}
\end{DUlineblock}

\begin{DUlineblock}{0em}
\item[] {[}WSS13{]} Wyman, Sloan \& Shirley 2013.
\item[] Simple Analytic Approximations to the CIE XYZ Color Matching Functions.
\item[] \sphinxurl{https://jcgt.org/published/0002/02/01/}
\end{DUlineblock}

\begin{DUlineblock}{0em}
\item[] {[}CIE18a{]} CIE 2018.
\item[] CIE standard illuminant A \sphinxhyphen{} 1 nm.
\item[] \sphinxurl{https://doi.org/10.25039/CIE.DS.8jsxjrsn}
\end{DUlineblock}

\begin{DUlineblock}{0em}
\item[] {[}CIE18b{]} CIE 2018.
\item[] CIE standard illuminant D55.
\item[] \sphinxurl{https://doi.org/10.25039/CIE.DS.qewfb3kp}
\end{DUlineblock}

\begin{DUlineblock}{0em}
\item[] {[}CIE18c{]} CIE 2018.
\item[] CIE standard illuminant D75.
\item[] \sphinxurl{https://doi.org/10.25039/CIE.DS.9fvcmrk4}
\end{DUlineblock}

\begin{DUlineblock}{0em}
\item[] {[}CIE19a{]} CIE 2019.
\item[] CIE 1931 colour\sphinxhyphen{}matching functions, 2 degree observer.
\item[] \sphinxurl{https://doi.org/10.25039/CIE.DS.xvudnb9b}
\end{DUlineblock}

\begin{DUlineblock}{0em}
\item[] {[}CIE19b{]} CIE 2019.
\item[] CIE 1964 colour\sphinxhyphen{}matching functions, 10 degree observer
\item[] \sphinxurl{https://doi.org/10.25039/CIE.DS.sqksu2n5}
\end{DUlineblock}

\begin{DUlineblock}{0em}
\item[] {[}L*20{]} Ly, Dyer, Feig, Chien \& Del Bino 2020.
\item[] Research Techniques Made Simple: Cutaneous Colorimetry: A Reliable Technique for Objective Skin Color Measurement.
\item[] \sphinxurl{https://doi.org/10.1016/j.jid.2019.11.003}
\end{DUlineblock}

\begin{DUlineblock}{0em}
\item[] {[}CIE22a{]} CIE 2022.
\item[] CIE standard illuminant D50.
\item[] \sphinxurl{https://doi.org/10.25039/CIE.DS.etgmuqt5}
\end{DUlineblock}

\begin{DUlineblock}{0em}
\item[] {[}CIE22b{]} CIE 2022.
\item[] CIE standard illuminant D65.
\item[] \sphinxurl{https://doi.org/10.25039/CIE.DS.hjfjmt59}
\end{DUlineblock}
\index{Delta\_E() (in module skinoptics.colors)@\spxentry{Delta\_E()}\spxextra{in module skinoptics.colors}}

\begin{fulllineitems}
\phantomsection\label{\detokenize{07_colors:skinoptics.colors.Delta_E}}
\pysigstartsignatures
\pysiglinewithargsret{\sphinxcode{\sphinxupquote{skinoptics.colors.}}\sphinxbfcode{\sphinxupquote{Delta\_E}}}{\sphinxparam{\DUrole{n}{L0}}\sphinxparamcomma \sphinxparam{\DUrole{n}{a0}}\sphinxparamcomma \sphinxparam{\DUrole{n}{b0}}\sphinxparamcomma \sphinxparam{\DUrole{n}{L1}}\sphinxparamcomma \sphinxparam{\DUrole{n}{a1}}\sphinxparamcomma \sphinxparam{\DUrole{n}{b1}}}{}
\pysigstopsignatures
\sphinxAtStartPar
Calculate the color difference \(\Delta E\) between between
a reference color (L:math:\sphinxtitleref{\textasciicircum{}*\_0}, a:math:\sphinxtitleref{\textasciicircum{}*\_0}, b:math:\sphinxtitleref{\textasciicircum{}*\_0}) and
a test color (L:math:\sphinxtitleref{\textasciicircum{}*\_1}, a:math:\sphinxtitleref{\textasciicircum{}*\_1}, b:math:\sphinxtitleref{\textasciicircum{}*\_1}).

\sphinxAtStartPar
\(\Delta E = \sqrt{(L^*_1 - L^*_0)*^2 + (a^*_1 - a^*_0)^2 + (b^*_1 - b*_0)^2}\)


\subsubsection{Parameters}
\label{\detokenize{07_colors:parameters}}
\sphinxAtStartPar
L0: float or np.ndarray
reference color L* coordinate {[}\sphinxhyphen{}{]}

\sphinxAtStartPar
a0: float or np.ndarray
reference color a* coordinate {[}\sphinxhyphen{}{]}

\sphinxAtStartPar
b0: float or np.ndarray
reference color b* coordinate {[}\sphinxhyphen{}{]}

\sphinxAtStartPar
L1: float or np.ndarray
test color L* coordinate {[}\sphinxhyphen{}{]}

\sphinxAtStartPar
a1: float or np.ndarray
test color a* coordinate {[}\sphinxhyphen{}{]}

\sphinxAtStartPar
b1: float or np.ndarray
test color b* coordinate {[}\sphinxhyphen{}{]}


\subsubsection{Returns}
\label{\detokenize{07_colors:returns}}
\sphinxAtStartPar
delta\_E: np.ndarray
color difference {[}\sphinxhyphen{}{]}

\end{fulllineitems}

\index{Delta\_L() (in module skinoptics.colors)@\spxentry{Delta\_L()}\spxextra{in module skinoptics.colors}}

\begin{fulllineitems}
\phantomsection\label{\detokenize{07_colors:skinoptics.colors.Delta_L}}
\pysigstartsignatures
\pysiglinewithargsret{\sphinxcode{\sphinxupquote{skinoptics.colors.}}\sphinxbfcode{\sphinxupquote{Delta\_L}}}{\sphinxparam{\DUrole{n}{L0}}\sphinxparamcomma \sphinxparam{\DUrole{n}{L1}}}{}
\pysigstopsignatures
\sphinxAtStartPar
Calculate the lightness difference \(\Delta`L:math:`^*\) between a reference color lightness L:math:\sphinxtitleref{\textasciicircum{}*\_0}
and a test color lightness L:math:\sphinxtitleref{\textasciicircum{}*\_1}.

\sphinxAtStartPar
\(\Delta L^* = L^*_1 - L^*_0\)


\subsubsection{Parameters}
\label{\detokenize{07_colors:id1}}
\sphinxAtStartPar
L0: float or np.ndarray
reference color L\textasciicircum{}* coordinate {[}\sphinxhyphen{}{]}

\sphinxAtStartPar
L1: float or np.ndarray
test color L\textasciicircum{}* coordinate {[}\sphinxhyphen{}{]}


\subsubsection{Returns}
\label{\detokenize{07_colors:id2}}
\sphinxAtStartPar
delta\_L: float or np.ndarray
lightness difference {[}\sphinxhyphen{}{]}

\end{fulllineitems}

\index{Delta\_a() (in module skinoptics.colors)@\spxentry{Delta\_a()}\spxextra{in module skinoptics.colors}}

\begin{fulllineitems}
\phantomsection\label{\detokenize{07_colors:skinoptics.colors.Delta_a}}
\pysigstartsignatures
\pysiglinewithargsret{\sphinxcode{\sphinxupquote{skinoptics.colors.}}\sphinxbfcode{\sphinxupquote{Delta\_a}}}{\sphinxparam{\DUrole{n}{a0}}\sphinxparamcomma \sphinxparam{\DUrole{n}{a1}}}{}
\pysigstopsignatures
\sphinxAtStartPar
Calculate the a:math:\sphinxtitleref{\textasciicircum{}*} difference \(\Delta`a:math:`^*\) between
a reference color (L:math:\sphinxtitleref{\textasciicircum{}*\_0}, a:math:\sphinxtitleref{\textasciicircum{}*\_0}, b:math:\sphinxtitleref{\textasciicircum{}*\_0}) and
a test color (L:math:\sphinxtitleref{\textasciicircum{}*\_1}, a:math:\sphinxtitleref{\textasciicircum{}*\_1}, b:math:\sphinxtitleref{\textasciicircum{}*\_1}).

\sphinxAtStartPar
\(\Delta a^* = a^*_1 - a^*_0\)


\subsubsection{Parameters}
\label{\detokenize{07_colors:id3}}
\sphinxAtStartPar
a0: float or np.ndarray
reference color a* coordinate {[}\sphinxhyphen{}{]}

\sphinxAtStartPar
a1: float or np.ndarray
test color a* coordinate {[}\sphinxhyphen{}{]}


\subsubsection{Returns}
\label{\detokenize{07_colors:id4}}
\sphinxAtStartPar
delta\_a: float or np.ndarray
a* difference {[}\sphinxhyphen{}{]}

\end{fulllineitems}

\index{Delta\_b() (in module skinoptics.colors)@\spxentry{Delta\_b()}\spxextra{in module skinoptics.colors}}

\begin{fulllineitems}
\phantomsection\label{\detokenize{07_colors:skinoptics.colors.Delta_b}}
\pysigstartsignatures
\pysiglinewithargsret{\sphinxcode{\sphinxupquote{skinoptics.colors.}}\sphinxbfcode{\sphinxupquote{Delta\_b}}}{\sphinxparam{\DUrole{n}{b0}}\sphinxparamcomma \sphinxparam{\DUrole{n}{b1}}}{}
\pysigstopsignatures
\sphinxAtStartPar
Calculate the b:math:\sphinxtitleref{\textasciicircum{}*} difference \(\Delta`b:math:`^*\) between
a reference color (L:math:\sphinxtitleref{\textasciicircum{}*\_0}, a:math:\sphinxtitleref{\textasciicircum{}*\_0}, b:math:\sphinxtitleref{\textasciicircum{}*\_0}) and
a test color (L:math:\sphinxtitleref{\textasciicircum{}*\_1}, a:math:\sphinxtitleref{\textasciicircum{}*\_1}, b:math:\sphinxtitleref{\textasciicircum{}*\_1}).

\sphinxAtStartPar
:math:\sphinxtitleref{Delta b\textasciicircum{}* = b\textasciicircum{}*\_1 \sphinxhyphen{} b\textasciicircum{}*\_0}


\subsubsection{Parameters}
\label{\detokenize{07_colors:id5}}
\sphinxAtStartPar
b0: float or np.ndarray
reference color b* coordinate {[}\sphinxhyphen{}{]}

\sphinxAtStartPar
b1: float or np.ndarray
test color b* coordinate {[}\sphinxhyphen{}{]}


\subsubsection{Returns}
\label{\detokenize{07_colors:id6}}
\sphinxAtStartPar
delta\_b: float or np.ndarray
b* difference {[}\sphinxhyphen{}{]}

\end{fulllineitems}

\index{EI() (in module skinoptics.colors)@\spxentry{EI()}\spxextra{in module skinoptics.colors}}

\begin{fulllineitems}
\phantomsection\label{\detokenize{07_colors:skinoptics.colors.EI}}
\pysigstartsignatures
\pysiglinewithargsret{\sphinxcode{\sphinxupquote{skinoptics.colors.}}\sphinxbfcode{\sphinxupquote{EI}}}{\sphinxparam{\DUrole{n}{R\_green}}\sphinxparamcomma \sphinxparam{\DUrole{n}{R\_red}}}{}
\pysigstopsignatures
\sphinxAtStartPar
Calculate the Erythema Index (EI) from the reflectances on chosen green 
(usually approx. 568 nm) and red bands (usually approx. 655 nm).
For details please check Takiwaki et al. 1994 {[}T*94{]} and Fullerton et al. 1996 {[}F*96{]}.

\sphinxAtStartPar
\(EI = 100[\mbox{log}_{10}(R_\mbox{red}) - \mbox{log}_{10}(R_\mbox{green})]\)


\subsubsection{Parameters}
\label{\detokenize{07_colors:id7}}
\sphinxAtStartPar
R\_green: float or np.ndarray
reflectance on a chosen green band. {[}\%{]}

\sphinxAtStartPar
R\_red: float or np.ndarray
reflectance on a chosen red band {[}\%{]}


\subsubsection{Returns}
\label{\detokenize{07_colors:id8}}
\sphinxAtStartPar
EI: float or np.ndarray
Erythema Index {[}\sphinxhyphen{}{]}

\end{fulllineitems}

\index{ITA() (in module skinoptics.colors)@\spxentry{ITA()}\spxextra{in module skinoptics.colors}}

\begin{fulllineitems}
\phantomsection\label{\detokenize{07_colors:skinoptics.colors.ITA}}
\pysigstartsignatures
\pysiglinewithargsret{\sphinxcode{\sphinxupquote{skinoptics.colors.}}\sphinxbfcode{\sphinxupquote{ITA}}}{\sphinxparam{\DUrole{n}{L}}\sphinxparamcomma \sphinxparam{\DUrole{n}{b}}\sphinxparamcomma \sphinxparam{\DUrole{n}{L0}\DUrole{o}{=}\DUrole{default_value}{50.0}}}{}
\pysigstopsignatures
\begin{DUlineblock}{0em}
\item[] Calculate the Individual Typology Angle (ITA) from L* and b* coordinates.
\item[] For details please check Chardon, Cretois \& Hourseau 1991 {[}CCH91{]}, DelBino et al. 2006 {[}D*06{]}
\item[] DelBino \& Bernerd 2013 {[}DB13{]} and Ly et al. {[}L*20{]}.
\end{DUlineblock}

\sphinxAtStartPar
\(ITA = \arctan\left(\frac{L-L_0}{b}\right)\frac{180}{\pi}\)


\subsubsection{Parameters}
\label{\detokenize{07_colors:id9}}
\sphinxAtStartPar
L: float or np.ndarray
L* coordinate {[}\sphinxhyphen{}{]}

\sphinxAtStartPar
b: float or np.ndarray
b* coordinate {[}\sphinxhyphen{}{]}

\sphinxAtStartPar
L0: float (default to 50.)
L0 coordinate {[}\sphinxhyphen{}{]}


\subsubsection{Returns}
\label{\detokenize{07_colors:id10}}
\sphinxAtStartPar
ITA: float or np.ndarray 
Individual Typology Angle {[}degrees{]}

\end{fulllineitems}

\index{ITA\_class() (in module skinoptics.colors)@\spxentry{ITA\_class()}\spxextra{in module skinoptics.colors}}

\begin{fulllineitems}
\phantomsection\label{\detokenize{07_colors:skinoptics.colors.ITA_class}}
\pysigstartsignatures
\pysiglinewithargsret{\sphinxcode{\sphinxupquote{skinoptics.colors.}}\sphinxbfcode{\sphinxupquote{ITA\_class}}}{\sphinxparam{\DUrole{n}{ITA}}}{}
\pysigstopsignatures
\begin{DUlineblock}{0em}
\item[] Skin color classification based on the Individual Typology Angle (see function ITA).
\item[] For details please check Chardon, Cretois \& Hourseau 1991 {[}CCH91{]}, DelBino et al. 2006 {[}D*06{]}
\item[] and Ly et al. {[}L*20{]}.
\end{DUlineblock}


\begin{savenotes}\sphinxattablestart
\sphinxthistablewithglobalstyle
\centering
\begin{tabulary}{\linewidth}[t]{TT}
\sphinxtoprule
\sphinxstyletheadfamily 
\sphinxAtStartPar
skin color classification
&\sphinxstyletheadfamily 
\sphinxAtStartPar
ITA range
\\
\sphinxmidrule
\sphinxtableatstartofbodyhook
\sphinxAtStartPar
very light
&
\sphinxAtStartPar
ITA \(> 55^\circ\)
\\
\sphinxhline
\sphinxAtStartPar
light
&
\sphinxAtStartPar
\(41^\circ <\) ITA \(\le 55^\circ\)
\\
\sphinxhline
\sphinxAtStartPar
intermediate
&
\sphinxAtStartPar
\(28^\circ <\) ITA \(\le 41^\circ\)
\\
\sphinxhline
\sphinxAtStartPar
tan
&
\sphinxAtStartPar
\(10^\circ <\) ITA \(\le 28^\circ\)
\\
\sphinxhline
\sphinxAtStartPar
brown
&
\sphinxAtStartPar
\(-30^\circ <\) ITA \(\le 10^\circ\)
\\
\sphinxhline
\sphinxAtStartPar
dark
&
\sphinxAtStartPar
ITA \(\le -30^\circ\)
\\
\sphinxbottomrule
\end{tabulary}
\sphinxtableafterendhook\par
\sphinxattableend\end{savenotes}
\begin{quote}\begin{description}
\sphinxlineitem{Parameters}
\sphinxAtStartPar
\sphinxstyleliteralstrong{\sphinxupquote{ITA}} (\sphinxstyleliteralemphasis{\sphinxupquote{float}}\sphinxstyleliteralemphasis{\sphinxupquote{ or }}\sphinxstyleliteralemphasis{\sphinxupquote{np.ndarray}}) \textendash{} Individual Typology Angle {[}degrees{]} (must be greater than \sphinxhyphen{}90 and less than 90)

\sphinxlineitem{Returns}
\sphinxAtStartPar
\begin{itemize}
\item {} 
\sphinxAtStartPar
\sphinxstylestrong{ITA\_class} (\sphinxstyleemphasis{str or np.ndarray}) \textendash{} skin color classification based on the Individual Typology Angle

\end{itemize}


\end{description}\end{quote}

\end{fulllineitems}

\index{Lab\_from\_XYZ() (in module skinoptics.colors)@\spxentry{Lab\_from\_XYZ()}\spxextra{in module skinoptics.colors}}

\begin{fulllineitems}
\phantomsection\label{\detokenize{07_colors:skinoptics.colors.Lab_from_XYZ}}
\pysigstartsignatures
\pysiglinewithargsret{\sphinxcode{\sphinxupquote{skinoptics.colors.}}\sphinxbfcode{\sphinxupquote{Lab\_from\_XYZ}}}{\sphinxparam{\DUrole{n}{X}}\sphinxparamcomma \sphinxparam{\DUrole{n}{Y}}\sphinxparamcomma \sphinxparam{\DUrole{n}{Z}}\sphinxparamcomma \sphinxparam{\DUrole{n}{illuminant}\DUrole{o}{=}\DUrole{default_value}{\textquotesingle{}D65\textquotesingle{}}}\sphinxparamcomma \sphinxparam{\DUrole{n}{observer}\DUrole{o}{=}\DUrole{default_value}{\textquotesingle{}10o\textquotesingle{}}}\sphinxparamcomma \sphinxparam{\DUrole{n}{K}\DUrole{o}{=}\DUrole{default_value}{1.0}}}{}
\pysigstopsignatures
\begin{DUlineblock}{0em}
\item[] Calculate CIE L*a*b* coordinates from CIE XYZ coordinates.
\item[] CIE XYZ and CIE L*a*b* coordinates must be for the same standard illuminant and standard observer.
\item[] For detailts please check CIE {[}CIE04{]}, Schanda 2006 {[}S06{]} and Hunt \& Pointer 2011 {[}HP11{]}.
\end{DUlineblock}

\sphinxAtStartPar
\(L^* = 116 f(Y/Y_n) - 16,\)
\(a^* = 500 [f(X/X_n) - f(Y/Y_n)],\)
\(b^* = 200 [f(Y/Y_n) - f(Z/Z_n)],\)

\sphinxAtStartPar
in which (\(X_n\) \(Y_n\), \(Z_n\)) is the white point and

\sphinxAtStartPar
\(f(t) = \left\{ 
\begin{matrix}
\sqrt[3]{t}, & \mbox{if }  t > \left(\frac{6}{29}\right)^3 \\
\frac{1}{3}\left(\frac{29}{6}\right)^2 t + \frac{4}{29}, & \mbox{if }  t \le \left(\frac{6}{29}\right)^3
\end{matrix}\right.\)


\subsubsection{Parameters}
\label{\detokenize{07_colors:id11}}
\sphinxAtStartPar
X: float or np.ndarray
X coordinate {[}\sphinxhyphen{}{]}

\sphinxAtStartPar
Y: float or np.ndarray
Y coordinate {[}\sphinxhyphen{}{]}

\sphinxAtStartPar
Z: float or np.ndarray
Z coordinate {[}\sphinxhyphen{}{]}

\sphinxAtStartPar
illuminant: str (default to ‘D65’)
the use can choose one of the following… ‘A’, ‘D50’, ‘D55’, ‘D65’ or ‘D75’
‘A’ refers to the CIE standard illuminant A
‘D50’ refers to the CIE standard illuminant D50
‘D55’ refers to the CIE standard illuminant D55
‘D65’ refers to the CIE standard illuminant D65
‘D75’ refers to the CIE standard illuminant D75

\sphinxAtStartPar
observer: str (default to ‘10o’)
the use can choose one of the following… ‘2o’ or ‘10o’
‘2o’ refers to the CIE 1931 2 degree standard observer
‘10o’ refers to the CIE 1964 10 degree standard observer

\sphinxAtStartPar
K: float (default to 1.)
scaling factor (usually 1. or 100.) {[}\sphinxhyphen{}{]}
K = 1. for CIE XYZ coordinates in range {[}0, 1{]}
K = 100. for CIE XYZ coordinates in range {[}0, 100{]}


\subsubsection{Returns}
\label{\detokenize{07_colors:id12}}
\sphinxAtStartPar
L: float or np.ndarray
L* coordinate {[}\sphinxhyphen{}{]}

\sphinxAtStartPar
a: float or np.ndarray
a* coordinate {[}\sphinxhyphen{}{]}

\sphinxAtStartPar
b: float or np.ndarray
b* coordinate {[}\sphinxhyphen{}{]}

\end{fulllineitems}

\index{Lab\_from\_spectrum() (in module skinoptics.colors)@\spxentry{Lab\_from\_spectrum()}\spxextra{in module skinoptics.colors}}

\begin{fulllineitems}
\phantomsection\label{\detokenize{07_colors:skinoptics.colors.Lab_from_spectrum}}
\pysigstartsignatures
\pysiglinewithargsret{\sphinxcode{\sphinxupquote{skinoptics.colors.}}\sphinxbfcode{\sphinxupquote{Lab\_from\_spectrum}}}{\sphinxparam{\DUrole{n}{all\_lambda}}\sphinxparamcomma \sphinxparam{\DUrole{n}{spectrum}}\sphinxparamcomma \sphinxparam{\DUrole{n}{lambda\_min}\DUrole{o}{=}\DUrole{default_value}{360}}\sphinxparamcomma \sphinxparam{\DUrole{n}{lambda\_max}\DUrole{o}{=}\DUrole{default_value}{830}}\sphinxparamcomma \sphinxparam{\DUrole{n}{lambda\_step}\DUrole{o}{=}\DUrole{default_value}{1}}\sphinxparamcomma \sphinxparam{\DUrole{n}{illuminant}\DUrole{o}{=}\DUrole{default_value}{\textquotesingle{}D65\textquotesingle{}}}\sphinxparamcomma \sphinxparam{\DUrole{n}{observer}\DUrole{o}{=}\DUrole{default_value}{\textquotesingle{}10o\textquotesingle{}}}\sphinxparamcomma \sphinxparam{\DUrole{n}{cmfs\_model}\DUrole{o}{=}\DUrole{default_value}{\textquotesingle{}CIE\textquotesingle{}}}\sphinxparamcomma \sphinxparam{\DUrole{n}{interp1d\_kind}\DUrole{o}{=}\DUrole{default_value}{\textquotesingle{}cubic\textquotesingle{}}}}{}
\pysigstopsignatures
\sphinxAtStartPar
Calculate the CIE L*a*b* coordinates from the reflectance or the transmittance spectrum.
First calculate CIE XYZ coordinates from the spectrum for a chosen standard illuminant
and standard observer and then calculate CIE L*a*b* coordinates from CIE XYZ coordinates
(see functions Lab\_from\_XYZ and XYZ\_from\_spectrum).


\subsubsection{Parameters}
\label{\detokenize{07_colors:id13}}
\sphinxAtStartPar
all\_lambda: np.ndarray
wavelength array

\sphinxAtStartPar
spectrum: np.ndarray
reflectance or transmittance spectrum respective to the wavelength array {[}\%{]}

\sphinxAtStartPar
lambda\_min: float (default to 360.)
lower limit of summation/integration (minimum wavelength to take into account) {[}nm{]}

\sphinxAtStartPar
lambda\_max: float (default to 830.)
upper limit of summation/integration (maximum wavelength to take into account) {[}nm{]}

\sphinxAtStartPar
lambda\_step: float (default to 1.)
summation interval (wavelength step) {[}nm{]}

\sphinxAtStartPar
illuminant: str (default to ‘D65’)
the use can choose one of the following… ‘A’, ‘D50’, ‘D55’, ‘D65’ or ‘D75’
‘A’ refers to the CIE standard illuminant A
‘D50’ refers to the CIE standard illuminant D50
‘D55’ refers to the CIE standard illuminant D55
‘D65’ refers to the CIE standard illuminant D65
‘D75’ refers to the CIE standard illuminant D75

\sphinxAtStartPar
observer: str (default to ‘10o’)
the use can choose one of the following… ‘2o’ or ‘10o’
‘2o’ refers to the CIE 1931 2 degree standard observer
‘10o’ refers to the CIE 1964 10 degree standard observer

\sphinxAtStartPar
cmfs\_model: str (default to ‘CIE’)
the user can choose one of the following… ‘CIE’, ‘Wyman\_singlelobe’ or ‘Wyman\_multilobe’
‘CIE’ for the linear interpolation of data from CIE datasets {[}CIE19a{]} {[}CIE19b{]}
‘Wyman\_singlelobe’ for the analytical functions from Wyman, Sloan \& Shirley 2013 {[}WSS13{]} (section 2.1)
‘Wyman\_multilobe’ for the analytical functions from Wyman, Sloan \& Shirley 2013 {[}WSS13{]} (section 2.2)

\sphinxAtStartPar
interp1d\_kind: str (default to ‘cubic’)
kind argument of scipy.interpolation.interp1d (see scipy documentation)
default to ‘cubic’ as recommended by CIE {[}CIE04{]} (see section 7.2.1.1)


\subsubsection{Returns}
\label{\detokenize{07_colors:id14}}
\sphinxAtStartPar
L: float
L* coordinate {[}\sphinxhyphen{}{]}

\sphinxAtStartPar
a: float
a* coordinate {[}\sphinxhyphen{}{]}

\sphinxAtStartPar
b: float
b* coordinate {[}\sphinxhyphen{}{]}

\end{fulllineitems}

\index{MI() (in module skinoptics.colors)@\spxentry{MI()}\spxextra{in module skinoptics.colors}}

\begin{fulllineitems}
\phantomsection\label{\detokenize{07_colors:skinoptics.colors.MI}}
\pysigstartsignatures
\pysiglinewithargsret{\sphinxcode{\sphinxupquote{skinoptics.colors.}}\sphinxbfcode{\sphinxupquote{MI}}}{\sphinxparam{\DUrole{n}{R\_red}}}{}
\pysigstopsignatures
\sphinxAtStartPar
Calculate the Melanin Index (MI) from the reflectance on a chosen red band
(usually approx. 655 nm).
For details please check Takiwaki et al. 1994 {[}T*94{]} and Fullerton et al. 1996 {[}F*96{]}.

\sphinxAtStartPar
\(MI = 100[-\mbox{log}_{10}(R_\mbox{red})]\)


\subsubsection{Parameters}
\label{\detokenize{07_colors:id15}}
\sphinxAtStartPar
R\_red: float or np.ndarray
reflectance on a chosen red band. {[}\%{]}


\subsubsection{Returns}
\label{\detokenize{07_colors:id16}}
\sphinxAtStartPar
MI: float or np.ndarray
Melanin Index {[}\sphinxhyphen{}{]}

\end{fulllineitems}

\index{XYZ\_from\_Lab() (in module skinoptics.colors)@\spxentry{XYZ\_from\_Lab()}\spxextra{in module skinoptics.colors}}

\begin{fulllineitems}
\phantomsection\label{\detokenize{07_colors:skinoptics.colors.XYZ_from_Lab}}
\pysigstartsignatures
\pysiglinewithargsret{\sphinxcode{\sphinxupquote{skinoptics.colors.}}\sphinxbfcode{\sphinxupquote{XYZ\_from\_Lab}}}{\sphinxparam{\DUrole{n}{L}}\sphinxparamcomma \sphinxparam{\DUrole{n}{a}}\sphinxparamcomma \sphinxparam{\DUrole{n}{b}}\sphinxparamcomma \sphinxparam{\DUrole{n}{illuminant}\DUrole{o}{=}\DUrole{default_value}{\textquotesingle{}D65\textquotesingle{}}}\sphinxparamcomma \sphinxparam{\DUrole{n}{observer}\DUrole{o}{=}\DUrole{default_value}{\textquotesingle{}10o\textquotesingle{}}}\sphinxparamcomma \sphinxparam{\DUrole{n}{K}\DUrole{o}{=}\DUrole{default_value}{1.0}}}{}
\pysigstopsignatures
\sphinxAtStartPar
Calculate CIE XYZ coordinates from CIE L*a*b* coordinates.
CIE XYZ and CIE L*a*b* coordinates must be for the same standard illuminant and standard observer.
For detailts please check CIE {[}CIE04{]}, Schanda 2006 {[}S06{]} and Hunt \& Pointer 2011 {[}HP11{]}.

\sphinxAtStartPar
\(X = f^{-1}[(L^* + 16)/116 + a^*/500]X_n\)
\(Y = f^{-1}[(L^* + 16)/116]Y_n\)
\(Z = f^{-1}[(L^* + 16)/116 - b^*/200]Z_n\)

\sphinxAtStartPar
in which (\(X_n\), \(Y_n\), \(Z_n\)) is the white point and

\sphinxAtStartPar
\(f^{-1}(u) = \left\{ 
\begin{matrix}
u^3, & \mbox{if }  u > \frac{6}{29} \\
3\left(\frac{6}{29}\right)^2\left(u - \frac{4}{29} \right), & \mbox{if } u \le \frac{6}{29}
\end{matrix}\right.\)


\subsubsection{Parameters}
\label{\detokenize{07_colors:id17}}
\sphinxAtStartPar
L: float or np.ndarray (must be in range {[}0, 100{]})
L* coordinate {[}\sphinxhyphen{}{]}

\sphinxAtStartPar
b: float or np.ndarray
a* coordinate {[}\sphinxhyphen{}{]}

\sphinxAtStartPar
b: float or np.ndarray
b* coordinate {[}\sphinxhyphen{}{]}

\sphinxAtStartPar
illuminant: str (default to ‘D65’)
the use can choose one of the following… ‘A’, ‘D50’, ‘D55’, ‘D65’ or ‘D75’
‘A’ refers to the CIE standard illuminant A
‘D50’ refers to the CIE standard illuminant D50
‘D55’ refers to the CIE standard illuminant D55
‘D65’ refers to the CIE standard illuminant D65
‘D75’ refers to the CIE standard illuminant D75

\sphinxAtStartPar
observer: str (default to ‘10o’)
the use can choose one of the following… ‘2o’ or ‘10o’
‘2o’ refers to the CIE 1931 2° standard observer
‘10o’ refers to the CIE 1964 10° standard observer

\sphinxAtStartPar
K: float (default to 1.)
scaling factor (usually 1. or 100.) {[}\sphinxhyphen{}{]}
K = 1. for CIE XYZ coordinates in range {[}0, 1{]}
K = 100. for CIE XYZ coordinates in range {[}0, 100{]}


\subsubsection{Returns}
\label{\detokenize{07_colors:id18}}
\sphinxAtStartPar
X: float or np.ndarray
X coordinate {[}\sphinxhyphen{}{]}

\sphinxAtStartPar
Y: float or np.ndarray
Y coordinate {[}\sphinxhyphen{}{]}

\sphinxAtStartPar
Z: float or np.ndarray
Z coordinate {[}\sphinxhyphen{}{]}

\end{fulllineitems}

\index{XYZ\_from\_sRGB() (in module skinoptics.colors)@\spxentry{XYZ\_from\_sRGB()}\spxextra{in module skinoptics.colors}}

\begin{fulllineitems}
\phantomsection\label{\detokenize{07_colors:skinoptics.colors.XYZ_from_sRGB}}
\pysigstartsignatures
\pysiglinewithargsret{\sphinxcode{\sphinxupquote{skinoptics.colors.}}\sphinxbfcode{\sphinxupquote{XYZ\_from\_sRGB}}}{\sphinxparam{\DUrole{n}{R}}\sphinxparamcomma \sphinxparam{\DUrole{n}{G}}\sphinxparamcomma \sphinxparam{\DUrole{n}{B}}\sphinxparamcomma \sphinxparam{\DUrole{n}{K}\DUrole{o}{=}\DUrole{default_value}{1.0}}\sphinxparamcomma \sphinxparam{\DUrole{n}{sRGB\_scale}\DUrole{o}{=}\DUrole{default_value}{\textquotesingle{}norm\textquotesingle{}}}}{}
\pysigstopsignatures
\begin{DUlineblock}{0em}
\item[] Calculate CIE XYZ coordinates from sRGB coordinates.
\item[] CIE XYZ coordinates are respective to the standard illuminant D65 and the 2 degree standard observer.
\item[] For details please check Stokes et al. {[}S*96{]}, and IEC {[}IEC99{]}.
\end{DUlineblock}

\sphinxAtStartPar
\(\begin{bmatrix}
X \\
Y \\
Z
\end{bmatrix}
=
\mathcal{M}^{-1}
\begin{bmatrix}
R_{linear} \\
G_{linear} \\
B_{linear}
\end{bmatrix}\)

\sphinxAtStartPar
in which

\sphinxAtStartPar
\(\begin{bmatrix}
R_{linear} \\
G_{linear} \\
B_{linear}
\end{bmatrix}
=
\begin{bmatrix}
\gamma^{-1}(R) \\
\gamma^{-1}(G) \\
\gamma^{-1}(B)
\end{bmatrix}\)

\sphinxAtStartPar
and

\sphinxAtStartPar
\(\gamma^{-1}(u) =  
\left \{ \begin{matrix}
u/12.92, & \mbox{if } u \le 0.04045 \\
[(u + 0.055)/1.055]^{2.4}, & \mbox{if } u > 0.04045 \\
\end{matrix} \right.\)
\begin{quote}\begin{description}
\sphinxlineitem{Parameters}\begin{itemize}
\item {} 
\sphinxAtStartPar
\sphinxstyleliteralstrong{\sphinxupquote{R}} (\sphinxstyleliteralemphasis{\sphinxupquote{float}}\sphinxstyleliteralemphasis{\sphinxupquote{ or }}\sphinxstyleliteralemphasis{\sphinxupquote{np.ndarray}}) \textendash{} R coordinate {[}\sphinxhyphen{}{]}

\item {} 
\sphinxAtStartPar
\sphinxstyleliteralstrong{\sphinxupquote{G}} (\sphinxstyleliteralemphasis{\sphinxupquote{float}}\sphinxstyleliteralemphasis{\sphinxupquote{ or }}\sphinxstyleliteralemphasis{\sphinxupquote{np.ndarray}}) \textendash{} G coordinate {[}\sphinxhyphen{}{]}

\item {} 
\sphinxAtStartPar
\sphinxstyleliteralstrong{\sphinxupquote{B}} (\sphinxstyleliteralemphasis{\sphinxupquote{float}}\sphinxstyleliteralemphasis{\sphinxupquote{ or }}\sphinxstyleliteralemphasis{\sphinxupquote{np.ndarray}}) \textendash{} B coordinate {[}\sphinxhyphen{}{]}

\end{itemize}

\end{description}\end{quote}

\sphinxAtStartPar
K: float (default to 1.)
scaling factor (usually 1. or 100.) {[}\sphinxhyphen{}{]}
K = 1. for CIE XYZ coordinates in range {[}0, 1{]}
K = 100. for CIE XYZ coordinates in range {[}0, 100{]}

\sphinxAtStartPar
sRGB\_scale: str (default to ‘norm’)
the user can choose one of the following… ‘norm’ or ‘8bit’
‘norm’ for sRGB coordinates in range {[}0,1{]} (normalized scale)
‘8bit’ for sRGB coordinates in range {[}0, 255{]} (8\sphinxhyphen{}bit scale)


\subsubsection{Returns}
\label{\detokenize{07_colors:id19}}
\sphinxAtStartPar
X: float or np.ndarray
X coordinate {[}\sphinxhyphen{}{]}

\sphinxAtStartPar
Y: float or np.ndarray
Y coordinate {[}\sphinxhyphen{}{]}

\sphinxAtStartPar
Z: float or np.ndarray
Z coordinate {[}\sphinxhyphen{}{]}

\end{fulllineitems}

\index{XYZ\_from\_spectrum() (in module skinoptics.colors)@\spxentry{XYZ\_from\_spectrum()}\spxextra{in module skinoptics.colors}}

\begin{fulllineitems}
\phantomsection\label{\detokenize{07_colors:skinoptics.colors.XYZ_from_spectrum}}
\pysigstartsignatures
\pysiglinewithargsret{\sphinxcode{\sphinxupquote{skinoptics.colors.}}\sphinxbfcode{\sphinxupquote{XYZ\_from\_spectrum}}}{\sphinxparam{\DUrole{n}{all\_lambda}}\sphinxparamcomma \sphinxparam{\DUrole{n}{spectrum}}\sphinxparamcomma \sphinxparam{\DUrole{n}{lambda\_min}\DUrole{o}{=}\DUrole{default_value}{360.0}}\sphinxparamcomma \sphinxparam{\DUrole{n}{lambda\_max}\DUrole{o}{=}\DUrole{default_value}{830.0}}\sphinxparamcomma \sphinxparam{\DUrole{n}{lambda\_step}\DUrole{o}{=}\DUrole{default_value}{1.0}}\sphinxparamcomma \sphinxparam{\DUrole{n}{illuminant}\DUrole{o}{=}\DUrole{default_value}{\textquotesingle{}D65\textquotesingle{}}}\sphinxparamcomma \sphinxparam{\DUrole{n}{observer}\DUrole{o}{=}\DUrole{default_value}{\textquotesingle{}10o\textquotesingle{}}}\sphinxparamcomma \sphinxparam{\DUrole{n}{cmfs\_model}\DUrole{o}{=}\DUrole{default_value}{\textquotesingle{}CIE\textquotesingle{}}}\sphinxparamcomma \sphinxparam{\DUrole{n}{K}\DUrole{o}{=}\DUrole{default_value}{1.0}}\sphinxparamcomma \sphinxparam{\DUrole{n}{interp1d\_kind}\DUrole{o}{=}\DUrole{default_value}{\textquotesingle{}cubic\textquotesingle{}}}}{}
\pysigstopsignatures
\sphinxAtStartPar
Calculate the CIE XYZ coordinates from the reflectance spectrum \(R(\lambda)\) or the
transmittance spectrum \(T(\lambda)\) for a chosen standard illuminant and standard observer.
Integration using the composite trapezoid rule from 360 nm to 830 nm.
If the wavelength array does not cover the whole region, extrapolation is perfomed.
For details please check CIE {[}CIE04{]} (see section 7).

\sphinxAtStartPar
\(X = \frac{K}{N} \int_\lambda R(\lambda) S(\lambda) \bar{x}(\lambda) d\lambda\)
\(Y = \frac{K}{N}\int_\lambda R(\lambda) S(\lambda) \bar{y}(\lambda) d\lambda\)
\(Z = \frac{K}{N}\int_\lambda R(\lambda) S(\lambda) \bar{z}(\lambda) d\lambda\)
in which 
\(N = \int_\lambda S(\lambda) \bar{y}(\lambda) d\lambda\)
\(K\) is a scaling factor, \(R(\lambda)\) is the reflectance spectrum of the object,
\(S(\lambda)\) is the relative spectral power distribution of the standard illuminant
and \(\bar{x}(\lambda)\), \(\bar{y}(\lambda)\) and \(\bar{z}(\lambda)\) are the
color\sphinxhyphen{}matching functions of the standard observer.
The reflectance spectrum \(R(\lambda)\) is replaced by the transmittance spectrum
\(T(\lambda)\) when dealing with color in some cases.


\subsubsection{Parameters}
\label{\detokenize{07_colors:id20}}
\sphinxAtStartPar
all\_lambda: np.ndarray
wavelength array

\sphinxAtStartPar
spectrum: np.ndarray
reflectance or transmittance spectrum respective to the wavelength array {[}\%{]}

\sphinxAtStartPar
lambda\_min: float (default to 360.)
lower limit of summation/integration (minimum wavelength to take into account) {[}nm{]}

\sphinxAtStartPar
lambda\_max: float (default to 830.)
upper limit of summation/integration (maximum wavelength to take into account) {[}nm{]}

\sphinxAtStartPar
lambda\_step: float (default to 1.)
summation interval (wavelength step) {[}nm{]}

\sphinxAtStartPar
illuminant: str (default to ‘D65’)
the use can choose one of the following… ‘A’, ‘D50’, ‘D55’, ‘D65’ or ‘D75’
‘A’ refers to the CIE standard illuminant A
‘D50’ refers to the CIE standard illuminant D50
‘D55’ refers to the CIE standard illuminant D55
‘D65’ refers to the CIE standard illuminant D65
‘D75’ refers to the CIE standard illuminant D75

\sphinxAtStartPar
observer: str (default to ‘10o’)
the use can choose one of the following… ‘2o’ or ‘10o’
‘2o’ refers to the CIE 1931 2 degree standard observer
‘10o’ refers to the CIE 1964 10 degree standard observer

\sphinxAtStartPar
cmfs\_model: str (default to ‘CIE’)
the user can choose one of the following… ‘CIE’, ‘Wyman\_singlelobe’ or ‘Wyman\_multilobe’
‘CIE’ for the linear interpolation of data from CIE datasets {[}CIE19a{]} {[}CIE19b{]}
‘Wyman\_singlelobe’ for the analytical functions from Wyman, Sloan \& Shirley 2013 {[}WSS13{]} (section 2.1)
‘Wyman\_multilobe’ for the analytical functions from Wyman, Sloan \& Shirley 2013 {[}WSS13{]} (section 2.2)

\sphinxAtStartPar
K: float (default to 1.)
scaling factor (usually 1. or 100.) {[}\sphinxhyphen{}{]}
K = 1. for CIE XYZ coordinates in range {[}0, 1{]}
K = 100. for CIE XYZ coordinates in range {[}0, 100{]}

\sphinxAtStartPar
interp1d\_kind: str (default to ‘cubic’)
kind argument of scipy.interpolation.interp1d (see scipy documentation)
default to ‘cubic’ as recommended by CIE {[}CIE04{]} (see section 7.2.1.1)


\subsubsection{Returns}
\label{\detokenize{07_colors:id21}}
\sphinxAtStartPar
X: float
X coordinate {[}\sphinxhyphen{}{]}

\sphinxAtStartPar
Y: float
Y coordinate {[}\sphinxhyphen{}{]}

\sphinxAtStartPar
Z: float
Z coordinate {[}\sphinxhyphen{}{]}

\end{fulllineitems}

\index{XYZ\_wp() (in module skinoptics.colors)@\spxentry{XYZ\_wp()}\spxextra{in module skinoptics.colors}}

\begin{fulllineitems}
\phantomsection\label{\detokenize{07_colors:skinoptics.colors.XYZ_wp}}
\pysigstartsignatures
\pysiglinewithargsret{\sphinxcode{\sphinxupquote{skinoptics.colors.}}\sphinxbfcode{\sphinxupquote{XYZ\_wp}}}{\sphinxparam{\DUrole{n}{illuminant}}\sphinxparamcomma \sphinxparam{\DUrole{n}{observer}}\sphinxparamcomma \sphinxparam{\DUrole{n}{cmfs\_model}\DUrole{o}{=}\DUrole{default_value}{\textquotesingle{}CIE\textquotesingle{}}}\sphinxparamcomma \sphinxparam{\DUrole{n}{K}\DUrole{o}{=}\DUrole{default_value}{1.0}}}{}
\pysigstopsignatures
\sphinxAtStartPar
The white point CIE XYZ coordinates for a chosen standard illuminant and standard observer.
\begin{quote}\begin{description}
\sphinxlineitem{Parameters}\begin{itemize}
\item {} 
\sphinxAtStartPar
\sphinxstyleliteralstrong{\sphinxupquote{illuminant}} (\sphinxstyleliteralemphasis{\sphinxupquote{str}}) \textendash{} the user can choose one of the following… ‘A’, ‘D50’, ‘D55’, ‘D65’ or ‘D75’

\item {} 
\sphinxAtStartPar
\sphinxstyleliteralstrong{\sphinxupquote{observer}} (\sphinxstyleliteralemphasis{\sphinxupquote{str}}) \textendash{} the user can choose one of the following… ‘2o’ or ‘10o’

\item {} 
\sphinxAtStartPar
\sphinxstyleliteralstrong{\sphinxupquote{cmfs\_model}} (\sphinxstyleliteralemphasis{\sphinxupquote{str}}\sphinxstyleliteralemphasis{\sphinxupquote{ (}}\sphinxstyleliteralemphasis{\sphinxupquote{default to \textquotesingle{}CIE\textquotesingle{}}}\sphinxstyleliteralemphasis{\sphinxupquote{)}}) \textendash{} the user can choose one of the following… ‘CIE’, ‘Wyman\_singlelobe’ or ‘Wyman\_multilobe’

\item {} 
\sphinxAtStartPar
\sphinxstyleliteralstrong{\sphinxupquote{K}} (\sphinxstyleliteralemphasis{\sphinxupquote{float}}) \textendash{} scaling factor (usually 1. or 100.) {[}\sphinxhyphen{}{]} (default to 1.)

\end{itemize}

\end{description}\end{quote}

\begin{DUlineblock}{0em}
\item[] ‘A’ refers to the CIE standard illuminant A
\item[] ‘D50’ refers to the CIE standard illuminant D50
\item[] ‘D55’ refers to the CIE standard illuminant D55
\item[] ‘D65’ refers to the CIE standard illuminant D65
\item[] ‘D75’ refers to the CIE standard illuminant D75
\end{DUlineblock}

\begin{DUlineblock}{0em}
\item[] ‘2o’ refers to the CIE 1931 2 degree standard observer
\item[] ‘10o’ refers to the CIE 1964 10 degree standard observer
\end{DUlineblock}

\begin{DUlineblock}{0em}
\item[] ‘CIE’ for the linear interpolation of data from CIE datasets {[}CIE19a{]} {[}CIE19b{]}
\item[] ‘Wyman\_singlelobe’ for the functions from Wyman, Sloan \& Shirley 2013 {[}WSS13{]} (section 2.1)
\item[] ‘Wyman\_multilobe’ for the functions from Wyman, Sloan \& Shirley 2013 {[}WSS13{]} (section 2.2)
\end{DUlineblock}

\begin{DUlineblock}{0em}
\item[] K = 1. for CIE XYZ coordinates in range {[}0, 1{]}
\item[] K = 100. for CIE XYZ coordinates in range {[}0, 100{]}
\end{DUlineblock}
\begin{quote}\begin{description}
\sphinxlineitem{Returns}
\sphinxAtStartPar
\begin{itemize}
\item {} 
\sphinxAtStartPar
\sphinxstylestrong{Xn} (\sphinxstyleemphasis{float}) \textendash{} white point X coordinate {[}\sphinxhyphen{}{]}

\item {} 
\sphinxAtStartPar
\sphinxstylestrong{Yn} (\sphinxstyleemphasis{float}) \textendash{} white point Y coordinate {[}\sphinxhyphen{}{]}

\item {} 
\sphinxAtStartPar
\sphinxstylestrong{Zn} (\sphinxstyleemphasis{float}) \textendash{} white point Z coordinate {[}\sphinxhyphen{}{]}

\end{itemize}


\end{description}\end{quote}

\end{fulllineitems}

\index{chroma() (in module skinoptics.colors)@\spxentry{chroma()}\spxextra{in module skinoptics.colors}}

\begin{fulllineitems}
\phantomsection\label{\detokenize{07_colors:skinoptics.colors.chroma}}
\pysigstartsignatures
\pysiglinewithargsret{\sphinxcode{\sphinxupquote{skinoptics.colors.}}\sphinxbfcode{\sphinxupquote{chroma}}}{\sphinxparam{\DUrole{n}{a}}\sphinxparamcomma \sphinxparam{\DUrole{n}{b}}}{}
\pysigstopsignatures
\sphinxAtStartPar
Calculate the chroma C* from a* and b* coordinates.

\sphinxAtStartPar
\(C^* = \sqrt{a^{*2} + b^{*2}}\)
\begin{quote}\begin{description}
\sphinxlineitem{Parameters}
\sphinxAtStartPar
\sphinxstyleliteralstrong{\sphinxupquote{a}} (\sphinxstyleliteralemphasis{\sphinxupquote{float}}\sphinxstyleliteralemphasis{\sphinxupquote{ or }}\sphinxstyleliteralemphasis{\sphinxupquote{np.ndarray}}) \textendash{} a* coordinate {[}\sphinxhyphen{}{]}

\end{description}\end{quote}

\sphinxAtStartPar
b: float or np.ndarray
b* coordinate {[}\sphinxhyphen{}{]}


\subsubsection{Returns}
\label{\detokenize{07_colors:id22}}
\sphinxAtStartPar
chroma: float or np.ndarray
chroma {[}\sphinxhyphen{}{]}

\end{fulllineitems}

\index{cmfs() (in module skinoptics.colors)@\spxentry{cmfs()}\spxextra{in module skinoptics.colors}}

\begin{fulllineitems}
\phantomsection\label{\detokenize{07_colors:skinoptics.colors.cmfs}}
\pysigstartsignatures
\pysiglinewithargsret{\sphinxcode{\sphinxupquote{skinoptics.colors.}}\sphinxbfcode{\sphinxupquote{cmfs}}}{\sphinxparam{\DUrole{n}{lambda0}}\sphinxparamcomma \sphinxparam{\DUrole{n}{observer}}\sphinxparamcomma \sphinxparam{\DUrole{n}{cmfs\_model}\DUrole{o}{=}\DUrole{default_value}{\textquotesingle{}CIE\textquotesingle{}}}}{}
\pysigstopsignatures
\begin{DUlineblock}{0em}
\item[] The CIE color\sphinxhyphen{}matching functions \(\bar{x}(\lambda)\), \(\bar{y}(\lambda)\) and \(\bar{z}(\lambda)\) 
\item[] for a chosen standard observer as a function of wavelength.
\end{DUlineblock}

\begin{DUlineblock}{0em}
\item[] wavelength range: {[}360 nm, 830 nm{]} (at 1 nm intervals for cmfs\_model = ‘CIE’)
\end{DUlineblock}
\begin{quote}\begin{description}
\sphinxlineitem{Parameters}\begin{itemize}
\item {} 
\sphinxAtStartPar
\sphinxstyleliteralstrong{\sphinxupquote{lambda0}} (\sphinxstyleliteralemphasis{\sphinxupquote{float}}\sphinxstyleliteralemphasis{\sphinxupquote{ or }}\sphinxstyleliteralemphasis{\sphinxupquote{np.ndarray}}) \textendash{} wavelength {[}nm{]} (must be in range {[}360., 830.{]} for cmfs\_model = ‘CIE’)

\item {} 
\sphinxAtStartPar
\sphinxstyleliteralstrong{\sphinxupquote{observer}} (\sphinxstyleliteralemphasis{\sphinxupquote{str}}) \textendash{} the user can choose one of the following… ‘2o’ or ‘10o’

\item {} 
\sphinxAtStartPar
\sphinxstyleliteralstrong{\sphinxupquote{cmfs\_model}} (\sphinxstyleliteralemphasis{\sphinxupquote{str}}\sphinxstyleliteralemphasis{\sphinxupquote{ (}}\sphinxstyleliteralemphasis{\sphinxupquote{default to \textquotesingle{}CIE\textquotesingle{}}}\sphinxstyleliteralemphasis{\sphinxupquote{)}}) \textendash{} the user can choose one of the following… ‘CIE’, ‘Wyman\_singlelobe’ or ‘Wyman\_multilobe’

\end{itemize}

\end{description}\end{quote}

\begin{DUlineblock}{0em}
\item[] ‘2o’ refers to the CIE 1931 2 degree standard observer
\item[] ‘10o’ refers to the CIE 1964 10 degree standard observer
\end{DUlineblock}

\begin{DUlineblock}{0em}
\item[] ‘CIE’ for the linear interpolation of data from CIE datasets {[}CIE19a{]} {[}CIE19b{]}
\item[] ‘Wyman\_singlelobe’ for the functions from Wyman, Sloan \& Shirley 2013 {[}WSS13{]} (section 2.1)
\item[] ‘Wyman\_multilobe’ for the functions from Wyman, Sloan \& Shirley 2013 {[}WSS13{]} (section 2.2)
\end{DUlineblock}
\begin{quote}\begin{description}
\sphinxlineitem{Returns}
\sphinxAtStartPar
\begin{itemize}
\item {} 
\sphinxAtStartPar
\sphinxstylestrong{xbar} (\sphinxstyleemphasis{float or np.ndarray}) \textendash{} \(\bar{x}(\lambda\)) color\sphinxhyphen{}matching function {[}\sphinxhyphen{}{]}

\item {} 
\sphinxAtStartPar
\sphinxstylestrong{ybar} (\sphinxstyleemphasis{float or np.ndarray}) \textendash{} \(\bar{y}(\lambda\)) color\sphinxhyphen{}matching function {[}\sphinxhyphen{}{]}

\item {} 
\sphinxAtStartPar
\sphinxstylestrong{zbar} (\sphinxstyleemphasis{float or np.ndarray}) \textendash{} \(\bar{z}(\lambda\)) color\sphinxhyphen{}matching function {[}\sphinxhyphen{}{]}

\end{itemize}


\end{description}\end{quote}

\end{fulllineitems}

\index{f\_Lab\_from\_XYZ() (in module skinoptics.colors)@\spxentry{f\_Lab\_from\_XYZ()}\spxextra{in module skinoptics.colors}}

\begin{fulllineitems}
\phantomsection\label{\detokenize{07_colors:skinoptics.colors.f_Lab_from_XYZ}}
\pysigstartsignatures
\pysiglinewithargsret{\sphinxcode{\sphinxupquote{skinoptics.colors.}}\sphinxbfcode{\sphinxupquote{f\_Lab\_from\_XYZ}}}{\sphinxparam{\DUrole{n}{u}}}{}
\pysigstopsignatures
\sphinxAtStartPar
The function f(u) used to calculate CIE L*a*b* coordinates from CIE XYZ coordinates (see function Lab\_from\_XYZ).

\sphinxAtStartPar
\(f(u) = \left\{ 
\begin{matrix}
\sqrt[3]{u}, & \mbox{if }  u > \left(\frac{6}{29}\right)^3 \\
\frac{1}{3}\left(\frac{29}{6}\right)^2 u + \frac{4}{29}, & \mbox{if }  u \le \left(\frac{6}{29}\right)^3
\end{matrix}\right.\)


\subsubsection{Parameters}
\label{\detokenize{07_colors:id23}}
\sphinxAtStartPar
u: float or np.ndarray
X/Xn, Y/Yn or Z/Zn ratio{[}\sphinxhyphen{}{]}


\subsubsection{Returns}
\label{\detokenize{07_colors:id24}}
\sphinxAtStartPar
f: float or np.ndarray
evaluated function {[}\sphinxhyphen{}{]}

\end{fulllineitems}

\index{hue() (in module skinoptics.colors)@\spxentry{hue()}\spxextra{in module skinoptics.colors}}

\begin{fulllineitems}
\phantomsection\label{\detokenize{07_colors:skinoptics.colors.hue}}
\pysigstartsignatures
\pysiglinewithargsret{\sphinxcode{\sphinxupquote{skinoptics.colors.}}\sphinxbfcode{\sphinxupquote{hue}}}{\sphinxparam{\DUrole{n}{a}}\sphinxparamcomma \sphinxparam{\DUrole{n}{b}}}{}
\pysigstopsignatures
\sphinxAtStartPar
Calculate the hue angle h* from a* and b* coordinates.

\sphinxAtStartPar
\(h^* = \arctan 2\left(\frac{b^*}{a^*}\right)\frac{180}{\pi}\)


\subsubsection{Parameters}
\label{\detokenize{07_colors:id25}}
\sphinxAtStartPar
a: float or np.ndarray
a* coordinate {[}\sphinxhyphen{}{]}

\sphinxAtStartPar
b: float or np.ndarray
b* coordinate {[}\sphinxhyphen{}{]}


\subsubsection{Returns}
\label{\detokenize{07_colors:id26}}
\sphinxAtStartPar
hue: float or np.ndarray
hue angle (in range {[}0, 360{]}) {[}degrees{]}

\end{fulllineitems}

\index{inv\_f\_Lab\_from\_XYZ() (in module skinoptics.colors)@\spxentry{inv\_f\_Lab\_from\_XYZ()}\spxextra{in module skinoptics.colors}}

\begin{fulllineitems}
\phantomsection\label{\detokenize{07_colors:skinoptics.colors.inv_f_Lab_from_XYZ}}
\pysigstartsignatures
\pysiglinewithargsret{\sphinxcode{\sphinxupquote{skinoptics.colors.}}\sphinxbfcode{\sphinxupquote{inv\_f\_Lab\_from\_XYZ}}}{\sphinxparam{\DUrole{n}{u}}}{}
\pysigstopsignatures
\sphinxAtStartPar
The \(f^{-1}(u)\) function, i.e. the inverse of the \(f(u)\) function (see function f\_Lab\_from\_XYZ).

\sphinxAtStartPar
\(f^{-1}(u) = \left\{ 
\begin{matrix}
u^3, & \mbox{if }  u > \frac{6}{29} \\
3\left(\frac{6}{29}\right)^2\left(u - \frac{4}{29} \right), & \mbox{if } u \le \frac{6}{29}
\end{matrix}\right.\)
\begin{quote}\begin{description}
\sphinxlineitem{Parameters}
\sphinxAtStartPar
\sphinxstyleliteralstrong{\sphinxupquote{u}} (\sphinxstyleliteralemphasis{\sphinxupquote{float}}\sphinxstyleliteralemphasis{\sphinxupquote{ or }}\sphinxstyleliteralemphasis{\sphinxupquote{np.ndarray}}) \textendash{} function variable {[}\sphinxhyphen{}{]}

\sphinxlineitem{Returns}
\sphinxAtStartPar
\begin{itemize}
\item {} 
\sphinxAtStartPar
\sphinxstylestrong{f} (\sphinxstyleemphasis{float or np.ndarray}) \textendash{} evaluated function {[}\sphinxhyphen{}{]}

\end{itemize}


\end{description}\end{quote}

\end{fulllineitems}

\index{inv\_nonlinear\_corr\_sRGB() (in module skinoptics.colors)@\spxentry{inv\_nonlinear\_corr\_sRGB()}\spxextra{in module skinoptics.colors}}

\begin{fulllineitems}
\phantomsection\label{\detokenize{07_colors:skinoptics.colors.inv_nonlinear_corr_sRGB}}
\pysigstartsignatures
\pysiglinewithargsret{\sphinxcode{\sphinxupquote{skinoptics.colors.}}\sphinxbfcode{\sphinxupquote{inv\_nonlinear\_corr\_sRGB}}}{\sphinxparam{\DUrole{n}{u}}}{}
\pysigstopsignatures
\sphinxAtStartPar
The inverse nonlinear correction for sRGB coordinates.

\sphinxAtStartPar
\(\gamma^{-1}(u) =  
\left \{ \begin{matrix}
u/12.92, & \mbox{if } u \le 0.04045 \\
[(u + 0.055)/1.055]^{2.4}, & \mbox{if } u > 0.04045 \\
\end{matrix} \right.\)
\begin{quote}\begin{description}
\sphinxlineitem{Parameters}
\sphinxAtStartPar
\sphinxstyleliteralstrong{\sphinxupquote{u}} (\sphinxstyleliteralemphasis{\sphinxupquote{float}}\sphinxstyleliteralemphasis{\sphinxupquote{ or }}\sphinxstyleliteralemphasis{\sphinxupquote{np.ndarray}}) \textendash{} nonlinear R, G or B coordinate {[}\sphinxhyphen{}{]}

\sphinxlineitem{Returns}
\sphinxAtStartPar
\begin{itemize}
\item {} 
\sphinxAtStartPar
\sphinxstylestrong{inv\_gamma} (\sphinxstyleemphasis{float or np.ndarray}) \textendash{} linear R, G or B coordinate {[}\sphinxhyphen{}{]}

\end{itemize}


\end{description}\end{quote}

\end{fulllineitems}

\index{nonlinear\_corr\_sRGB() (in module skinoptics.colors)@\spxentry{nonlinear\_corr\_sRGB()}\spxextra{in module skinoptics.colors}}

\begin{fulllineitems}
\phantomsection\label{\detokenize{07_colors:skinoptics.colors.nonlinear_corr_sRGB}}
\pysigstartsignatures
\pysiglinewithargsret{\sphinxcode{\sphinxupquote{skinoptics.colors.}}\sphinxbfcode{\sphinxupquote{nonlinear\_corr\_sRGB}}}{\sphinxparam{\DUrole{n}{u}}}{}
\pysigstopsignatures
\sphinxAtStartPar
The nonlinear correction for sRGB coordinates.

\sphinxAtStartPar
\(\gamma(u) =  
\left \{ \begin{matrix}
12.92 u, & \mbox{if } u \le 0.0031308 \\
1.055 u^{1/2.4} - 0.055, & \mbox{if } u > 0.0031308 \\
\end{matrix} \right.\)
\begin{quote}\begin{description}
\sphinxlineitem{Parameters}
\sphinxAtStartPar
\sphinxstyleliteralstrong{\sphinxupquote{u}} (\sphinxstyleliteralemphasis{\sphinxupquote{float}}\sphinxstyleliteralemphasis{\sphinxupquote{ or }}\sphinxstyleliteralemphasis{\sphinxupquote{np.ndarray}}) \textendash{} linear R, G or B coordinate {[}\sphinxhyphen{}{]}

\sphinxlineitem{Returns}
\sphinxAtStartPar
\begin{itemize}
\item {} 
\sphinxAtStartPar
\sphinxstylestrong{gamma} (\sphinxstyleemphasis{float or np.ndarray}) \textendash{} nonlinear R, G or B coordinate {[}\sphinxhyphen{}{]}

\end{itemize}


\end{description}\end{quote}

\end{fulllineitems}

\index{rspd() (in module skinoptics.colors)@\spxentry{rspd()}\spxextra{in module skinoptics.colors}}

\begin{fulllineitems}
\phantomsection\label{\detokenize{07_colors:skinoptics.colors.rspd}}
\pysigstartsignatures
\pysiglinewithargsret{\sphinxcode{\sphinxupquote{skinoptics.colors.}}\sphinxbfcode{\sphinxupquote{rspd}}}{\sphinxparam{\DUrole{n}{lambda0}}\sphinxparamcomma \sphinxparam{\DUrole{n}{illuminant}}}{}
\pysigstopsignatures
\begin{DUlineblock}{0em}
\item[] The relative spectral power distribution S(\(\lambda\)) of a chosen standard illuminant
\item[] as a function of wavelength.
\item[] Linear interpolation of data from CIE datasets {[}CIE18a{]} {[}CIE22a{]} {[}CIE18b{]} {[}CIE22b{]} {[}CIE18c{]}.
\end{DUlineblock}

\begin{DUlineblock}{0em}
\item[] wavelength range:
\item[] {[}300 nm, 830 nm{]} (at 1 nm intervals, for illuminant = ‘A’, ‘D50’ or ‘D65’)
\item[] or {[}300 nm, 780 nm{]} (at 5 nm intervals, for illuminant = ‘D55’ or ‘D75’)
\end{DUlineblock}
\begin{quote}\begin{description}
\sphinxlineitem{Parameters}\begin{itemize}
\item {} 
\sphinxAtStartPar
\sphinxstyleliteralstrong{\sphinxupquote{lambda0}} (\sphinxstyleliteralemphasis{\sphinxupquote{float}}\sphinxstyleliteralemphasis{\sphinxupquote{ or }}\sphinxstyleliteralemphasis{\sphinxupquote{np.ndarray}}) \textendash{} wavelength {[}nm{]} (must be in range {[}300 nm, 830 nm{]} or {[}300 nm, 780 nm{]})

\item {} 
\sphinxAtStartPar
\sphinxstyleliteralstrong{\sphinxupquote{illuminant}} (\sphinxstyleliteralemphasis{\sphinxupquote{str}}) \textendash{} the user can choose one of the following… ‘A’, ‘D50’, ‘D55’, ‘D65’ or ‘D75’

\end{itemize}

\end{description}\end{quote}

\begin{DUlineblock}{0em}
\item[] ‘A’ refers to the CIE standard illuminant A
\item[] ‘D50’ refers to the CIE standard illuminant D50
\item[] ‘D55’ refers to the CIE standard illuminant D55
\item[] ‘D65’ refers to the CIE standard illuminant D65
\item[] ‘D75’ refers to the CIE standard illuminant D75
\end{DUlineblock}
\begin{quote}\begin{description}
\sphinxlineitem{Returns}
\sphinxAtStartPar
\begin{itemize}
\item {} 
\sphinxAtStartPar
\sphinxstylestrong{rspd} (\sphinxstyleemphasis{float or np.ndarray}) \textendash{} relative spectral power distribution {[}\sphinxhyphen{}{]}

\end{itemize}


\end{description}\end{quote}

\end{fulllineitems}

\index{sRGB\_from\_XYZ() (in module skinoptics.colors)@\spxentry{sRGB\_from\_XYZ()}\spxextra{in module skinoptics.colors}}

\begin{fulllineitems}
\phantomsection\label{\detokenize{07_colors:skinoptics.colors.sRGB_from_XYZ}}
\pysigstartsignatures
\pysiglinewithargsret{\sphinxcode{\sphinxupquote{skinoptics.colors.}}\sphinxbfcode{\sphinxupquote{sRGB\_from\_XYZ}}}{\sphinxparam{\DUrole{n}{X}}\sphinxparamcomma \sphinxparam{\DUrole{n}{Y}}\sphinxparamcomma \sphinxparam{\DUrole{n}{Z}}\sphinxparamcomma \sphinxparam{\DUrole{n}{K}\DUrole{o}{=}\DUrole{default_value}{1.0}}\sphinxparamcomma \sphinxparam{\DUrole{n}{sRGB\_scale}\DUrole{o}{=}\DUrole{default_value}{\textquotesingle{}norm\textquotesingle{}}}}{}
\pysigstopsignatures
\begin{DUlineblock}{0em}
\item[] Calculate sRGB coordinates from CIE XYZ coordinates.
\item[] CIE XYZ coordinates must be for the standard illuminant D65 and the 2 degree standard observer.
\item[] For details please check Stokes et al. {[}S*96{]} and IEC {[}IEC99{]}.
\end{DUlineblock}

\sphinxAtStartPar
\(\begin{bmatrix}
R \\
G \\
B
\end{bmatrix}
=
\begin{bmatrix}
\gamma(R_{linear}) \\
\gamma(G_{linear}) \\
\gamma(B_{linear})
\end{bmatrix}\)

\sphinxAtStartPar
in which

\sphinxAtStartPar
\(\begin{bmatrix}
R_{linear} \\
G_{linear} \\
B_{linear}
\end{bmatrix}
=
\mathcal{M}
\begin{bmatrix}
X \\
Y \\
Z
\end{bmatrix}\)

\sphinxAtStartPar
and

\sphinxAtStartPar
\(\gamma(u) =  
\left \{ \begin{matrix}
12.92 u, & \mbox{if } u \le 0.0031308 \\
1.055 u^{1/2.4} - 0.055, & \mbox{if } u > 0.0031308 \\
\end{matrix} \right.\)
\begin{quote}\begin{description}
\sphinxlineitem{Parameters}\begin{itemize}
\item {} 
\sphinxAtStartPar
\sphinxstyleliteralstrong{\sphinxupquote{X}} (\sphinxstyleliteralemphasis{\sphinxupquote{float}}\sphinxstyleliteralemphasis{\sphinxupquote{ or }}\sphinxstyleliteralemphasis{\sphinxupquote{np.ndarray}}) \textendash{} X coordinate {[}\sphinxhyphen{}{]}

\item {} 
\sphinxAtStartPar
\sphinxstyleliteralstrong{\sphinxupquote{Y}} (\sphinxstyleliteralemphasis{\sphinxupquote{float}}\sphinxstyleliteralemphasis{\sphinxupquote{ or }}\sphinxstyleliteralemphasis{\sphinxupquote{np.ndarray}}) \textendash{} Y coordinate {[}\sphinxhyphen{}{]}

\item {} 
\sphinxAtStartPar
\sphinxstyleliteralstrong{\sphinxupquote{Z}} (\sphinxstyleliteralemphasis{\sphinxupquote{float}}\sphinxstyleliteralemphasis{\sphinxupquote{ or }}\sphinxstyleliteralemphasis{\sphinxupquote{np.ndarray}}) \textendash{} Z coordinate {[}\sphinxhyphen{}{]}

\item {} 
\sphinxAtStartPar
\sphinxstyleliteralstrong{\sphinxupquote{K}} (\sphinxstyleliteralemphasis{\sphinxupquote{float}}) \textendash{} scaling factor (usually 1. or 100.) {[}\sphinxhyphen{}{]} (default to 1.)

\item {} 
\sphinxAtStartPar
\sphinxstyleliteralstrong{\sphinxupquote{sRGB\_scale}} \textendash{} the user can choose one of the following… ‘norm’ or ‘8bit’

\end{itemize}

\end{description}\end{quote}

\begin{DUlineblock}{0em}
\item[] K = 1. for CIE XYZ coordinates in range {[}0, 1{]}
\item[] K = 100. for CIE XYZ coordinates in range {[}0, 100{]}
\end{DUlineblock}

\begin{DUlineblock}{0em}
\item[] ‘norm’ for sRGB coordinates in range {[}0,1{]} (normalized scale)
\item[] ‘8bit’ for sRGB coordinates in range {[}0, 255{]} (8\sphinxhyphen{}bit scale)
\end{DUlineblock}
\begin{quote}\begin{description}
\sphinxlineitem{Returns}
\sphinxAtStartPar
\begin{itemize}
\item {} 
\sphinxAtStartPar
\sphinxstylestrong{R} (\sphinxstyleemphasis{float or np.ndarray}) \textendash{} R coordinate {[}\sphinxhyphen{}{]}

\item {} 
\sphinxAtStartPar
\sphinxstylestrong{G} (\sphinxstyleemphasis{float or np.ndarray}) \textendash{} G coordinate {[}\sphinxhyphen{}{]}

\item {} 
\sphinxAtStartPar
\sphinxstylestrong{B} (\sphinxstyleemphasis{float or np.ndarray}) \textendash{} B coordinate {[}\sphinxhyphen{}{]}

\end{itemize}


\end{description}\end{quote}

\end{fulllineitems}

\index{sRGB\_from\_lambda0() (in module skinoptics.colors)@\spxentry{sRGB\_from\_lambda0()}\spxextra{in module skinoptics.colors}}

\begin{fulllineitems}
\phantomsection\label{\detokenize{07_colors:skinoptics.colors.sRGB_from_lambda0}}
\pysigstartsignatures
\pysiglinewithargsret{\sphinxcode{\sphinxupquote{skinoptics.colors.}}\sphinxbfcode{\sphinxupquote{sRGB\_from\_lambda0}}}{\sphinxparam{\DUrole{n}{lambda0}}\sphinxparamcomma \sphinxparam{\DUrole{n}{cmfs\_model}\DUrole{o}{=}\DUrole{default_value}{\textquotesingle{}CIE\textquotesingle{}}}\sphinxparamcomma \sphinxparam{\DUrole{n}{sRGB\_scale}\DUrole{o}{=}\DUrole{default_value}{\textquotesingle{}norm\textquotesingle{}}}}{}
\pysigstopsignatures
\sphinxAtStartPar
Calculate the sRGB coordinates respective to the color of a monochromatic light (single wavelength).

\sphinxAtStartPar
wavelength range: {[}360 nm, 830 nm{]}


\subsubsection{Parameters}
\label{\detokenize{07_colors:id27}}
\sphinxAtStartPar
lambda0: float or np.ndarray
wavelength of the monochromatic light

\sphinxAtStartPar
cmfs\_model: str (default to ‘CIE’)
the user can choose one of the following… ‘CIE’, ‘Wyman\_singlelobe’ or ‘Wyman\_multilobe’
‘CIE’ for the linear interpolation of data from CIE datasets {[}CIE19a{]} {[}CIE19b{]}
‘Wyman\_singlelobe’ for the analytical functions from Wyman, Sloan \& Shirley 2013 {[}WSS13{]} (section 2.1)
‘Wyman\_multilobe’ for the analytical functions from Wyman, Sloan \& Shirley 2013 {[}WSS13{]} (section 2.2)

\sphinxAtStartPar
sRGB\_scale: str (default to ‘norm’)
the user can choose one of the following… ‘norm’ or ‘8bit’
‘norm’ for sRGB coordinates in range {[}0,1{]} (normalized scale)
‘8bit’ for sRGB coordinates in range {[}0, 255{]} (8\sphinxhyphen{}bit scale)

\end{fulllineitems}

\index{sRGB\_from\_spectrum() (in module skinoptics.colors)@\spxentry{sRGB\_from\_spectrum()}\spxextra{in module skinoptics.colors}}

\begin{fulllineitems}
\phantomsection\label{\detokenize{07_colors:skinoptics.colors.sRGB_from_spectrum}}
\pysigstartsignatures
\pysiglinewithargsret{\sphinxcode{\sphinxupquote{skinoptics.colors.}}\sphinxbfcode{\sphinxupquote{sRGB\_from\_spectrum}}}{\sphinxparam{\DUrole{n}{all\_lambda}}\sphinxparamcomma \sphinxparam{\DUrole{n}{spectrum}}\sphinxparamcomma \sphinxparam{\DUrole{n}{lambda\_min}\DUrole{o}{=}\DUrole{default_value}{360}}\sphinxparamcomma \sphinxparam{\DUrole{n}{lambda\_max}\DUrole{o}{=}\DUrole{default_value}{830}}\sphinxparamcomma \sphinxparam{\DUrole{n}{lambda\_step}\DUrole{o}{=}\DUrole{default_value}{1}}\sphinxparamcomma \sphinxparam{\DUrole{n}{cmfs\_model}\DUrole{o}{=}\DUrole{default_value}{\textquotesingle{}CIE\textquotesingle{}}}\sphinxparamcomma \sphinxparam{\DUrole{n}{interp1d\_kind}\DUrole{o}{=}\DUrole{default_value}{\textquotesingle{}cubic\textquotesingle{}}}\sphinxparamcomma \sphinxparam{\DUrole{n}{sRGB\_scale}\DUrole{o}{=}\DUrole{default_value}{\textquotesingle{}norm\textquotesingle{}}}}{}
\pysigstopsignatures
\sphinxAtStartPar
Calculate the sRGB coordinates from the reflectance or the transmittance spectrum.
First calculate CIE XYZ coordinates (respective to the standard illuminant D65 and
the 2 degree standard observer) from the spectrum and then calculate sRGB coordinates from
CIE XYZ coordinates (see functions sRGB\_from\_XYZ and XYZ\_from\_spectrum).


\subsubsection{Parameters}
\label{\detokenize{07_colors:id28}}
\sphinxAtStartPar
all\_lambda: np.ndarray
wavelength array

\sphinxAtStartPar
spectrum: np.ndarray
reflectance or transmittance spectrum respective to the wavelength array {[}\%{]}

\sphinxAtStartPar
lambda\_min: float (default to 360.)
lower limit of summation/integration (minimum wavelength to take into account) {[}nm{]}

\sphinxAtStartPar
lambda\_max: float (default to 830.)
upper limit of summation/integration (maximum wavelength to take into account) {[}nm{]}

\sphinxAtStartPar
lambda\_step: float (default to 1.)
summation interval (wavelength step) {[}nm{]}

\sphinxAtStartPar
cmfs\_model: str (default to ‘CIE’)
the user can choose one of the following… ‘CIE’, ‘Wyman\_singlelobe’ or ‘Wyman\_multilobe’
‘CIE’ for the linear interpolation of data from CIE datasets {[}CIE19a{]} {[}CIE19b{]}
‘Wyman\_singlelobe’ for the analytical functions from Wyman, Sloan \& Shirley 2013 {[}WSS13{]} (section 2.1)
‘Wyman\_multilobe’ for the analytical functions from Wyman, Sloan \& Shirley 2013 {[}WSS13{]} (section 2.2)

\sphinxAtStartPar
interp1d\_kind: str (default to ‘cubic’)
kind argument of scipy.interpolation.interp1d (see scipy documentation)
default to ‘cubic’ as recommended by CIE {[}CIE04{]} (see section 7.2.1.1)

\sphinxAtStartPar
sRGB\_scale: str (default to ‘norm’)
the user can choose one of the following… ‘norm’ or ‘8bit’
‘norm’ for sRGB coordinates in range {[}0,1{]} (normalized scale)
‘8bit’ for sRGB coordinates in range {[}0, 255{]} (8\sphinxhyphen{}bit scale)


\subsubsection{Returns}
\label{\detokenize{07_colors:id29}}
\sphinxAtStartPar
R: float
R coordinate {[}\sphinxhyphen{}{]}

\sphinxAtStartPar
G: float
G coordinate {[}\sphinxhyphen{}{]}

\sphinxAtStartPar
B: float
B coordinate {[}\sphinxhyphen{}{]}

\end{fulllineitems}

\index{transf\_matrix\_sRGB\_linear\_from\_XYZ() (in module skinoptics.colors)@\spxentry{transf\_matrix\_sRGB\_linear\_from\_XYZ()}\spxextra{in module skinoptics.colors}}

\begin{fulllineitems}
\phantomsection\label{\detokenize{07_colors:skinoptics.colors.transf_matrix_sRGB_linear_from_XYZ}}
\pysigstartsignatures
\pysiglinewithargsret{\sphinxcode{\sphinxupquote{skinoptics.colors.}}\sphinxbfcode{\sphinxupquote{transf\_matrix\_sRGB\_linear\_from\_XYZ}}}{}{}
\pysigstopsignatures
\sphinxAtStartPar
The transformation matrix employed to obtain linear sRGB coordinates from CIE XYZ coordinates.

\sphinxAtStartPar
\(\mathcal{M} = 
\begin{bmatrix}
3.24062 & -1.5372 & -0.4986 \\
-0.9689 & 1.8758 & 0.0415 \\
0.0557 & -0.2040 & 1.0570
\end{bmatrix}\)
\begin{quote}\begin{description}
\sphinxlineitem{Returns}
\sphinxAtStartPar
\begin{itemize}
\item {} 
\sphinxAtStartPar
\sphinxstylestrong{M} (\sphinxstyleemphasis{np.ndarray}) \textendash{} transformation matrix

\end{itemize}


\end{description}\end{quote}

\end{fulllineitems}

\index{xy\_from\_XYZ() (in module skinoptics.colors)@\spxentry{xy\_from\_XYZ()}\spxextra{in module skinoptics.colors}}

\begin{fulllineitems}
\phantomsection\label{\detokenize{07_colors:skinoptics.colors.xy_from_XYZ}}
\pysigstartsignatures
\pysiglinewithargsret{\sphinxcode{\sphinxupquote{skinoptics.colors.}}\sphinxbfcode{\sphinxupquote{xy\_from\_XYZ}}}{\sphinxparam{\DUrole{n}{X}}\sphinxparamcomma \sphinxparam{\DUrole{n}{Y}}\sphinxparamcomma \sphinxparam{\DUrole{n}{Z}}}{}
\pysigstopsignatures
\begin{DUlineblock}{0em}
\item[] Calculate CIE xy chromaticities from CIE XYZ coordinates.
\end{DUlineblock}

\begin{DUlineblock}{0em}
\item[] \(x = \frac{X}{X + Y + Z}\)
\item[] \(y = \frac{Y}{X + Y + Z}\)
\end{DUlineblock}
\begin{quote}\begin{description}
\sphinxlineitem{Parameters}\begin{itemize}
\item {} 
\sphinxAtStartPar
\sphinxstyleliteralstrong{\sphinxupquote{X}} (\sphinxstyleliteralemphasis{\sphinxupquote{float}}\sphinxstyleliteralemphasis{\sphinxupquote{ or }}\sphinxstyleliteralemphasis{\sphinxupquote{np.ndarray}}) \textendash{} X coordinate {[}\sphinxhyphen{}{]}

\item {} 
\sphinxAtStartPar
\sphinxstyleliteralstrong{\sphinxupquote{Y}} (\sphinxstyleliteralemphasis{\sphinxupquote{float}}\sphinxstyleliteralemphasis{\sphinxupquote{ or }}\sphinxstyleliteralemphasis{\sphinxupquote{np.ndarray}}) \textendash{} Y coordinate {[}\sphinxhyphen{}{]}

\item {} 
\sphinxAtStartPar
\sphinxstyleliteralstrong{\sphinxupquote{Z}} (\sphinxstyleliteralemphasis{\sphinxupquote{float}}\sphinxstyleliteralemphasis{\sphinxupquote{ or }}\sphinxstyleliteralemphasis{\sphinxupquote{np.ndarray}}) \textendash{} Z coordinate {[}\sphinxhyphen{}{]}

\end{itemize}

\sphinxlineitem{Returns}
\sphinxAtStartPar
\begin{itemize}
\item {} 
\sphinxAtStartPar
\sphinxstylestrong{x} (\sphinxstyleemphasis{float or np.ndarray}) \textendash{} CIE x chromaticity {[}\sphinxhyphen{}{]}

\item {} 
\sphinxAtStartPar
\sphinxstylestrong{y} (\sphinxstyleemphasis{float or np.ndarray}) \textendash{} CIE y chromaticity {[}\sphinxhyphen{}{]}

\end{itemize}


\end{description}\end{quote}

\end{fulllineitems}

\index{xy\_wp() (in module skinoptics.colors)@\spxentry{xy\_wp()}\spxextra{in module skinoptics.colors}}

\begin{fulllineitems}
\phantomsection\label{\detokenize{07_colors:skinoptics.colors.xy_wp}}
\pysigstartsignatures
\pysiglinewithargsret{\sphinxcode{\sphinxupquote{skinoptics.colors.}}\sphinxbfcode{\sphinxupquote{xy\_wp}}}{\sphinxparam{\DUrole{n}{illuminant}}\sphinxparamcomma \sphinxparam{\DUrole{n}{observer}}}{}
\pysigstopsignatures
\begin{DUlineblock}{0em}
\item[] The white point CIE xy chromaticities for a chosen standard illuminant and standard observer.
\item[] Calculated from the white point CIE XYZ coordinates (see function XYZ\_wp).
\end{DUlineblock}
\begin{quote}\begin{description}
\sphinxlineitem{Parameters}\begin{itemize}
\item {} 
\sphinxAtStartPar
\sphinxstyleliteralstrong{\sphinxupquote{illuminant}} (\sphinxstyleliteralemphasis{\sphinxupquote{str}}) \textendash{} the user can choose one of the following… ‘A’, ‘D50’, ‘D55’, ‘D65’ or ‘D75’

\item {} 
\sphinxAtStartPar
\sphinxstyleliteralstrong{\sphinxupquote{observer}} (\sphinxstyleliteralemphasis{\sphinxupquote{str}}) \textendash{} the user can choose one of the following… ‘2o’ or ‘10o’

\end{itemize}

\end{description}\end{quote}

\begin{DUlineblock}{0em}
\item[] ‘A’ refers to the CIE standard illuminant A
\item[] ‘D50’ refers to the CIE standard illuminant D50
\item[] ‘D55’ refers to the CIE standard illuminant D55
\item[] ‘D65’ refers to the CIE standard illuminant D65
\item[] ‘D75’ refers to the CIE standard illuminant D75
\end{DUlineblock}

\begin{DUlineblock}{0em}
\item[] ‘2o’ refers to the CIE 1931 2 degree standard observer
\item[] ‘10o’ refers to the CIE 1964 10 degree standard observer
\end{DUlineblock}
\begin{quote}\begin{description}
\sphinxlineitem{Returns}
\sphinxAtStartPar
\begin{itemize}
\item {} 
\sphinxAtStartPar
\sphinxstylestrong{xn} (\sphinxstyleemphasis{float}) \textendash{} white point CIE x chromaticity {[}\sphinxhyphen{}{]}

\item {} 
\sphinxAtStartPar
\sphinxstylestrong{yn} (\sphinxstyleemphasis{float}) \textendash{} white point CIE x chromaticity {[}\sphinxhyphen{}{]}

\end{itemize}


\end{description}\end{quote}

\end{fulllineitems}



\chapter{Installation}
\label{\detokenize{index:installation}}
\sphinxAtStartPar
\sphinxstylestrong{SkinOptics} can be installed using \sphinxcode{\sphinxupquote{pip}} in the command prompt:

\begin{sphinxVerbatim}[commandchars=\\\{\}]
\PYG{g+gp}{\PYGZdl{} }pip\PYG{+w}{ }install\PYG{+w}{ }skinoptics
\end{sphinxVerbatim}


\chapter{Prerequisite Packages}
\label{\detokenize{index:prerequisite-packages}}
\begin{DUlineblock}{0em}
\item[] \sphinxstylestrong{SkinOptics} is structured based on other Python packages.
\item[] In order to be able to properly run it, please also install all of the following…
\end{DUlineblock}
\begin{itemize}
\item {} 
\sphinxAtStartPar
\sphinxhref{https://numpy.org/}{Numpy}

\item {} 
\sphinxAtStartPar
\sphinxhref{https://scipy.org/}{SciPy}

\item {} 
\sphinxAtStartPar
\sphinxhref{https://pandas.pydata.org/}{pandas}

\end{itemize}


\chapter{Modules}
\label{\detokenize{index:modules}}
\sphinxAtStartPar
\sphinxstylestrong{SkinOptics} is currently composed of seven modules:

\begin{DUlineblock}{0em}
\item[] \sphinxhyphen{} \sphinxstylestrong{utils.py}
\item[] Module with mathematical and auxiliary functions.
\end{DUlineblock}

\begin{DUlineblock}{0em}
\item[] \sphinxhyphen{} \sphinxstylestrong{dataframes.py}
\item[] Module with pandas DataFrames of the .txt and .csv files stored at the “datasets” folder.
\end{DUlineblock}

\begin{DUlineblock}{0em}
\item[] \sphinxhyphen{} \sphinxstylestrong{absorption\_coefficient.py}
\item[] Module with functions for modeling the absorption coefficient and calculating related quantities.
\end{DUlineblock}

\begin{DUlineblock}{0em}
\item[] \sphinxhyphen{} \sphinxstylestrong{scattering\_cofficient.py}
\item[] module with functions for modeling the scattering coefficient and calculating related quantities
\end{DUlineblock}

\begin{DUlineblock}{0em}
\item[] \sphinxhyphen{} \sphinxstylestrong{refractive\_index.py}
\item[] module with functions for modeling the refractive index and calculating related quantities
\end{DUlineblock}

\begin{DUlineblock}{0em}
\item[] \sphinxhyphen{} \sphinxstylestrong{anysotropy\_factor.py}
\item[] module with functions for modeling the anisotropy factor and calculating related quantities
\end{DUlineblock}

\begin{DUlineblock}{0em}
\item[] \sphinxhyphen{} \sphinxstylestrong{colors.py}
\item[] module with functions for \#fixme
\end{DUlineblock}


\chapter{Tutorials}
\label{\detokenize{index:tutorials}}\begin{itemize}
\item {} 
\sphinxAtStartPar
tutorial\_optical\_properties.ipynb

\item {} 
\sphinxAtStartPar
tutorial\_colors.ipynb

\end{itemize}


\chapter{Crosscheck}
\label{\detokenize{index:crosscheck}}\begin{itemize}
\item {} 
\sphinxAtStartPar
crosscheck\_colors.ipynb

\end{itemize}


\chapter{Reproducing Results}
\label{\detokenize{index:reproducing-results}}\begin{itemize}
\item {} 
\sphinxAtStartPar
2011DelgadoAtencio\_CUDAMCML.ipynb

\item {} 
\sphinxAtStartPar
\#fixme

\end{itemize}


\chapter{License}
\label{\detokenize{index:license}}
\sphinxAtStartPar
\sphinxstylestrong{SkinOptics} is released under GNU General Public License v3.0:
\begin{quote}

\begin{DUlineblock}{0em}
\item[] This program is free software: you can redistribute it and/or modify
\item[] it under the terms of the GNU General Public License as published by
\item[] the Free Software Foundation, either version 3 of the License, or
\item[] (at your option) any later version.
\end{DUlineblock}

\begin{DUlineblock}{0em}
\item[] This program is distributed in the hope that it will be useful,
\item[] but WITHOUT ANY WARRANTY; without even the implied warranty of
\item[] MERCHANTABILITY or FITNESS FOR A PARTICULAR PURPOSE.  See the
\item[] GNU General Public License for more details.
\end{DUlineblock}

\begin{DUlineblock}{0em}
\item[] You should have received a copy of the GNU General Public License
\item[] along with this program.  If not, see \textless{}\sphinxurl{https://www.gnu.org/licenses/}\textgreater{}.
\end{DUlineblock}
\end{quote}


\chapter{Version History}
\label{\detokenize{index:version-history}}
\begin{DUlineblock}{0em}
\item[] 0.1.0 \sphinxhyphen{} release date: 30/August/2024
\end{DUlineblock}


\chapter{Citing SkinOptics}
\label{\detokenize{index:citing-skinoptics}}
\sphinxAtStartPar
If \sphinxstylestrong{SkinOptics} is being useful for a presentation, publication or other material, please consider citing it.

\sphinxAtStartPar
Suggestion:

\begin{DUlineblock}{0em}
\item[] V. P. G. Lima, “SkinOptics documentation \sphinxhyphen{} version 0.1.0,” (2024).
\item[] Available at \textless{}\sphinxurl{https://skinoptics.readthedocs.io/}\textgreater{}.
\end{DUlineblock}

\sphinxAtStartPar
BibTex entry:
\begin{quote}

\begin{DUlineblock}{0em}
\item[] @misc\{Lima2024,
\item[]
\begin{DUlineblock}{\DUlineblockindent}
\item[] author = \{V. P. G. Lima\},
\item[] title = \{\{SkinOptics documentation \sphinxhyphen{} version 0.1.0\}\},
\item[] year  = \{2024\},
\item[] url   = \{\sphinxurl{https://skinoptics.readthedocs.io/}\}
\end{DUlineblock}
\item[] \}
\end{DUlineblock}
\end{quote}

\sphinxAtStartPar
Please do remember to cite \#fixme

\sphinxAtStartPar
\#fixme checar todas as fórmulas
\#fixme terminar colors
\#fixme ‘float or np.ndarray’ por ‘float, np.arrady’


\chapter{Credits}
\label{\detokenize{index:credits}}
\begin{DUlineblock}{0em}
\item[] This documentation was written in reStructuredText and built with \sphinxhref{https://www.sphinx-doc.org/en/master/}{Sphinx}.
\item[] It is displayed with the \sphinxhref{https://github.com/pradyunsg/furo}{Furo Theme}.
\item[] The favicon image is from \sphinxhref{https://openmoji.org/}{OpenMoji}.
\end{DUlineblock}


\chapter{References}
\label{\detokenize{index:references}}
\begin{DUlineblock}{0em}
\item[] References consulted to implement \sphinxstylestrong{SkinOptics} tools are stated in the API reference.
\end{DUlineblock}


\chapter{Author Information}
\label{\detokenize{index:author-information}}
\begin{DUlineblock}{0em}
\item[] Victor Porto Gontijo de Lima
\item[] Brazilian 🟩 🟨 🟦 ⬜
\item[] Research, Development \& Innovation
\end{DUlineblock}

\begin{DUlineblock}{0em}
\item[] MSc student in Theoretical and Experimental Physics
\item[] University of São Paulo (USP)
\item[] contact e\sphinxhyphen{}mail: victorporto@ifsc.usp.br
\end{DUlineblock}

\begin{DUlineblock}{0em}
\item[] Physicist (Technician)
\item[] Federal University of São Carlos (UFSCar)
\item[] contact e\sphinxhyphen{}mail: victor.lima@ufscar.br
\end{DUlineblock}

\sphinxAtStartPar
If you find any mistakes in the scripts or have any suggestions, feel free to send me a message! :)


\chapter{Funding}
\label{\detokenize{index:funding}}
\sphinxAtStartPar
The development of \sphinxstylestrong{SkinOptics} was only possible due to multiple sources of finance.

\begin{DUlineblock}{0em}
\item[] 1\sphinxhyphen{} Brazilian government funding agencies:
\end{DUlineblock}
\begin{itemize}
\item {} 
\sphinxAtStartPar
process number 88887.631088/2021\sphinxhyphen{}00 (PROEX \sphinxhyphen{} CAPES) {[}MSC scholarship, 2021 \sphinxhyphen{} 2023{]}

\item {} 
\sphinxAtStartPar
process number 465360/2014\sphinxhyphen{}9 (\sphinxhref{https://www.ifsc.usp.br/cepof/}{INCT} \sphinxhyphen{} CNPq) {[}Biophotonics Group{]}

\item {} 
\sphinxAtStartPar
process number 2014/50857\sphinxhyphen{}8 (\sphinxhref{https://www.ifsc.usp.br/cepof/}{INCT} \sphinxhyphen{} FAPESP) {[}Biophotonics Group{]}

\item {} 
\sphinxAtStartPar
process number 2013/07276\sphinxhyphen{}1 (\sphinxhref{https://www.ifsc.usp.br/cepof/}{CePID CePOF} \sphinxhyphen{} FAPESP) {[}Biophotonics Group{]}

\end{itemize}

\begin{DUlineblock}{0em}
\item[] 2\sphinxhyphen{} My grandmas Maria do Carmo and Eva, my grandpas Severo and Josa, my mom Rosana, my dad Silvério and my sister Amanda Lima, who have always provided me material conditions for living and studying.
\end{DUlineblock}

\begin{DUlineblock}{0em}
\item[] 3\sphinxhyphen{} My work with management, teaching, research and university extension as a brazilian government employee:
\end{DUlineblock}
\begin{itemize}
\item {} 
\sphinxAtStartPar
currently at the Federal University of São Carlos (UFSCar) {[}2024 \sphinxhyphen{} today{]}

\item {} 
\sphinxAtStartPar
previously at the Federal Institute of Education, Science and Technology of Brasília (IFB) {[}2023 \sphinxhyphen{} 2024{]}

\end{itemize}

\begin{DUlineblock}{0em}
\item[] 4\sphinxhyphen{} My boyfriend Luís Fernando Brinatti with whom I share home, housework and future.
\end{DUlineblock}


\chapter{Acknowledgments}
\label{\detokenize{index:acknowledgments}}
\sphinxAtStartPar
All things I accomplish are only possible because I have a community of hard\sphinxhyphen{}working and caring people around me.

\sphinxAtStartPar
I would like to thank…
\begin{itemize}
\item {} 
\sphinxAtStartPar
my family and friends, who I love so much and who keep me breathing.

\item {} 
\sphinxAtStartPar
my MSc supervisor Lilian Tan Moriyama, who is guiding me throughout the amazing and intriguing field of Biophotonics and Biomedical Optics.

\item {} 
\sphinxAtStartPar
some colleagues Thereza Cury Fortunato, Otávio Perez Palamoni, Sofia Maria Brandão Santos, Semira Silva de Arruda, Johan Sebastian Diaz Tovar, Maria Júlia de Arruda Mazzotti Marques, Alessandra Ramos Lima, Cristina Kurachi and Vanderlei Salvador Bagnato for all aid and discussions.

\item {} 
\sphinxAtStartPar
my BSc supervisor Mariana Penna Lima Vitenti, who guided my first steps on the field of Scientific Programming.

\item {} 
\sphinxAtStartPar
my friend Arthur Willian Soares Alves, who always helps me to deal with computer\sphinxhyphen{}related issues.

\end{itemize}


\renewcommand{\indexname}{Python Module Index}
\begin{sphinxtheindex}
\let\bigletter\sphinxstyleindexlettergroup
\bigletter{s}
\item\relax\sphinxstyleindexentry{skinoptics.absorption\_coefficient}\sphinxstyleindexpageref{03_absorption_coefficient:\detokenize{module-skinoptics.absorption_coefficient}}
\item\relax\sphinxstyleindexentry{skinoptics.anisotropy\_factor}\sphinxstyleindexpageref{06_anisotropy_factor:\detokenize{module-skinoptics.anisotropy_factor}}
\item\relax\sphinxstyleindexentry{skinoptics.colors}\sphinxstyleindexpageref{07_colors:\detokenize{module-skinoptics.colors}}
\item\relax\sphinxstyleindexentry{skinoptics.dataframes}\sphinxstyleindexpageref{02_dataframes:\detokenize{module-skinoptics.dataframes}}
\item\relax\sphinxstyleindexentry{skinoptics.refractive\_index}\sphinxstyleindexpageref{05_refractive_index:\detokenize{module-skinoptics.refractive_index}}
\item\relax\sphinxstyleindexentry{skinoptics.scattering\_coefficient}\sphinxstyleindexpageref{04_scattering_coefficient:\detokenize{module-skinoptics.scattering_coefficient}}
\item\relax\sphinxstyleindexentry{skinoptics.utils}\sphinxstyleindexpageref{01_utils:\detokenize{module-skinoptics.utils}}
\end{sphinxtheindex}

\renewcommand{\indexname}{Index}
\printindex
\end{document}